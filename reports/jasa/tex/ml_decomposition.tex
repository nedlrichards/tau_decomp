\documentclass[preprint,NumberedRefs]{JASA}

\begin{document}

\title[Mixed layer tilt and spice]{North Pacific upper ocean spice and isopycnal tilt sound-speed structures and their effects on acoustic propagation}
\author{Edward L. Richards}
\email{edwardlrichards@gmail.com}
\affiliation{Ocean Sciences, University of California Santa Cruz, Santa Cruz, California 95064, USA}
\author{John A. Colosi}
\affiliation{Department of Oceanography, Naval Postgraduate School, Monterey, California 93943, USA}

\preprint{E. L. Richards, JASA}	%if you want want this message to appear in upper right corner of title page

\date{\today}


\begin{abstract}

\end{abstract}

\maketitle

\section{\label{sec:intro} Introduction}

The paper is organized as follows: Section~\ref{sec:decompostion} discusses the dynamic decompostion of the observed sound speed field into spice and tilt structures.

\section{North Pacific transcet}
This study is based on a 970 km SeaSoar conductivity, salinity and depth (CTD) transect taken over 4 days in the North Pacific\citep{cole2010seasonal}, Fig. \ref{fig:transcet}. The observation extended from the sea surface to 430 m depth, with an average cycle length of 1.28 km. The SeaSoar observations were first gridded with 1 km horizontal and 0.5 m vertical resolution. The transcet is largely parrellel to the 16 $^\circ C$ isotherm, calculated from the World Ocean Atlas decadal spring average.

\begin{figure}
\includegraphics{../figures/transcet.png}
    \caption{\label{fig:transcet}{Location of SeaSoar transcet, with overlay of spring 50 m isotherms.}}
\end{figure}


Sound speed, $c$, computed from the CTD data using the Thermodynamic Equation of SeaWater 2010 (TEOS-10), is shown in Fig. \ref{fig:c_grid}. The mixed layer is apparent as relatively high sound speed over a strong negative sound speed gradient that extends into the deep ocean. Horizontal and vertical differences in sound speed are apparent throughout the mixed layer observations \citep{colosi2020observations}, caused by internal waves, eddies and ocean spice. This study will first separate the variations due to internal waves and eddies from ocean spice, and then compare the acoustic propagation though the separated fields.

\begin{figure}
\includegraphics{../figures/sound_speed_transcet_sld.png}
\caption{\label{fig:c_grid}{Sound speed observation of the mixed layer, the black line is the sonic layer depth.}}
\end{figure}

The straightforward physical model of the mixed layer is wind induced mixing at the air-sea interface homogenizes the vertical properties and leads to an unstratified layer. An example of a homogenized mixed layer profile was measured at 252 km, shown in Fig. \ref{fig:profiles}. Although marginal stratification is required for dynamic stability, the potential density, $\sigma$, of the mixed layer is essentially constant. The spice profile, $\gamma$, which quantifies density compensated changes in ocean properties, is also essentially constant. The sound speed is dominated by the adiabatic pressure gradient, roughly 0.16 (m/s) / m. The downward sound speed gradient creates an upward diffracting duct bounded on the top by the free surface, modeled as a perfect reflector.

\begin{figure}
\includegraphics{../figures/sld_profile.png}
    \caption{\label{fig:profiles}{(a) Potential density (b) spiciness and (c) sound speed measured at 249 km (blue curves) and 252 km (orange curves).}}
\end{figure}

The sonic layer depth (SLD), the depth of maximum $c$ above the thermocline, is the only free variable for vertically homogeneous mixed layer. Increases in the SLD increase the viability of the mixed layer duct, one metric of this is the mode cutoff frequency. The SLD is shown as horizontal grey lines in Fig. \ref{fig:profiles}(c), and the black line in Fig. \ref{fig:c_grid}. Over the entire trances, the SLD increases with range, and a linear regression fit begins at 78.5 m and has a slope of 2 m / 100 km. While the large scale trend of increasing SLD indicates the mixed layer duct is strengthening, significant variability is observed in SLD depth over the transect. The rapid variation in 3 km between the profiles at 249 and 252 km is one such example that indicates fine scale range dependence of the mixed layer is acoustically significant.

The transect shows that the mixed layer sometimes has significant vertical sound speed structure. This structure can arise from from ocean processes that create stratification in the upper mixed layer, like eddies. Alternatively, vertical structure in sound speed can arise from density compensated salinity and temperature variation, or spice. The local spice, $\gamma$, is used as a measure of the density compensated variation \citep{klymak2015spice},
\begin{equation}
    \gamma=\textrm{sgn}(T-\bar{T}) \sqrt{\alpha_0^2(T-\bar{T})^2 +\beta_0^2(S-\bar{S})^2},
    \label{eq:gamma}
\end{equation}
where $\bar{T}$ and $\bar{S}$ are the means of temperature and salinity at constant density, and a representative ($\bar{T}$, $\bar{S}$) value defines $\alpha_0=\partial \sigma / \partial T$ and $\beta_0=\partial \sigma / \partial S$. $\gamma$ is potential variable that represents linearized distance in density units along (T, S) isopycnals. Density compensation occurs when salinity and temperature increase or decrease proportionally, these changes in both variables are reinforcing in sound speed. The effect of spice is a dominant factor in the sound speed profile at 249 km in Fig. \ref{fig:profiles}.

The density at 40 m depth decreases with range, with a linear regression intercept at $\sigma=25.3$ kg/m$^3$ and a slope of -0.04 (kg/m$^3$) / 100 km. The decrease in density means that as range increases, mixed layer isopycnals transition to positions with stable stratification below the mixed layer.

\section{\label{sec:decomposition}Spice and tilt decompostion}
A dynamic decomposition is used to separated the observed sound speed variations into contributions from: (1) isopycnal tilt and (2) density compensated sound speed variation proportional to ocean spice. This decomposition follows the method proposed for a similar SeaSoar transect by Dzieciuch \emph{et al.}\citep{dzieciuch2004}, although spice in the mixed layer is treated differently here. The decomposition is first described for the case where spice is observed in a stratified ocean, which is applied below the mixed layer. A modified treatment of spice in the marginally stratified mixed layer is discussed next. Finally, the decomposed sound speed fields are compared.

\begin{figure}
\includegraphics{../figures/sig_tau_interp.png}
    \caption{\label{fig:cntrs}{Contours of: (a) Potential density (b) spiciness. The lowpass estimate of each stable isopycnal position is shown in red in panel (a). The lowpass estimate of the $\sigma=25.31$ (kg/m$^3$) is shown in (b), to represent common processing of all isopycnals. The full record shows isopycnals shoal and enter the mixed layer with decreasing range. The stable isopycnal estimate ends at the first location the isopycnal enters the mixed layer. Intermittent outcropping of the isopycnal are apparent as gaps in the $\sigma=25.21$ (kg/m$^3$) isopycnal position. The spice record has been imputed at the isopycnal outcroppings to form a continuous series.}}
\end{figure}

\subsection{Stratified ocean decomposition}
The dynamic decomposition begins by defining a gridded quantity, $\sigma$ is used as a simple demonstration, by interpolation of values along isopycnals. This definition allows for the generation of new fields that differs from the observed field only in isopycnal positions, which is used to remove dynamic isopycnal tilt. The isopycnal positions are defined as $\sigma(x, z) = z^{-1}(x, \sigma)$, which is discretized as $z_i(x; \sigma_i)$. A two dimensional linear interpolation is defined for $\sigma$,
\begin{equation}
    \sigma\approx\mathcal{L}(x, y; \sigma_i, z_i).
    \label{eq:lin_intr}
\end{equation}
The positions $z_i(x, \sigma)$ are then modeled as the superposition of fine scale dynamics and a stable background position $\bar{z}(x, \sigma)$. The background density field, $\bar{\sigma}$, is defined by substituting $\bar{z}(x, \sigma)$ for $z(x, \sigma)$ in Eq. \eqref{eq:lin_intr}. The estimate of $\bar{\sigma}$ is then a vertically stretched version of the observed $\sigma$ field without fine scale tilt dynamics.

The background isopycnal position is estimated with a spatial lowpass filter. This study uses a cutoff of 50 km for the estimate of $\bar{z}(x, \sigma)$, which is shown as red lines in Fig. \ref{fig:cntrs}(a). The estimated position where the isopycnal moves below the mixed layer is used as the start of the $\bar{z}(x, \sigma)$ estimate, apparent around 250 km for $\sigma=25.25$ (kg/m$^3$). A conservative estimate is made for the start position of $\bar{z}(x, \sigma)$ is made by eye. In contrast, isopycnals in the mixed layer are considered unstable and no attempt is made to estimate a background isopycnal position.

The same isopycnal surfaces can also define estimates of other field quantities without fine scale isopycnal tilt. While sound speed is ultimately the field quantity of interest, the variable of ocean spice is used here instead,
\begin{equation}
    \gamma\approx\mathcal{L}(x, z; \gamma_i(x), z_i),
    \label{eq:lin_intr_gamma}
\end{equation}
where the range dependent $\gamma_i(x)$ is measured along the $\sigma_i$ isopycnal. The accuracy of Eq. \eqref{eq:lin_intr_gamma} depends on the ocean stratification through the positions of the isopycnals, $z_i$, and has the smallest error in well stratified regions.

when $\gamma$ has significant non-linear variation over depths with little $\sigma$ gradient. In well stratified areas of the ocean, however, Eq. \ref{eq:lin_intr} and Eq. \ref{eq:lin_intr_gamma} are accurate and sufficient to compute sound speed with the inverse of Eq. \eqref{eq:gamma}, $c(S, T, p)=c(\gamma^{-1}, p)$.

The position of three isopycnals are shown in Fig. \ref{fig:cntrs}(a). Isopycnals below the mixed layer, \emph{i.e.} $\sigma = 25.29$ kg/m$^3$, are continuous in $x$ and vary slowly in $z$. Isopycnals above the mixed layer, \emph{i.e.} $\sigma = 25.21$ kg/m$^3$, may outcrop, causing discontinuities in $x$, and vary rapidly in $z$. The large scale trend of decreasing mixed layer density with $x$ means that isopycnals transition from unstable mixed layer isopycnals to stable isopycnals below the mixed layer.

The $\gamma$ field in Eq. \eqref{eq:lin_intr_gamma} allows for the additional definition of a stable background value along isopycnals, $\hat{\gamma}_i(x)$. This leads to four possible $\gamma$ fields with the combination of parameter pairs: the stable field, ($\bar{\gamma}_i$, $\bar{z}_i$); the tilt field, ($\bar{gamma_i}$, $z_i$); the spice field, ($gamma_i$, $\bar{z}_i$); and the total field, ($gamma_i$, $z_i$).

The sound speed can be computed from ($\sigma$, $\gamma$) through an inverse of Eq. \eqref{eq:gamma}. The inverse function is computed by iteration. The values of $\alpha_0$ and $\beta_0$ used for Eq. \eqref{eq:gamma} define an angle $\theta_0$ of maximum $d\gamma$ and zero $d\sigma$. The linearized estimate of ($\sigma$, $\gamma$) is first estimated from $\theta_0$. Then, iteratively, the value of $\sigma$ is first refined with the local value of $\theta$ and then gamma is corrected with $\theta_0$.

Internal waves largely followed the Garret-Munk spectrum up to the mixed layer, while eddies were the largest contribution to tilt in the mixed layer.  The decomposition works by contouring the observed field into isopycnals, and then spatially lowpass filtering isopycnal position and spice along the isopycnal to estimate an unperturbed background condition.

\subsection{Mixed layer decomposition}
Estimation of the stable isopycnal position and spice with a lowpass filter is most effective outside of the mixed layer where the ocean is well stratified in regions of significant spice variation. Two challenges in the dynamic decomposition of \citep{dzieciuch2004} arise in the mixed layer: (1) significant variations of $\tau$ occur in positions with small $\sigma$ gradients, and (2) isopycnal locations vary rapidly and may not have a stable position. Both challenges are related to the constraint of sampling the spice field at isopycnal positions imposed by the dynamic decomposition of \citep{dzieciuch2004}, and a and a linear superposition model for $\gamma$ in the mixed layer is proposed to avoid this sampling requirement.

The $\sigma$ field is only effected by (2), which is treated by truncating $\hat{z}_i(x; \sigma_i)$ when an isopycnal shows significant surface excursions or outcropping. The last stable isopycnal density is then extrapolated to the surface, resulting in a completely mixed layer. However, both challenges are significant to the construction of $\gamma$ since this interpolation depends on the isopycnal positions.



First, while the error of the interpolator for $\sigma$ can be made small by increasing the number of isopycnals, the position of isopycnals may not occur in regions of significant spice variation. While dynamic stability limits on N$^2$ prevent $\partial \sigma / \partial z$ from reaching 0, observations come close to the theoretical sampling limit where significant changes in $\partial \tau / \partial z$ occur at positions with very small changes in $\sigma$.

 Inside the marginally stratified mixed layer, however, non-linear spice variation can occur between isopycnals with very fine spacing. An example of two mixed layer $\sigma$ and $\gamma$ profiles are shown in Fig. \ref{fig:profiles} (a) and (b), which were taken 3 km apart. Both $\sigma$ profiles are well mixed above the pycnocline at around 110 m, and the vertical spacing of isopycnals is large. However the profile at 249 km shows significant spice variation above the pycnocline, and a linear interpolation for $\gamma$ that uses isopycnal positions will be undersampled in this region.

Secondly, the low-passed isopycnal positions are an estimate of where the isopycnals would be in the absence of isopycnal tilt from internal waves and eddies. While isopycnals are observed in the mixed layer, the position of isopycnals in the mixed layer are highly variable and yield physically implausible low-passed positions. In keeping with the well mixed assumption, this study does not attempt to estimate a stable position for isopycnals in the mixed layer.

A linear superposition model is used instead of a linear interpolator for the estimate of $\tau$ in the mixed layer. In essence, all observed $\tau$ variability that is not explained by the tilt field is included in the spice field, without any warping to account for the mean isopycnal levels. The linear superposition model is written
\begin{equation}
    \tau_{observed} = \tau_{bg} + \Delta \tau_{tilt} + \Delta \tau_{spice},
\end{equation}
with the spice field is defined as $\tau_{bg} + \Delta \tau_{spice}$, and similarly for tilt. The value of $\Delta \tau_{spice}$ is be estimated by subtracting $\tau_{tilt}$ from $\tau_{observed}$.

\begin{figure}
\includegraphics{../figures/rms_profile.png}
    \caption{\label{fig:c_rms}{The mean sound speed profile and the RMS profile of the deviation from the background field of the measured, tilt and spice fields. }}
\end{figure}

\section{\label{sec:propagation}Mixed layer acoustic propagation}
Acoustic propagation simulations are used to predict the effect of the separated tilt and spice sound speed structures on acoustic propagation in the mixed layer duct, with a focus on a single convergence zone length, Fig. \ref{fig:decomp_xmission}. A few kilometers from the source the acoustic energy in the mixed layer duct propagation is dominated by a few modes up to the first convergence zone. Outside of the source region acoustic energy in the mixed layer is monotonically decreasing.

\begin{figure}
\includegraphics{../figures/decomp_xmission.png}
    \caption{\label{fig:decomp_x}{Left panels is sound speed field, right panel is acoustic pressure. The rows are the (a) tilt, (b) spice and (c) total fields.}}
\end{figure}

Since mixed layer propagation is largely vertically uniform, the integrated mixed layer energy is used as a primary quantification of the duct.
\begin{equation}
    \textrm{Loss}_{\textrm{ML}} = -10 \, \textrm{log}_{10} \left( \frac{1}{D} \int^{D}_0 p p^* \,  dz \right)
\end{equation}
An fixed integration depth, $D=$120 m, is sufficient since energy does not persist in the strong downward refracting region below the mixed layer.

\subsection{Background field prorogation}
The background field is very similar to the range independent field predicted by the mean sound speed profile.
\begin{figure}
\includegraphics{../figures/mode_shapes.png}
    \caption{\label{fig:bg_modes}{Shapes of the three mode closest to the longest loop length pair.}}
\end{figure}

Since the only significant boundry interaction was the pressre release at $z=0$, normal modes were calculated using a tridiagonal eigenvalue decompostion. The modes with significant mixed layer interaction were identified by long loop lengths,
\begin{equation}
    l_{mn} = \frac{2 \pi}{k_n - k_m}.
    \label{eq:loop_length}
\end{equation}
The longest loop lengths were observed for mixed layer mode 1, shown in Fig. \ref{fig:bg_modes}.

The multiple mixed layer modes with the same number of zero crossings in the mixed layer interfere and cause acoustic leakage out of the mixed layer.
\begin{figure}
\includegraphics{../figures/bg_eng_loss.png}
    \caption{Acoustic energy in the background mixed layer for a 400 Hz source at 400 m depth. Significantly more loss is observed at ranges less than 300 km, shown as yellow lines. The loss is shown as thick black line, and mean plus and minus the RMS estimate are thin black lines. The 10th and 90th percentile estimates are dashed blue lines. The difference in agreement between the two percentiles and the RMS indicate a relativly small number of high loss events significantly impact the variance estimate.}
    \label{fig:bg_eng}
\end{figure}





\subsection{Mixed layer blocking features}
The mixed layer duct simulations is dominated by local features and propagation that is effected similarly over the entire propagation range. The most significant local features observed cause significant loss in the mixed layer duct, and are called blocking features. These are defined as regions where the acoustic energy falls by 3 dB over less than 5 km, relative to the background field. Blocking features are much more significant at 400 Hz than 1 kHz.

\subsection{Propagation without blocking features}

For regions without blocking features, the range dependence of the sound speed field can slowly change the acoustic energy significantly, sometimes accumulating the total effect of a blocking feature. The range dependence can also act to delay diffraction loss from the channel.

The acoustic energy in the mixed layer at low frequencies is significantly affected by diffraction, which is apparent in range independent simulations. In this study, the background field is used to compute the diffraction effect in a slowly range dependent environment.

\bibliographystyle{jasanum2}
\bibliography{eRichards_master}

\end{document}
