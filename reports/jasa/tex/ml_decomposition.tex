\documentclass[preprint,NumberedRefs]{JASA}
\usepackage{multirow}
\usepackage{booktabs}

\begin{document}
\title[Mixed layer tilt and spice]{North Pacific upper ocean spice and isopycnal tilt sound-speed structures and their effects on acoustic propagation}
\author{Edward L. Richards}
\email{edwardlrichards@gmail.com}
\affiliation{Ocean Sciences, University of California Santa Cruz, Santa Cruz, California 95064, USA}
\author{John A. Colosi}
\affiliation{Department of Oceanography, Naval Postgraduate School, Monterey, California 93943, USA}

\preprint{E. L. Richards, JASA}	%if you want this message to appear in upper right corner of title page

\date{\today}


\begin{abstract}
Mixed layer acoustic propagation is predicted for fields that approximate two dynamically separate sources of sound speed variation: the tilting of isopycnals and “spice,” density compensated temperature and salinity changes. These fields are decomposed from a transect measurement over approximately 1000 km of the northeast Pacific Ocean. At a frequency with only one mixed layer mode, both fields contain blocking features, defined as sound speed structures less than 5 km wide that create significant acoustic energy loss. The transect shows two distinct regions where blocking features are caused by tilt or spice, indicating the acoustic importance of these dynamic processes depends on location. Statistics of mixed layer acoustic energy are investigated at two frequencies, averaging one or three mixed layer modes, both with and without blocking features. The spice field was found to produce higher loss than the tilt field. Both fields were found to create more loss at lower frequencies when locations with blocking features were included, and more loss at higher frequencies without blocking features. Mixed layer mode amplitudes show more mixed layer modes at the higher frequency increases mode coupling, which both reduces loss at localized features and increases diffuse loss.

\end{abstract}

\maketitle

\section{\label{sec:intro} Introduction}
Strong vertical mixing from by wave breaking at the air-sea interface homogenizes the upper ocean and forms a mixed layer. A vertically homogonous mixed layer has an adiabatic sound speed gradient of roughly 0.016 m/s, and creates an acoustic duct bounded on the top by the acoustically reflecting sea surface. The acoustic characteristics of the mixed layer duct are distinct from a shallow water duct with the same sound speed structure because it interacts with the deep ocean sound speed channel\citep{porter93}, especially at lower frequencies. The mixed layer duct is expected to be stable at frequencies above the lower limit\citep{Urick1982Prop} and range independent models predict long range acoustic propagation is possible within the duct. Range or time dependent measurements of the mixed layer show that it is has highly variable soundspeed\citep{cole2010seasonal,rudnick1999compensation,klymak2015}, which can significantly alter or completely block duct propagation\citep{colosi2020observations,colosi21}. This study investigates the relative impact of dynamic processes on the mixed layer duct and decomposes a mixed layer transect observation into tilt and spice fields following Dzieciuch \emph{et al.}\citep{dzieciuch2004}. Parabolic equation acoustic simulations\citep{collins93} show both fields can alter or block mixed layer propagation, and the relative importance of these dynamics dependens on transect location.

While vertical mixing is expected to stabilize the vertical mixed layer properties, range dependence is also expected as the bulk properties of the surface ocean changes\citep{ferrari2000}. The variation of temperature and salinity in the upper ocean tends to be highly correlated, and the relative effects of these changes tend to moderate the overall effect on density. On large scales, temperature varies at twice the rate of the salinity moderation. At shorter scales, however, the temperature and salinity variations often compensate completely. Density compensated waters are either relatively warm and salty or cool and fresh, and this variation is termed ocean "spice"\citep{munk1981evolution}. The two scales of density variation are rectified at ocean fronts were density changes abruptly. Similarly, density compensated spice variation tends to be front like, although mixed layer observations do show significant vertical variation in spice near the fronts.

Importantly for the acoustic problem, density compensated temperature and salinity variations are reenforcing in sound speed. Spice variations can significantly change acoustic propagation, although spice variability and its accompanying impact on sound speed changes significantly with geographical location\citep{colosi12,colosi13,murat2021}. The importance of spice in an acoustic environment and the variation of spice over time remain topics of active research.

%Measurements of the effect of ocean spice on sound speed is complicated because it is necessary to separate out sound speed variations that accompany density variations. Internal waves and ocean eddies are well studied examples of dynamic processes with a density signature that tilts isopycnals. Temperature and salinity measurements over space or time are required to separate out spice and tilt sources of sound speed variation.

%which can be demonstrated using representative sound speed variations\citep{colosi21}. Alternatively, dynamic decompositions can be used that split the observed sound speed field into variations that are accompanied by density signatures and those that are density compensated\citep{dzieciuch2004}. Sound speed variations that have a density signature are dynamically active under the influence of gravity, which tends to tilt isopycnals. This effect is especially marked in the mixed layer which has low buoyancy frequencies. The sound speed field that is described well using isopycnals is termed the tilt field and is expected to be the result of internal waves, eddies and submesoscale ocean dynamics.

%While ocean temperature measurements are the simplest to make and in many cases are sufficient for ocean acoustic simulations, ocean spice measurements are complicated by the need for salinity as well. The salinity measurements are used to identify the amount of density compensation in the observed temperature variations. Ocean moorings with pumped salinity and temperature measurements have been used to measure spice on the shelf break, Philippine Sea, and arctic ocean. These separate locations have shown significant variations in spice, and also demonstrated the difficulty of predicting the importance of ocean spice from the complexity of large-scale ocean dynamics.

While the transect observation and accompanying acoustic transmissions considered here were previously analyzed\citep{colosi2020observations}, this paper extends this analysis by decomposing the observed sound speed variation into spice and tilt contributions. Transect measurements of spice with high vertical and horizontal resolution allow the acoustic simulation of the mixed layer with the proper spatial scales of spice and tilt sound speed variation. This allows for a separation of ocean spice that preserved the spatial scales of the features, although the temporal variation is not well characterized. The dynamic decomposition of a transect measurement of ocean sound speed was first made by Dzieciuch \emph{et al.}\citep{dzieciuch2004} from a similar SeaSoar measurement.

An mixed layer acoustic duct propagation scenario is simulated for the decomposed and observed fields at 400 and 1000 Hz with the parabolic equation model RAM\cite{collins93}. The average transcet sound speed profile predicts one and three modes trapped in the mixed layer duct at these frequencies, respoectivly, which is used to study the effects of mode coupling from sound speed range dependence. A transmission length of one convergence zone\citep{jensen2011computational}, roughly 50 km in the subtropical Pacific Ocean, is chosen to remove the consideration of coupling into the duct at convergence zones\citep{colosi2020observations}.

Acoustic simulations were made for source positions every 10 km along the transcet. While most results are reported as mixed layer energy statistics, discrete blocking features\citep{colosi2020observations} observed throughout the transcet are treated separatly here. Blocking features are transcet ranges where almost all of the acoustic energy was lost from the duct, and this high loss is usally observed for all source ranges that sample the blocking feature. Blocking features appeared for both decomposed fields and were most evident at 400 Hz. Mixed layer statistics are discussed both with and without blocking features. Differences in the relative loss across the two frequencies show these events represent a distinct loss mechanism compared to non-localized range dependent loss that is assumed to exist at all ranges.

The paper is organized as follows: Section~\ref{sec:transcet} describes the oceanographic mixed layer observations that motivate this study. Section~\ref{sec:decomposition} discusses the dynamic decomposition that separated the observed sound speed field into spice and tilt fields. The acoustics of the separated sound speed fields are compared with the observed field in Sec.~\ref{sec:propagation}. The statistics of mixed layer energy loss are used to compare a range of source positions over the complete transect length. Finally, the study conclusions are summarized in Sec.~\ref{sec:conclusion}.

\section{North Pacific transect}\label{sec:transcet}
\begin{figure}
\includegraphics{../figures/transcet.png}
    \caption{\label{fig:transcet}{Location of the east to west SeaSoar transect in red. Isotherms in $^\circ$C are computed at 50 m depth from the World Ocean Atlas decadal spring temperature average\citep{WOA}.}}
\end{figure}

This study is based on a 970 km SeaSoar conductivity, salinity, and depth (CTD) transect taken over 4 days in the northeast Pacific ocean\citep{cole2010seasonal}, Fig. \ref{fig:transcet}. The 970 km long transect is largely parallel to the $16 \ ^\circ$C isotherm at a mixed layer depth of 50 m from the World Ocean Atlas spring decadal average\citep{WOA}. Small differences in slope between the track and the isotherm predict a slight warming of the mixed layer over the transect.

The SeaSoar conductivity and temperature (CTD) observation\citep{colosi2020observations} extend from the sea surface to 430 m depth, with an average cycle length of 1.28 km. These observations were first interpolated from the vehicle's sawtooth path to a grid with 1 km horizontal and 0.5 m vertical resolution, and WOA climatology is appended to a depth of 4000 m\citep{WOA}. Sound speed, $c$, computed from the CTD data using the Thermodynamic Equation of Seawater 2010 (TEOS-10), is shown in Fig. \ref{fig:c_grid}. The mixed layer is apparent as relatively high sound speed over a strong negative sound speed gradient that extends into the deep ocean. Horizontal and vertical variability in sound speed is apparent throughout the mixed layer observations from internal waves, fronts, eddies, and ocean spice\citep{colosi2020observations}.

\begin{figure}
\includegraphics{../figures/sound_speed_transcet_sld.png}
\caption{\label{fig:c_grid}{Sound speed observation of the mixed layer, the black line is the sonic layer depth.}}
\end{figure}

An example of a largely homogonous mixed layer profile was measured at 252 km transect range, Fig. \ref{fig:profiles}. A marginal stratification is imposed on the observations to ensure dynamic stability\citep{barker2017stabilizing}. However, the potential density, $\sigma$, of the mixed layer is essentially constant. The spice profile, $\gamma$, quantifies density compensated changes in ocean properties and is also essentially constant with a small decrease around 35 m depth. The sound speed is fit to a linear gradient of 0.0153 1/s, largely determined by the adiabatic gradient. The downward sound speed gradient creates an upward diffracting duct bounded on the top by the free surface, modeled as a perfect reflector.

\begin{figure}
\includegraphics{../figures/sld_profile.png}
    \caption{\label{fig:profiles}{(a) Potential density, $\sigma$ (b) spice, $\gamma$ and (c) sound speed, $c$, measured at 249 km (blue curves) and 252 km (orange curves).}}
\end{figure}

The sonic layer depth (SLD), the depth of maximum $c$ above the thermocline, is the only free variable for vertically homogeneous mixed layer. A deeper SLD indicates a stronger duct that traps lower acoustic frequencies and higher angles of propagation. The SLD varies between 132 m and 32.5 m over the transcet, shown as black and grey lines in Figs. \ref{fig:c_grid} and \ref{fig:profiles} (c). The SLD increases with range over the transect, a linear regression fit has a 78.5 m intercept and a slope of 2 m / 100 km. While the large-scale trend of increasing SLD indicates the mixed layer duct is strengthening, significant variability is also observed in SLD depth over the entire transect. The rapid variation in between the profiles at 249 and 252 km Fig. \ref{fig:profiles} is one example that indicates fine scale range dependence of the mixed layer is acoustically significant.

While the SLD completely describes a vertically homogenous mixed layer, the mode cutoff frequency\citep{Urick1982Prop} is a more complete metric that describes non-adiabatic sound speed gradients,
\begin{equation}
    f_{min}(n) = \frac{3(4n-1)}{16} \sqrt{\frac{c_0^3}{2h^3} \left( \frac{\partial c}{\partial z} \right) ^ {-1}},
    \label{eq:f_cutoff}
\end{equation}
which has units of Hz. The SLD is $h$, and a refence sound speed of $c_0 = 1500$ m/s is assumed. Several observations from Colosi and Rudnick (2020)\cite{colosi2020observations} on mode cutoff frequency are sumarized here. The observed sound speed gradients varies from 0 to above 0.1, and are on average higher than the adiabatic gradient of 0.016 1/s. The first four cutoff frequencies for the transcet average sound speed profile are 210, 480, 760 and 1040 Hz. There are significant variations in the cutoff frequency due to both changes in SLD and the sound speed gradient, and the cutoff for mode 1 raises past 1 kHz at several locations along the transcet.

While the mode cutoff frequency is a valuble meteric for predicting mixed layer duct viability from oceanagraphic observations, the assumption of a linear sound speed does not capture the full variety of mixed layer profiles. The profile at 249 km transect range, Fig. \ref{fig:profiles} (c), demonstrates the significant changes of vertical sound speed structure observed in the mixed layer. These sound speed changes may be caused by the tilting of isopycnals, or spice. A global definition of density compensated variation exists, termed "spiceness"\citep{mcdougall2015spiciness}, but a local definition of spice is prefered when the temperature and salinity variation is small enough that the equation of state of seawater is approximatly linear\citep{ferrari2000}. The profile of local spice, $\gamma$, is used as a measure of the density compensated variation \citep{klymak2015spice} of ocean properties,
\begin{equation}
    \gamma=\textrm{sgn}(T-\bar{T}) \sqrt{\alpha_0^2(T-\bar{T})^2 +\beta_0^2(S-\bar{S})^2},
    \label{eq:gamma}
\end{equation}
where $\bar{T}$ and $\bar{S}$ are the means of temperature and salinity at constant density, and a representative ($\bar{T}$, $\bar{S}$) value defines $\alpha_0=\partial \sigma / \partial T$ and $\beta_0=\partial \sigma / \partial S$. The potential variable $\gamma$ represents linearized distance in density units along isopycnals. Density compensation occurs when salinity and temperature both increase or decrease change with relative angle $\theta$ in the (T, S) diagram. These changes in both variables are compensating in density but reinforcing in sound speed. For the profiles at 249 km in Fig. \ref{fig:profiles}, the essentially constant potential density shows the high variablity in the spice causes the non-linear sound speed variation above the thermocline .


\section{\label{sec:decomposition}Spice and tilt decomposition}
A dynamic decomposition is used to separate the observed sound speed variations into contributions from isopycnal tilt and spice. This decomposition largely follows the method proposed for a similar SeaSoar transect by Dzieciuch \emph{et al.}\citep{dzieciuch2004}, however, spice in the mixed layer is treated differently here. The decomposition is first described for the case of the stratified region below the mixed layer. A linear superposition model for spice is then proposed for the marginally stratified mixed layer, and the results of the two methods are compared. Finally, the root mean square sound speed variations of the decomposed components are compared.

\begin{figure}
\includegraphics{../figures/sig_tau_interp.png}
    \caption{\label{fig:cntrs}{Contours of: (a) potential density, $\sigma$ (b) spice, $\gamma$. The low-pass estimate of each stable isopycnal position is shown in (a) as red curve. The isopycnals shoal and enter the mixed layer with decreasing range, where the isopycnal position becomes highly variable with range. The stable isopycnal estimate ends at the first location the isopycnal shows significant decorrelation with lower isopycnals. The low-pass estimate of spice for the $\sigma=25.75$ (kg/m$^3$) is shown in (b) as a red curve, indicative of processing for all isopycnals.}}
\end{figure}

\subsection{Stratified ocean decomposition}

%The density at 40 m depth decreases with range, with a linear regression intercept at $\sigma=25.3$ kg/m$^3$ and a slope of -0.04 (kg/m$^3$) / 100 km. The decrease in density means that as range increases, mixed layer isopycnals transition to positions with stable stratification below the mixed layer.

The dynamic decomposition begins by defining a gridded quantity, $\sigma$ is used for demonstration, by interpolation of values between isopycnals. The isopycnal $z$ position is defined $\sigma(x, z) = z^{-1}(x, \sigma)$ and discretized as $z_i(x; \sigma_i)$. A two-dimensional linear interpolation is defined for $\sigma$,
\begin{equation}
    \sigma(x,y)\approx\mathcal{L}(x, y; \sigma_i, z_i).
    \label{eq:lin_intr}
\end{equation}
The isopycnal positions $z_i(x, \sigma)$ are modeled as the superposition of fine scale dynamics on a stable background position $\bar{z}(x, \sigma)$. The background density field, $\bar{\sigma}$, is defined by substituting $\bar{z}(x, \sigma)$ for $z(x, \sigma)$ in Eq. \eqref{eq:lin_intr}. The estimate of $\bar{\sigma}(x,y)$ is a vertically stretched version of the observed $\sigma(x,y)$ field without fine scale tilt dynamics.

The background isopycnal position, $\bar{z}(x, \sigma)$, is estimated with a spatial low-pass filter. This study uses a cutoff of 50 km, the approximate length of one acoustic convergence zone\cite{jensen2011computational}. This estimate of $\bar{z}(x, \sigma)$ is shown as red lines in Fig. \ref{fig:cntrs}(a). The position of isopycnals inside the mixed layer are unstable and can change across the entire mixed layer depth within the horizontal resolution of observation, consistent with density changes in the mixed layer occurring at fronts. However, a number of isopycnals are observed to have multiple fronts, like the sharp increase and decrease in the depth of $\sigma=25.65$ kg/m$^3$ at the left of Fig. \ref{fig:cntrs}(a). These isopycnal positions are expected to be also temporally variable and poorly described as a perturbation from the contours low-passed position. Therefore, instead of tracking the positions of unstable isopycnals, the dynamic decomposition only estimates the mean position for isopycnals beneath the mixed layer.

Unstable isopycnals move below the mixed layer and become stable with increasing range, consistent with density increasing with the increasing temperature predicted in Fig. \ref{fig:transcet}. A conservative estimate is made for the start position of $\bar{z}(x, \sigma)$ where the position of the isopycnal becomes correlated with the denser isopycnals. For example, Fig. \ref{fig:cntrs} shows $\sigma=25.69$ (kg/m$^3$) stays below the mixed layer after 250 km. These positions where the dynamic decomposition begins tracking isopycnal positions create horizontal discontinuities in the stable background field, consistent with the front model for density change in the mixed layer.

The stable isopycnal position estimates can also define other field quantities without fine scale isopycnal tilt. While sound speed is ultimately the field quantity of interest, the variable of ocean spice is used here instead to quantify density compensated mixing,
\begin{equation}
    \gamma(x, z)\approx\mathcal{L}(x, z; \gamma_i(x), z_i),
    \label{eq:lin_intr_gamma}
\end{equation}
where $\gamma_i(x)$ is measured along the $\sigma_i$ isopycnal. The accuracy of Eq. \eqref{eq:lin_intr_gamma} depends on isopycnal spacing, with the smallest error in highly stratified regions.

The value of $\gamma_i(x)$ in Eq. \eqref{eq:lin_intr_gamma} can be modelled as the superposition of fine scale dynamic structure on a stable background, $\bar{\gamma}_i(x)$. Similar to the estimate of background isopycnal position, a low-pass filter with 50 km cutoff length is used to estimate $\bar{\gamma}_i(x)$. The combinations of the low-pass and observed values of $\sigma_i$ and $\gamma_i(x)$ can be substituted into Eqs. \eqref{eq:lin_intr} and \eqref{eq:lin_intr_gamma} to produce for $\gamma(x,z)$ fields. The stable background field is computed with ($\bar{\gamma}_i$, $\bar{z}_i$), the tilt field with ($\bar{\gamma_i}$, $z_i$), the spice field with ($\gamma_i$, $\bar{z}_i$), and the observed total field with ($\gamma_i$, $z_i$).

The sound speed is then computed from ($\sigma$, $\gamma$) through an inverse of Eq. \eqref{eq:gamma} computed by iteration. The values of $\alpha_0$ and $\beta_0$ from Eq. \eqref{eq:gamma} define the angle $\phi_0$, a linearized estimate of the angle in ($\theta$, S) space of maximum $d\gamma$ and zero $d\sigma$. The value of ($\sigma$, $\gamma$) is first estimated from $\theta_0$. Then, the value of $\sigma$ is refined with the local value of $\theta$, and finally $\gamma$ is corrected with $\theta_0$. This process can be repeated to a desired precision of ($\theta$, S), which are combined with the isopycnal position to compute a value of sound speed.

\subsection{Mixed layer decomposition}
\begin{figure}
\includegraphics{../figures/sound_speed_comp.png}
    \caption{\label{fig:c_diff}{Difference in spice field sound speed between stratified decomposition and linear superposition methods. The two decompositions are equal at the stable position of the lightest tracked isopycnal. The discontinuity in stable isopycnal position at 270 km is the last tracked position of isopycnal $\sigma=25.69$ kg/m $^3$. Significant vertical variation of sound speed is apparent in the mixed layer above the last tracked position of the isopycnal.}}
\end{figure}

Estimation of the stable isopycnal position and spice with a low-pass filter is most effective outside of the mixed layer where the ocean is well stratified in regions of significant spice variation. Two challenges in the dynamic decomposition of Dzieciuch \emph{et al.}\citep{dzieciuch2004} arise in the mixed layer: (1) significant variations of $\gamma$ occur in positions with small $\sigma$ gradients, and (2) isopycnal locations vary rapidly and may not have a stable position. A linear superposition model for $\gamma$ is proposed here for the mixed layer to avoid the requirement of sampling along isopycnals.

The linear superposition model for $\gamma$ attributes all observed variability not explained by the tilt field to the spice field, without any vertical stretching to account for the mean isopycnal levels. The linear superposition model is written
\begin{equation}
    \gamma_{observed} = \gamma_{bg} + \Delta \gamma_{tilt} + \Delta \gamma_{spice},
    \label{eq:lin_sup}
\end{equation}
with the spice field is defined as $\gamma_{bg} + \Delta \gamma_{spice}$, and similarly for tilt. The value of $\Delta \gamma_{spice}$ is estimated by subtracting $\gamma_{tilt}$ from $\gamma_{observed}$. This approach uses the background and tilt fields, which do not require vertical stretching of mixed layer isopycnals, to estimate the spice field from the observed field.

The difference between the field $\gamma_{spice}$ for the stratified decomposition of Eq.~\eqref{eq:lin_intr_gamma} and the linear superposition model are compared in Fig.~\ref{fig:c_diff}. This difference is only shown above the last stable isopycnal position, the location where these two solutions agree, where the linear superposition model is used. While the difference field often shows uniform vertical structure, there are also locations with significant vertical variation. Sampling this vertical structure with the stratified decomposition requires stable isopycnal positions well into the mixed layer, and some positions show sampling would be required in the top 25 m of the mixed layer. The linear superposition model avoids the need to introduce non-physical stable isopycnal positions to sample the observed spice variation in the mixed layer.

\subsection{Decomposed sound speed statistics}
\begin{figure}
\includegraphics{../figures/rms_profile.png}
    \caption{\label{fig:c_rms}{The mean sound speed profile and the RMS profile of the deviation from the background field of the measured, tilt and spice fields. }}
\end{figure}

The mean sound speed profile and root mean square (RMS) statistics over the entire transect are shown in Fig.~\ref{fig:c_rms}. The mean sound speed profile has a linear increase of sound speed with depth to the 80 m SLD, and is fit to a slope of 0.024 1/s. There is a sharp decrease in sound speed below the SLD at the thermocline. The RMS statistics of the observed, tilt and spice fields all have maximum values below the SLD. The shape of the spice field RMS is similar to the shape of the depth averaged spice variance reported by Ferrari and Rudnick (2000)\citep{ferrari2000}. While spice variance was shown to monotonically decrease with isopycnal density, a dpeth average spice maximum was found at the base of the mixed layer caused by variablity both along isopycnals and from crossings of tilted isopycnals. Despite having significantly reduced small scale isopycnal tilt, the dynamically separated spice field still maintained a peak RMS value at the base of the mixed layer.

The peak RMS values of the tilt and observed fields occur around 110 m depth. The tilt RMS then increases to a smaller maximum at the surface. Internal waves largely followed the Garret-Munk spectrum up to the SLD. Eddies and fronts were the largest contribution to tilt in the near the surface.

The RMS values of tilt are significantly higher than spice except around the SLD, where the maxima of the spice field is close to the tilt field minima. The spice significantly modifies the observed RMS in this region, both reducing the minimum value and moving the position up about 20 m.

\section{\label{sec:propagation}Mixed layer acoustic propagation}
The separate effects of tilt and spice are compared with the observed sound speed field and the smooth background over 60 km propagation sections. Each propagation section is separated by 10 km, a distance chosen as a compromise that samples the observed range variation with reasonable independence. Two source frequencies, 400 and 1000 Hz, are used for comparison of propagation with an average of one and three mixed layer modes, respectivly.

A representative section is shown in Fig. \ref{fig:decomp_x} for a 400 Hz acoustic source at 40 m depth. Acoustic propagation is modeled with the parabolic equation (PE) code RAM\citep{collins93}, and normal modes are used to analyze the vertical structure of mixed layer acoustic energy, Sec. \ref{ssec:bg}. The PE results, right column  of Fig. \ref{fig:decomp_x}, show some fields have significant loss and changes in vertical distribution of mixed layer acoustic energy.

\begin{figure}
\includegraphics{../figures/decomp_xmission.png}
    \caption{\label{fig:decomp_x}{Left panels is sound speed field; right panel is acoustic pressure. Rows are the: (a) background (b) tilt, (c) spice, and (d) observed fields. The region up to 7.5 km from the source has significant downgoing energy for all fields, and the first convergence zone is apparent starting 47 km from the source. Significant mixed layer loss between the source and first convergence zone is apparent as downgoing energy below the approximately 120 m deep mixed layer. The mixed layer loss is strongest around 250 and 260 km for the spice and total fields, and smaller loss is also observed in the tilt and total fields around 240 km.}}
\end{figure}

The background, tilt, spice and observed sound speed fields are shown in the left column of Fig. \ref{fig:decomp_x}. The background sound speed field varies slowly with range, with a small discontinuity around 265 km where the dynamic decomposition stops tracking an isopycnal. The background acoustic field shows canonical mixed layer acoustic propagation. The mixed layer duct is obscured by high angle propagation up to about 7.5 km from the source, and in the first convergence zone starting around 47 km. A highly absorbent layer introduced at the bottom of the PE domain suppresses bottom interactions that would otherwise contribute un-ducted energy to the mixed layer at source ranges between 7.5 and 47 km. After the removal of higher angle arrivals, the mixed layer between the source region and the first convergence zone has no external sources of acoustic energy. The mixed layer acoustic energy in this region is roughly uniform in depth with magnitude that slowly decreases with range.

The decomposition of the observed sound speed into tilt and spice fields is shown in the left column of Fig. \ref{fig:decomp_x}, panels (b) and (c). The observed SLD variation, left panel (d), is almost exclusively partitioned into the tilt field, panel (b). The most acoustically significant SLD variation is a shoaling between transect range 230 to 240 km that leads to mixed layer energy loss in the tilt field. A surface concentration of lower sound speed also appears in the tilt and observed fields between 270 and 280 km transect range. This tilt feature increases the mean sound speed gradient of the mixed layer, both strengthening the mixed layer duct and moving acoustic energy to shallower depths.

The spice sound speed field, left panel (c), shows both vertical fronts and features with significant depth variation. The most acoustically significant feature in the spice field is a high sound speed feature between 250 and 260 km transect range, two profiles from which are shown in Fig. \ref{fig:profiles}. This feature causes significant loss of acoustic energy from the mixed layer duct that is concentrated at the feature edges. There is also a marked change to the vertical pressure distribution that indicates coupling between the mixed layer normal modes. The loss of mixed layer energy is so large from this blocking feature that it significantly reduces the viability of the acoustic duct.

Similar blocking features are also observed at other ranges and can appear either in only one or multiple dynamic fields. A metric defining blocking features is defined through mixed layer energy in Sec. \ref{ssec:blocking}. Statistics of the mixed layer energy are significantly influenced by these sporadic high loss events, and so are reported both with and without these blocking features in Sec. \ref{ssec:energy}.

\subsection{Background mixed layer}\label{ssec:bg}
\begin{figure}
\includegraphics{../figures/mode_shapes.png}
    \caption{\label{fig:bg_modes}{(a) Mean sound speed profile of the background field between 230 and 280 km, (b) Shapes of mixed layer mode 1 (mode \#237) and surrounding modes at 400 Hz. The sound speed profile has a SLD of 99.5 m and is fit to a slope of 0.017 1/s. Multiple modes exist with similar shapes in the mixed layer, and all have a significant tail that extends to the compensation depth of the mixed layer around 3200 m depth.}}
\end{figure}
An example of three normal modes for the range averaged background field from 230 to 290 km are shown in Fig. \ref{fig:bg_modes}. These modes are centered around mixed layer mode 1 (MLM1), which for this profile has a mode number 237. The MLM1 is defined here as the mode with the most energy and no zero crossings in the mixed layer. It occurs within a mode number of the first peak in mode loop length\citep{jensen2011computational},
\begin{equation}
    l_{m} = \frac{2 \pi}{k_{m+1} - k_m}.
    \label{eq:loop_length}
\end{equation}
The loop length values of the modes around MLM1 are approximately 50 km and define the convergence zone length. The mixed layer duct modes create high peaks in loop length values, which can also identify mixed layer mode 2 (MLM2) and higher.

The mode shape of MLM1 is not orthogonal to neighboring modes in Fig.~\ref{fig:bg_modes} over the vertical span of the mixed layer. Instead, the coherent sum of these similarly shaped modes reenforce or diminish the amplitude of MLM1 with marginal changes to the vertical distribution of the mixed layer pressure field. The interaction between modes leads to a very long-range cycling of energy first out from and then back into the mixed layer \citep{porter93}, which cause loss of mixed layer energy for the propagation ranges considered here. This energy cycling is most significant for acoustic frequencies close to the mode 1 cutoff of Eq.~\eqref{eq:f_cutoff}.

Two quantifications of mixed layer energy are considered, the vertical integration of the PE pressure result, and a limited depth projection of a MLM onto the PE pressure result. The vertical integration of the PE pressure contains no information about the vertical distribution of energy, while the modal quantification describes the energy in the mixed layer with the same vertical distribution as the MLM. The limited depth mode projection is proposed as an alternative to mode amplitudes from coupled mode calculations that approximates mixed layer orthogonality even when multiple modes have the same number of mixed layer zero crossings.

The vertical integration of mixed layer energy is computed from the PE acoustic pressure result, $p(x, z)$, as
\begin{equation}
    \textrm{Loss} = -20 \, \textrm{log}_{10} \left( \frac{1}{D} \int^{D}_0 \left| p(x, z) \right| \,  dz \right).
    \label{eq:int_eng}
\end{equation}
Since acoustic pressure is downward refracted below the mixed layer, this energy calculation is largely insensitive to the lower boundary of integration. A fixed integration depth, $D$, of 150 m is chosen to be significantly beneath the SLD for all observations.

The limited depth mode projection computes the inner product of a MLM, $\psi(z)$, and the PE pressure field up to the $n$-th zero crossing, $z_n$,
\begin{equation}
    \textrm{Loss}_{\textrm{\vspace{0.001} MLM}} = -20 \, \left( \textrm{log}_{10} \frac{1}{z_n} \left| \int^{z_n}_0 \,  p(x, z) \ \psi(z) \,  dz \ \right| - \textrm{log}_{10} \frac{1}{z_n} \int^{z_n}_0 \, \left| \psi(z) \right| \,  dz \right).
    \label{eq:proj_eng}
\end{equation}
The normalization term accounts for the partial mode energy in the limited depth integration of Eq.~\eqref{eq:proj_eng}, which is equal to the water density for modes with no energy outside the mixed layer\citep{jensen2011computational}.

\begin{figure}
\includegraphics{../figures/bg_eng_loss.png}
    \caption{Spreading compensated acoustic energy in the range independent background mixed layer, Eq.~\eqref{eq:int_eng}, for a 400 Hz source at 40 m depth. Significantly more loss is observed at ranges less than 300 km, shown as ligth (yellow) lines. The mean loss is shown as thick black line, and the mean plus or minus the RMS estimate are dashed (blue) lines.}
    \label{fig:bg_eng}
\end{figure}
The integrated energy of Eq.~\eqref{eq:int_eng} is shown in Fig.~\ref{fig:bg_eng} for the range independent (RI) background field for a 400 Hz source at 40 m depth. This 60 km range averaged background field is used as the reference mixed layer energy loss for each source position. Sources at transect ranges before 300 km are plotted as light (yellow) lines and transect ranges beyond 300 km are dark (gray) lines. For 400 Hz, significantly more loss is predicted for many source positions at transect ranges less than 300 km. The largest energy losses before 300 km are 7 dB down from the 7.5 to 47 km, compared with a maximum of 2 dB for source positions beyond 300 km transect range. The larger loses in the background mixed layer correspond with the relatively shallow SLD observed at smaller ranges in the transect of Fig.~\ref{fig:c_grid}. The background mixed layer loss is far less significant at 1 kHz with the largest loss of 0.5 dB across the entire transect.

\subsection{Blocking features}\label{ssec:blocking}
\begin{figure}
\includegraphics{../figures/integrated_loss.png}
    \caption{Maximum mixed layer energy loss over 5 km. Source ranges with a loss greater than 3 dB (gray dotted line) over 5 km are considered to contain a blocking feature. At 400 Hz, blocking features are typically detected at all consecutive source that positions contain the same feature. At 1 kHz, blocking features have mostly smaller magnitudes and are more sporadic.}
    \label{fig:blocking}
\end{figure}

Acoustic simulations with a 400 Hz source at 40 m show some localized blocking features in the observed transect that cause significant loss in the mixed layer duct, \emph{i.e.} ranges 250 - 260 km in Fig~\ref{fig:decomp_x} (c) and (d). The large losses from sporadic blocking events significantly effect statistical characterization of transmission loss, and the results of analysis both with and without blocking features are reported. These blocking features are first identified with excess mixed layer energy loss. The mixed layer energy is computed with Eq.~\eqref{eq:int_eng} and then normalized to the RI background loss. A blocking feature is defined here as a region where the mixed layer energy falls over 3 dB in 5 km, between source ranges of 7.5 and 47 km. While a transmission range may have more than one blocking region, the maximum loss is used to charactorize the most dominant blocking feature.

The maximum excess mixed layer loss over 5 km is shown in Fig.~\ref{fig:blocking} for a source at 40 m depth. The source frequencies is 400 Hz for the top panel and 1 kHz for the bottom. At 400 Hz there are a few regions of high loss for the total, tilt, and spice fields. These regions are typically flat topped, indicating the same feature dominates mixed layer propagation for all source positions that include the feature. The mixed layer is more intermittant at higher frequencies since only some of the blocking features appear at 1 kHz. The mixed layer blocking at 1 kHz are more dependent on the mode phases at the blocking feature because they are rarely apparent for consecutive source positions. The comparison of the two frequencies show that many sound speed features only block mixed layer propagation at low frequencies, however these positions are still treated separatly for statistical charactorization at 1 kHz because they are expected to cause significant mode coupling.

The dynamic decomposition shows different partitions of the 8 observed blocking features at 400 Hz between the dynamic fields. Blocking features typically appear in the observed field and either the tilt or spice field. However, some blocking features appear in only the spice field in the range between 600 and 850 km, indicating the tilt field stabilizes the observed field at these positions. This interpretation is consistent with the low loss in the tilt field at these locations. This stabilization effect is observed at positions with concentrations of low sound speed at the surface in the tilt field, an example of this was observed in Fig. \ref{fig:decomp_x} around 270 km.

While the blocking features in the observed field are relatively evenly spaced across the transect range, these features appear in the tilt field before 450 km and in the spice field afterwards. The deepening of the mixed layer, observed with the SLD in Fig. \ref{fig:c_grid}, can explain the decrease in sensitivity to changes in mixed layer depth from internal waves. The increase in spice induced loss after 450 km indicates that spice is regionally variable over the 1000 km range of the transect.

\subsection{Mixed layer acoustic energy}\label{ssec:energy}
The effectiveness of mixed layer ducting between the decomposed sound speed fields is compared with the integrated energy and mode projected energy metrics described in Sec. \ref{ssec:bg}. Two sets of energy statistics are shown for transmission ranges with and without blocking features. These comparisons show the relative importance between intermittent high loss events and diffuse scatter and the compare spice and tilt effects on mixed layer propagation.

\begin{figure}
\includegraphics{../figures/shallow_eng.png}
    \caption{Mixed layer energy relative to the RI background for decomposed sound speed fields and a source at 40 m depth. Left column is propagation at 400 Hz, right column is at 1 kHz. The top 4 rows are statistics for the complete set of simulated source positions. The bottom 3 rows are statistics for source positions without blocking events (W/O Blocking). The integrated energy for all transmissions in the set are shown as light grey lines, half circles at -10 dB indicate a line moves beyond the plotted scale. Linear regression fits of the mean (bold) and RMS ($\pm$) at 7.5 km and 47 km are shown in the two columns right of the line plots.}
    \label{fig:shal_eng}
\end{figure}

The comparison of integrated mixed layer energy, Eq.~\eqref{eq:int_eng}, is shown in Fig.~\ref{fig:shal_eng} for a 40 m source depth. The frequency of the left column is 400 Hz, and the right is 1 kHz. The complete set of source positions is shown in the top section, while the set of source positions without blocking features are shown on the bottom. The mean and RMS loss are both are reduced by removing the blocking feature source positions. The background field indicates marginal range dependent effects, the highest RMS is only 0.6 dB at 400 Hz and 47 km range, and the discussion of range dependence will focus on the dynamic sound speed fields.

The complete set of source positions predicts small mean loss over the source region before 7.5 km at 400 Hz, and small mean gains over this region at 1 kHz. This increased coupling into the mixed layer at higher frequency is expected to exist at all regions with a wide range of high angle modes, including the convergence zones, but these are not studied here. The statistics at 47 km predict mean loss for all fields and frequencies.

The tilt field has the smallest energy loss at both frequencies. The total loss is more for 400 Hz, although the difference between 7.5 and 47 km is the same for both frequencies. The RMS at 400 Hz is significantly higher than that at 1 kHz, and this statistic is expected to be skewed by high loss events at both frequencies. The position of the upper RMS bound compared to propagation realizations indicate that an energy gain of 1.5 dB at 1 kHz is possible, while 3 dB gain at 400 Hz is a significant overprediction not realized by any source position. The tilt field observed over the total range of the transect is relatively favorable to propagation, and often causes marginal loss or even small gains relative to the RI background up to the first convergence zone.

The most mean energy loss over the mixed layer is predicted for the spice field at both frequencies, with almost 1.5 dB more is predicted at 400 Hz. No mean gain of energy is seen for the source region, although the variance of energy at 7.5 km is comparable to that of the tilt field. The RMS at 47 km is higher than the mean at 400 Hz, and while this statistic is loss dominated some fields do show marginal gains of energy. Overall, ocean spice is an important source of mixed layer energy loss and has a significantly larger effect than tilt.

The observed field has mean statistics similar to the spice field with significantly higher variance. This variance contains the blocking features of both the tilt and spice fields and is skewed by high loss events. The increased variance in the observed field is consistent across frequency and range observations which indicates the two fields often have a compounding effect on each other. In the mean, however, the combination of the tilt and spice field has less loss than the prediction of the spice alone, and in aggregate the tilt field stabilizes the total field.

The mixed layer energy statistics of regions without blocking features are shown in the bottom three rows of Fig.~\ref{fig:shal_eng}. For consistency, source positions were removed from this analysis that had blocking features at either frequency. Outside of an overal adjustment in mean and RMS loss, many of the observations from the complete statistics apply to the statistics without blocking. Two notable exceptions are: (1) the mean loss at 47 km range is the same or higher at 1 kHz without blocking features, and (2) RMS values are similar across field type and frequency. The higher mean loss at 1 kHz is a reversal from the complete field statistics but is consistent with the view that higher frequency acoustics are more sensitive to small-scale sound speed fluctuations. The consistancy in RMS loss simplifies the comparision of the mean values and clearly identifies the stablizing effect of tilt in the observed fields. The differences in statistics between the observation sets support the discussion of blocking features as disctinct from the non-localized loss mechanisms expected to be present for all source ranges.

\begin{figure}
\includegraphics{../figures/shallow_eng_proj.png}
    \caption{Projected mixed layer energy relative to the RI background for decomposed sound speed fields and a source at 40 m depth. Presentation of the result follows Fig. \ref{fig:shal_eng}, but with a y-axis lower bound of -20 dB. The mode used in the energy projection of Eq. \eqref{eq:proj_eng} is MLM1 at 400 Hz and MLM2 at 1 kHz.}
    \label{fig:shal_proj}
\end{figure}
The vertical structure of mixed layer mode energy is analyzed next by projecting the energy field onto MLM with Eq.~\eqref{eq:proj_eng}. The MLM1 is used 400 Hz and MLM2 at 1 kHz, which have the largest respective source amplitude. The energy projection results are shown in Fig.~\ref{fig:shal_proj}, which has the same presentation of rows and columns as Fig.~\ref{fig:shal_eng}. The dynamic range of these results is increased to include a lower bound of -20 dB to show the increased mean and RMS loss. One challenge in interpreting these results is that mode coupling can both increase or decrease significantly, most apparent in the relatively uncluttered 1 kHz results without blocking. This can lead to sporadic dips in mode amplitude, but this local effect is minimized by reporting statistics results from a linear regression fit.

The mean and RMS values for the mode projection show more loss compared with the total energy at all ranges and frequencies except for the background field. The background field shows the same statistics at 400 Hz as the integrated energy, consistent with the mode 1 shape of the background energy seen in Fig.~\ref{fig:decomp_x} (a). The 1 kHz statistics have a mean of approximately -1 dB for the entire range, indicating not all energy is projected onto MLM2. Like the total energy, the background statistics is essentitally flat accross the transmission range and all range depedent loss is found in the dynamic fields.

The projected and total mode energy show the same trend for the complete source set at 400 Hz, althought there is coupling out of MLM1 in exccess of total energy loss. The projected statistic has around 1 dB more mean loss for the spice and observed fields at all ranges, and approximatly 0.5 dB more RMS for these fields. Taken together with the increased mean loss indicates that some ranges are expected to couple energy into MLM2, an example of which was shown in Fig. \ref{fig:decomp_x}.

The increase in mean and RMS loss for the mode projection is significantly larger for 1 kHz. The largest mean loss is around -10 dB for the spice and observed field. When combined with the RMS values of more than 5 dB, many transmission ranges predict near complete coupling out of MLM2. The coupling serves to maintain the mixed layer energy since there is significantly less total energy loss for these fields at 1 kHz than at 400 Hz. Finally, while the tilt field consistantly shows the least loss, the RMS values show the largest increase and are the highest of all fields at 47 km. The large coupling in the tilt field at 1 kHz frequency may be an indication how this field serves to descrease loss in the observed field total energy.

Without blocking features, the projected and total energy statistics at 400 Hz are essentially identical, indicating that extreme blocking events are necassary to create mode coupling. At 1 kHz there is a substantial decrease in the projected statistics, but the mean and RMS loss values are still substantially higher than the total and mode coupling is expected to be ubiquitous in the mixed layer at 1 kHz.
%\begin{figure}
%\includegraphics{../figures/deep_blocking.png}
    %\caption{Mixed layer energy relative to the range independent background for decomposed sound speed fields and a source at 200 m depth. Presentation of the result follows Fig. \ref{fig:shal_eng}, but with a y-axis lower bound of -20 dB.}
    %\label{fig:deep}
%\end{figure}

\section{Conclusion}\label{sec:conclusion}
A transcet of the mixed layer was decomposed to produce fields with sound speed variation either from isopycnal tilt or spice. This decomposition largely followed Dzieciuch \emph{et al.}\citep{dzieciuch2004}, but introduced a linear superposition model for the mixed layer. The linear superposition model allows for more of the observed spice variation in the upper mixed layer, but does not compensate for isopycnal motion like the statified decomposition used below the mixed layer. The tilt and spice fields show clear partitioning of observed mixed layer sound speed features. For example, much of the small scale change of mixed layer depth was partitioned into the tilt field and follows the prediction from Garret-Munk internal waves. Similarly, much of the front like vertical structure of sound speed in the mixed layer was partitioned into the spice field.

The different horizontal and vertical sturture of the two dynamic fields made acoustic propagation behave very differently, both between the two decomposed fields and when compared with the observed field. Acoustic propagation simulations were made at 400 and 1000 Hz, which had one or three mixed layer modes respectivly for the mean sound speed profile. There was significant struture observed in PE simulations for many different source positions, the most significant discrete structures were blocking features. Blocking features appear in some transect positions localized events that cause large energy loss, mainly seen at 400 Hz.

A mixed layer energy loss metric was introduced to quantify blocking features, which were identified in all fields. The was a clear separation in transect location between the tilt and spice field. The tilt blocking occured at transect locations with mean mixed layer depths about 10 m shallower than the locations where blocking was dominated by spice. There was also evidence of regional variation in the influence of spice, although the cause of this was not identified.

Mixed layer energy loss statistics were analyzed both with and without blocking features. Blocking features lead to high mean and RMS losses in all fields, and these losses were highest at 400 Hz. The total loss was significantly lower in transmissions without blocking features, and the losses were highest at 1 kHz. This indicated that lower frequencies were more sensitive to discrete blocking events, while higher frequencies were more sensitive to small scale and non-localized loss mechanisms. Mode amplitude analysis showed higher losses for a single mode than the total field, indicating coupling was significant for both frequencies. This difference went away at 400 Hz without blocking features, and mode coupling at this frequency is expected only in large and coherent loss events. While the mode coupling was significantly reduced without blocking features at 1 kHz, the loss values for one mode significantly exceeded that for the total mixed layer, and mode coupling is expected to be a ubiquitous. This increase in mode coupling at 1 kHz both reduces the loss at blocking features, and increases the diffuse loss in ranges without blocking features. This diffuse loss was almost negligable for the tilt field, and was significant for the spice and observed fields.

\bibliographystyle{jasanum2}
\bibliography{eRichards_master}

\end{document}
