\documentclass[preprint,NumberedRefs]{JASA}
\usepackage{multirow}
\usepackage{booktabs}

\begin{document}
\title[Mixed layer tilt and spice]{North Pacific upper ocean spice and isopycnal tilt sound-speed structures and their effects on acoustic propagation}
\author{Edward L. Richards}
\email{edwardlrichards@gmail.com}
\affiliation{Ocean Sciences, University of California Santa Cruz, Santa Cruz, California 95064, USA}
\author{John A. Colosi}
\affiliation{Department of Oceanography, Naval Postgraduate School, Monterey, California 93943, USA}

\preprint{E. L. Richards, JASA}	%if you want want this message to appear in upper right corner of title page

\date{\today}


\begin{abstract}

\end{abstract}

\maketitle

\section{\label{sec:intro} Introduction}

The paper is organized as follows: Section~\ref{sec:decompostion} discusses the dynamic decompostion of the observed sound speed field into spice and tilt structures.

\section{North Pacific transcet}
This study is based on a 970 km SeaSoar conductivity, salinity and depth (CTD) transect taken over 4 days in the northeast Pacific ocean\citep{cole2010seasonal}, Fig. \ref{fig:transcet}. The 970 km long transcet is largely parrellel to the $16 \ ^\circ$C isotherm at a mixed layer depth of 50 m according to the World Ocean Atlas spring decadal average. Small differences in slope between the track and the isotherm predict a slight warming of the mixed layer over the transcet. Strong and coherent salinity fronts are expected to be a perminant feature in the region.

\begin{figure}
\includegraphics{../figures/transcet.png}
    \caption{\label{fig:transcet}{Location of SeaSoar transcet in red. Isotherms in $^\circ$C are computed at 50 m depth from the World Ocean Atlas decadal spring temperature average.}}
\end{figure}

The SeaSoar observation (described in detail by Colosi and Rudnick \citep{colosi2020observations}) extend from the sea surface to 430 m depth, with an average cycle length of 1.28 km. These observations were first interpolated from the vehicle's sloped paths to a grid with 1 km horizontal and 0.5 m vertical resolution. Sound speed, $c$, computed from the CTD data using the Thermodynamic Equation of SeaWater 2010 (TEOS-10), is shown in Fig. \ref{fig:c_grid}. The mixed layer is apparent as relatively high sound speed over a strong negative sound speed gradient that extends into the deep ocean. Horizontal and vertical variability in sound speed is apparent throughout the mixed layer observations, caused by internal waves, fronts, eddies and ocean spice\citep{colosi2020observations}. Ocean spice, or density compensated salinity and temperature variation, can be separated in this observation from ocean processes like internal waves that cause displacement of denisty surfaces\citep{dzieciuch2004}. This study will first separate the sound speed variation associated with isopycnal tilt from density compensated variation, and then quanitify the independent effects of these components on acoustic propagation.

\begin{figure}
\includegraphics{../figures/sound_speed_transcet_sld.png}
\caption{\label{fig:c_grid}{Sound speed observation of the mixed layer, the black line is the sonic layer depth.}}
\end{figure}

The nominal model of the mixed layer is that wind induced mixing at the air-sea interface homogenizes the vertical properties and creates an unstratified surface layer. An example of a largely homogonous mixed layer profile was measured at 252 km transcet range, Fig. \ref{fig:profiles}. Although marginal stratification is required for dynamic stability, the potential density, $\sigma$, of the mixed layer is essentially constant. The spice profile, $\gamma$, quantifies density compensated changes in ocean properties and is also essentially constant. The sound speed in this case is determined by the adiabatic pressure gradient, roughly 0.16 1/s. The downward sound speed gradient creates an upward diffracting duct bounded on the top by the free surface, modeled as a perfect reflector.

\begin{figure}
\includegraphics{../figures/sld_profile.png}
    \caption{\label{fig:profiles}{(a) Potential density (b) spiciness and (c) sound speed measured at 249 km (blue curves) and 252 km (orange curves).}}
\end{figure}

The sonic layer depth (SLD), the depth of maximum $c$ above the thermocline, is the only free variable for vertically homogeneous mixed layer. The SLD is shown as a black and grey line in Figs. \ref{fig:c_grid} and \ref{fig:profiles} (c), and the SLD depth varies between 132 m and 32.5 m. Increases in the SLD depth increase the viability of the mixed layer acoustic duct. One metric of this viability is the mode cutoff frequency,
\begin{equation}
    f_{min}(n) = \frac{3(4n-1)}{16} \sqrt{\frac{c_0^3}{2h^3} \left( \frac{\partial c}{\partial z} \right) ^ {-1}},
    \label{eq:f_cutoff}
\end{equation}
which has units of Hz. A constant value of $c_0 = 1500$ m/s is assumed. The observed sound speed gradient varies between 0 and values as high as 0.096, and are on average highter than the adiabatic gradient of 0.16 1/s\cite{colosi2020observations}. The cuttoff frequencies for mode 1 for the observed SLD and sound speed gradient varies between roughly 40 and 200 Hz, which are significantly lower than the acoustic frequencies (400 and 1000 Hz) used in this study.

Over the entire transcet, the SLD increases with range, a linear regression fit has a 78.5 m interscept and a slope of 2 m / 100 km. While the large scale trend of increasing SLD indicates the mixed layer duct is strengthening, significant variability is also observed in SLD depth over the entire transect. The SLD is also shown as horizontal grey lines in Fig. \ref{fig:profiles}. The rapid variation in 3 km between the profiles at 249 and 252 km is one such example that indicates fine scale range dependence of the mixed layer is acoustically significant.

The transect in Fig. \ref{fig:c_grid} and the profiles shown in Fig. \ref{fig:profiles} demonstrate the mixed layer has significant changes of vertical sound speed structure with range. This range variation can arise from from ocean processes that tilt stratification contours in the upper mixed layer like eddies and internal waves. Alternatively, range variation in sound speed can be from changes in salinity and temperature that are density compensated, termed "spice". The profile of local spice, $\gamma$, is used as a measure of the density compensated variation \citep{klymak2015spice},
\begin{equation}
    \gamma=\textrm{sgn}(T-\bar{T}) \sqrt{\alpha_0^2(T-\bar{T})^2 +\beta_0^2(S-\bar{S})^2},
    \label{eq:gamma}
\end{equation}
where $\bar{T}$ and $\bar{S}$ are the means of temperature and salinity at constant density, and a representative ($\bar{T}$, $\bar{S}$) value defines $\alpha_0=\partial \sigma / \partial T$ and $\beta_0=\partial \sigma / \partial S$. The potential variable $\gamma$ represents linearized distance in density units along isopycnals. Density compensation occurs when salinity and temperature both increase or decrease change with relative angle $\theta$ in the (T, S) diagram. These changes in both variables are compensating in density but reinforcing in sound speed.

The effect of spice is a dominant factor in sound speed variation above the thermocline in at 249 km, while tilt processes are responsible for the approximatly 5 m change in thermocline depth between the two profiles.

The density at 40 m depth decreases with range, with a linear regression intercept at $\sigma=25.3$ kg/m$^3$ and a slope of -0.04 (kg/m$^3$) / 100 km. The decrease in density means that as range increases, mixed layer isopycnals transition to positions with stable stratification below the mixed layer.

\section{\label{sec:decomposition}Spice and tilt decompostion}
A dynamic decomposition is used to separated the observed sound speed variations into contributions from: (1) isopycnal tilt and (2) sound speed variation from ocean spice. This decomposition largely follows the method proposed for a similar SeaSoar transect by Dzieciuch \emph{et al.}\citep{dzieciuch2004}. However, spice in the mixed layer is treated differently here. The decomposition is first described for the case for the stratified region below the mixed layer, following Dzieciuch \emph{et al.}\citep{dzieciuch2004}. A linear superpostion model for spice is then proposed for the marginally stratified mixed layer, and the results of the two methods are compared. Finally, the root mean square sound speed variations of the decomposed components are compared.

\begin{figure}
\includegraphics{../figures/sig_tau_interp.png}
    \caption{\label{fig:cntrs}{Contours of: (a) potential density (b) spice. The lowpass estimate of each stable isopycnal position is shown in (a) as red curve. The isopycnals shoal and enter the mixed layer with decreasing range, where the isopycnal position becomes highly variable with range. The stable isopycnal estimate ends at the first location the isopycnal shows significant decorrelation with lower isopycnals. The lowpass estimate of spice for the $\sigma=25.75$ (kg/m$^3$) is shown in (b) as a red curve, indicative of processing for all isopycnals.}}
\end{figure}

\subsection{Stratified ocean decomposition}
The dynamic decomposition begins by defining a gridded quantity, $\sigma$ is used for demonstration, by interpolation of values between isopycnals. The isopycnal $z$ position is defined $\sigma(x, z) = z^{-1}(x, \sigma)$ and discretized as $z_i(x; \sigma_i)$. A two dimensional linear interpolation is defined for $\sigma$,
\begin{equation}
    \sigma(x,y)\approx\mathcal{L}(x, y; \sigma_i, z_i).
    \label{eq:lin_intr}
\end{equation}
The positions $z_i(x, \sigma)$ are modeled as the superposition of fine scale dynamics on a stable background position $\bar{z}(x, \sigma)$. The background density field, $\bar{\sigma}$, is defined by substituting $\bar{z}(x, \sigma)$ for $z(x, \sigma)$ in Eq. \eqref{eq:lin_intr}. The estimate of $\bar{\sigma}(x,y)$ is a vertically stretched version of the observed $\sigma(x,y)$ field without fine scale tilt dynamics.

The background isopycnal position, $\bar{z}(x, \sigma)$, is estimated with a spatial lowpass filter. This study uses a cutoff of 50 km, the approximate length of one acoustic convergence zone\cite{jensen2011computational}. This estimate of $\bar{z}(x, \sigma)$ is shown as red lines in Fig. \ref{fig:cntrs}(a). The position of isopycnals inside the mixed layer are unstable and can change across the entire mixed layer depth within the horizontal resolution of observation. 

Unstable isopycnals move below the mixed layer and become stable with increasing range, consistant with the increasing temperature and decreasing mixed layer density predicted in Fig. \ref{fig:transcet} and Fig. \ref{fig:c_grid}. A conservative estimate is made for the start position of $\bar{z}(x, \sigma)$ where the position of the isopycnal becomes decorrelated with the denser isopycnals. For example, Fig. \ref{fig:cntrs} shows this range for $\sigma=25.69$ (kg/m$^3$) is around 250 km. The stable density field, $\bar{\sigma}(x,z)$ is used for estimates of the background and spice sound speed fields, while the observed density field is used to estimate the tilt and observed sound speed fields.

The stable isopycnal position estimates can also define other field quantities without fine scale isopycnal tilt. While sound speed is ultimately the field quantity of interest, the variable of ocean spice is used here instead,
\begin{equation}
    \gamma(x, z)\approx\mathcal{L}(x, z; \gamma_i(x), z_i),
    \label{eq:lin_intr_gamma}
\end{equation}
where $\gamma_i(x)$ is measured along the $\sigma_i$ isopycnal. The accuracy of Eq. \eqref{eq:lin_intr_gamma} depends on isopycnal spacing, with the smallest error in highly stratified regions.

The value of $\gamma_i(x)$ in Eq. \eqref{eq:lin_intr_gamma} can be modelled as the superposition of finescale dynamic structure on a stable background, $\bar{\gamma}_i(x)$. Similar to the estimate of background isopycnal position, a lowpass filter with 50 km cuttoff length is used to estimate $\bar{\gamma}_i(x)$. The combinations of the lowpass and observed values of $\sigma_i$ and $\gamma_i(x)$ can be subsituted into Eqs. \eqref{eq:lin_intr} and \eqref{eq:lin_intr_gamma} to produce 4 $\gamma(x,z)$ fields. The stable background field is computed with ($\bar{\gamma}_i$, $\bar{z}_i$), the tilt field with ($\bar{\gamma_i}$, $z_i$), the spice field with ($\gamma_i$, $\bar{z}_i$), and the observed total field with ($\gamma_i$, $z_i$).

%The sound speed is then computed from ($\sigma$, $\gamma$) through an inverse of Eq. \eqref{eq:gamma} computed by iteration. The values of $\alpha_0$ and $\beta_0$ from Eq. \eqref{eq:gamma} define the angle $\phi_0$, a linearized estimate of the angle in ($\theta$, S) space of maximum $d\gamma$ and zero $d\sigma$. The value of ($\sigma$, $\gamma$) is first estimated from $\theta_0$. Then, the value of $\sigma$ is refined with the local value of $\theta$, and finally $\gamma$ is corrected with $\theta_0$. This process can be repeated to a desired precision of ($\theta$, S), which are combined with the isopycnal position to compute a value of sound speed.

\subsection{Mixed layer decomposition}
Estimation of the stable isopycnal position and spice with a lowpass filter is most effective outside of the mixed layer where the ocean is well stratified in regions of significant spice variation. Two challenges in the dynamic decomposition of Dzieciuch \emph{et al.}\citep{dzieciuch2004} arise in the mixed layer: (1) significant variations of $\gamma$ occur in positions with small $\sigma$ gradients, and (2) isopycnal locations vary rapidly and may not have a stable position. A linear superposition model for $\gamma$ is propoased here for the mixed layer to avoid the requirement of sampling along isopycnals.

The linear superposition model for $\gamma$ attributes all observed $\gamma$ variability not explained by the tilt field to the spice field, without any vertical streching to account for the mean isopycnal levels. The linear superposition model is written
\begin{equation}
    \gamma_{observed} = \gamma_{bg} + \Delta \gamma_{tilt} + \Delta \gamma_{spice},
\end{equation}
with the spice field is defined as $\gamma_{bg} + \Delta \gamma_{spice}$, and similarly for tilt. The value of $\Delta \gamma_{spice}$ is be estimated by subtracting $\gamma_{tilt}$ from $\gamma_{observed}$.

\begin{figure}
\includegraphics{../figures/sound_speed_comp.png}
    \caption{\label{fig:c_diff}{Difference in spice field sound speed between stratified decomposition and linear superposition. The two decompositions are equal at the low-pass postion of the lightest tracked isopycnal. The discontinutity in isopycnal position at 270 km is the last position of isopycnal tracking for $\sigma=25.69$ kg/m $^3$. Significant vertical variation of sound speed is apparent in the mixed layer above the last tracked position of the isopycnal.}}
\end{figure}


\begin{figure}
\includegraphics{../figures/rms_profile.png}
    \caption{\label{fig:c_rms}{The mean sound speed profile and the RMS profile of the deviation from the background field of the measured, tilt and spice fields. }}
\end{figure}

The mean sound speed profile and RMS statistics over the entire transcet are shown in Fig. \ref{fig:c_rms}. The mean sound speed profile has a linear increase of sound speed with depth to the 80 m SLD. There is a sharp decrease in sound speed below the SLD at the thermocline. The RMS statistics of the observed, tilt and spice fields all have maximum values below the SLD, below the depths most relevant to mixed layer propagation. The peak RMS values of the tilt and observed fields occur around 110 m depth, while the spice peak value is much closer to the SLD at 90 m. The RMS values of tilt are significantly higher than spice except around the SLD, where the maxima of the spice field is close to the tilt field minima. The spice significantly modifies the observed RMS  in this region, both reducing the minimum value and moving the position up about 20 m.

The tilt RMS then increases to a smaller maximum at the surface. Internal waves largely followed the Garret-Munk spectrum up to the SLD. Eddies and fronts were the largest contribution to tilt in the near the surface.

\section{\label{sec:propagation}Mixed layer acoustic propagation}
The separate effects of tilt and spice are compared with the observed sound speed field and the smooth background over 60 km propagation sections. Each propagation section is separated by 10 km, a distance chosen as a heuristic compromise that samples the observed range variation with reasonable independence. Two source frequencies, 400 and 1000 Hz, are used for comparison of low and mid frequcy propagation.

A representative section is shown in Fig. \ref{fig:decomp_x} for a 400 Hz acoustic source at 40 m depth. Acoustic propagation is modeled with the parabolic equation (PE) code RAM, and normal modes are used to analyze the vertical structure of mixed layer acoustic energy, Sec. \ref{ssec:bg}. The PE results in the right column of Fig. \ref{fig:decomp_x} show some fields predict significant loss and changes in vertical distribution of mixed layer acoustic energy.

\begin{figure}
\includegraphics{../figures/decomp_xmission.png}
    \caption{\label{fig:decomp_x}{Left panels is sound speed field, right panel is acoustic pressure. Rows are the: (a) background (b) tilt, (c) spice, and (d) observed fields. The region up to 7.5 km from the source has significant downgoing energy for all fields, and the first convergence zone is apparent starting 47 km from the source. Significant mixed layer loss between the source and first convergence zone is apparent as downgoing energy below the approxiatly 120 m deep mixed layer. The mixed layer loss is strongest around 250 and 260 km for the spice and total fields, and smaller loss is also observed in the tilt and total fields around 240 km.}}
\end{figure}

The background, tilt, spice and observed sound speed fields are shown in the left column of Fig. \ref{fig:decomp_x}, panel (a). The background sound speed field varies slowly with range, with a small discontinuity around 265 km where the dynamic decomposition stops tracking an isopycnal. The background acoustic field shows cononical mixed layer acoustic propagation. The mixed layer duct is obscured by high angle propagation up to about 7.5 km from the source, and in the first convergence zone starting around 47 km. A highly absorbent layer introduced at the bottom of the PE domain suppresses bottom interactions that would otherwise contribute un-ducted energy to the mixed layer at source ranges between 7.5 and 47 km. After the removal of higher angle arrivals, the mixed layer between the source region and the first convergence zone has no external sources of acoustic energy. The mixed layer acoustic energy in this region is roughly uniform with depth with magnitude that slowly decreases with range.

The decomposition of the observed sound speed into tilt and spice fields is shown in the left column of Fig. \ref{fig:decomp_x}, panels (b) and (c). The observed mixed layer depth variation, left panel (d), is almost exclusively partitioned into the tilt field, panel (b). The most acoustically significant variation is a shoaling of the mixed layer between transect range 230 to 240 km that leads to mixed layer energy loss in the tilt field. A surface concentration of lower sound speed also appears in the tilt and observed fields between 270 and 280 km transect range. This tilt feature increases the mean sound speed gradient of the mixed layer, both strengthening the mixed layer duct and moving acoustic energy to shallower depths.

The spice sound speed field, left panel (c), shows both vertical fronts and features with significant depth variation. The most acoustically significant feature in the spice field is a high sound speed feature between 250 and 260 km transect range. This feature causes significant loss of acoustic energy from the mixed layer duct, which is concentrated at the feature edges. There is also a significant change to the vertical pressure distribution that indicate coupling between the mixed layer normal modes. The loss of mixed layer energy is so large from this blocking feature that it significantly reduces the viability of the acoustic duct.

Similar blocking features are also observed at other ranges, and can appear either in only one or multiple dynamic fields. A metric defining blocking features is defined through mixed layer energy in Sec. \ref{ssec:energy}. Statistics of the mixed layer energy are significantly influenced by these sporadic high loss, and so are reported both including and excluding these blocking features in Sec. \ref{ssec:blocking}.

\subsection{Background mixed layer}\label{ssec:bg}
\begin{figure}
\includegraphics{../figures/mode_shapes.png}
    \caption{\label{fig:bg_modes}{(a) Mean sound speed profile of the background field between 230 and 280 km, (b) Shapes of mixed layer mode 1 (mode \#237) and surrounding modes at 400 Hz. Multiple mode exist with similar shapes in the mixed layer, and all have a significant tail that entends to the compensation depth of the mixed layer around 3200 m depth.}}
\end{figure}
The background field is used to define the expected behavior of acoustic propagation in the mixed layer, and create a reference energy loss for comparison between the dynamic sound speed fields.

An example of three normal modes of the range averaged background field from 230 to 290 km are shown in Fig. \ref{fig:bg_modes}. These modes are centered around mixed layer mode 1 (MLM1), which for this profile has a mode number 237. The MLM1 is defined here as the mode with the most energy and no zero crossings in the mixed layer. It occures within a mode number of the first peak in mode loop length\citep{jensen2011computational}, defined
\begin{equation}
    l_{m} = \frac{2 \pi}{k_{m+1} - k_m}.
    \label{eq:loop_length}
\end{equation}
The loop length values of the modes around MLM1 are approximatly 50 km and define the convergence zone length. The mixed layer duct modes create high preaks in loop length values, which can also identify mixed layer mode 2 (MLM2) and higher when they exist.

The mode shape of MLM1 is not orthogonal to neighboring modes in Fig \ref{fig:bg_modes} over the vertical span of the mixed layer. Instead, the coherent sum of these similarly shaped modes reenforce or deminish the amplitude of MLM1 with marginal changes to the vertical distribution of the mixed layer pressure field. The interaction between modes leads to a very long range cycling of energy first out from and then back into the mixed layer \citep{porter93}, which cause loss of mixed layer energy for the propagation ranges considered here. This energy cycling is most significant for acoustic frequencies close to the mode 1 cutoff of Eq. \eqref{eq:f_cutoff}.

Two quanitifications of mixed layer energy are considered, the vertical integration of the PE pressure result, and a limited depth projection of MLM1 onto the PE pressure result. The vertical integration of the PE pressure contains no information about the vertical distributution of energy, while the modal quantification describes the energy in the mixed layer that has the vertical distribution of a mixed layer mode. The limited depth mode projection is proposed as an alternative to coupled mode solutions of mode amplitude that approximates the orthogonal interpretation of mixed layer modes even when more than one mode have the same number of mixed layer zero crossings.

The vertical integraton of mixed layer energy is computed from the PE acoustic pressure result as
\begin{equation}
    \textrm{Loss}_{\textrm{ML}} = -10 \, \textrm{log}_{10} \left( \frac{1}{D} \int^{D}_0 p(x, z) \, p^* (x, z) \,  dz \right).
    \label{eq:int_eng}
\end{equation}
Since acoustic pressure is strongly downward refracted below the mixed layer, this energy calculation is largly insensitive to the lower boundry of integration. A fixed integration depth, $D=$150 m, is choosen to be significantly beneath the SLD for all observations. The limited depth mode projection computes the dot product of a MLM, $\psi$ and the PE pressure field up to the $n$-th zero crossing, $z_n$,
\begin{equation}
    \textrm{Loss}_{\textrm{P1}} = -20 \, \left( \textrm{log}_{10} \left| \frac{1}{z_n} \int^{z_n}_0 \, p(x, z) \ \psi(z) \,  dz \right| - \textrm{log}_{10} \frac{1}{z_0} \int^{z_n}_0 \, \psi^2(z) \,  dz \right).
    \label{eq:proj_eng}
\end{equation}
The normalization term acounts for the partial mode energy in the limited depth integration of Eq. \eqref{eq:proj_eng}, which is equal to the water density for modes with no energy outside the mixed layer\citep{jensen2011computational}.

\begin{figure}
\includegraphics{../figures/bg_eng_loss.png}
    \caption{Acoustic energy in the background mixed layer for a 400 Hz source at 40 m depth. Significantly more loss is observed at ranges less than 300 km, shown as yellow lines. The mean loss is shown as thick black line, and the mean plus or minus the RMS estimate are dashed black lines.}
    \label{fig:bg_eng}
\end{figure}
The integrated energy of Eq. \eqref{eq:int_eng} is shown in Fig. \ref{fig:bg_eng} for the range independent background field for a 400 Hz source at 40 m depth. This 60 km range averaged background field is used as the reference mixed layer energy loss for each source position. Sources at transcet ranges before 300 km are plotted as light yellow, and transect ranges beyond 300 km are plotted as dark blue. For 400 Hz, significantly more loss is predicted for many source positions at transcet ranges less than 300 km. The largest energy losses before 300 km are 7 dB down from the 7.5 to 47 km, compared with a maximum of 2 dB for source positions beyond 300 km transcet range. The larger loses in the background mixed layer corrispond with the relativly shallow SLD observed at smaller ranges in the transcet of Fig. \ref{fig:c_grid}. The background mixed layer loss is far less significant at 1 kHz with a largest loss of 0.5 dB across the entire transcet.

\subsection{Blocking features}
Acoustic simulations with a 400 Hz source at 40 m show some localized features in the observed transcet that cause significant loss in the mixed layer duct, \emph{i.e} ranges 250 - 260 km in Fig \ref{fig:decomp_x} (d). These are called blocking features here. The large influence of these sporadic events suggest that statistical charactorizations of transmissions should treat ranges with blocking features separately. A metric for identifing blocking features is proposed first based on excess mixed layer energy loss. Blocking features can then be identified in each of the three dynamic fields, and often occur in more than one field at the same range. Blocking features are more apparent at 400 Hz, but these positions still play a significant role when interpreting the mixed layer energy statistics at 1 kHz.

\begin{figure}
\includegraphics{../figures/integrated_loss.png}
    \caption{Maximum mixed layer energy loss over 5 km. Source ranges with a loss greater than 3 dB (gray dashed line) over 5 km are considered to contain a blocking feature. At 400 Hz, blocking features are typically detected at all consecutative source that positions contain the same feature. At 1 kHz, blocking features have smaller magntide and are more sporadic.}
    \label{fig:blocking}
\end{figure}

The definition of blocking features used here are regions where the acoustic energy falls by more than 3 dB over less than 5 km between 7.5 and 47 km, shown in Fig \ref{fig:blocking}. The energy of the mixed layer is normalized to the RI background loss of Sec. \ref{ssec:bg} to remove the effects of mixed layer duct leakage. This blocking metric is shown in Fig. \ref{fig:blocking} for 400 and 1000 Hz. At 400 Hz there are a few regions of high loss for the total, tilt and spice fields. These regions are typically flat topped, indicating the same feature dominates mixed layer propagation for all source positions that include the feature. Only some of the blocking features appear at 1 kHz, indicating the mixed layer is more viable at higher frequencies. The blocking features at 1 kHz are rarely apparent for consecutive source positions, indicating mixed layer blocking is more dependent on the mode phases at the blocking feature. The comparison of the two frequencies show that many soundspeed features only block mixed layer propagation at low frequencies, although these positions are still expected to cause significant mode coupling in higher frequency acoustics.

The dynamic decomposition shows different types of partitioning of blocking features between fields, which typically appear in the observed field and either the tilt or spice field. However, blocking features appear in only the spice field for a much of the range between 600 and 850 km, indicating the tilt field stabalized the observed field at these positions. This interpretation is consitant with the low loss in the tilt field at these locations. This stabalization effect is observed at positions with concentrations of low soundspeed at the surface in the tilt field, an example which was observed in Fig. \ref{fig:decomp_x} around 270 km.

While the 8 blocking features in the observed field are relativly evenly spaced accross the transcet range, these features appear in the tilt field before 450 km and in the spice field afterwards. The deepening of the mixed layer, observed with the SLD in Fig. \ref{fig:c_grid}, can explain the decrease in sensitivity to changes in mixed layer depth from internal waves. The increase in spice induced loss after 450 km indicates that spice is regionally variable even far from the boundries of the Pacific basin.

\subsection{Mixed layer acoustic energy}
The effectivness of mixed layer ducting is compared here between the decomposed soundspeed fields with the integrated energy and mode projected energy metrics described in Sec. \ref{ssec:modes}. Two distinct sets of energy statistics are shown, one includes the transmission ranges with blocking features and one that does not. These comparisons show the relative importance of spice and tilt on mixed layer propgation, as well as the relative importance between intermittant high loss events and diffuse scatter.

\begin{figure}
\includegraphics{../figures/shallow_eng.png}
    \caption{Background compensated mixed layer energy for decomposed sound speed fields. Left column is propgation at 400 Hz, right column is at 1 kHz. The top 4 rows are statistics for the complete set of simulated source positions. The bottom 3 rows are statistics for source positions without blocking events (W/O Blocking). The integrated energy for all transmissions in the set are shown as light grey lines, half circles at -10 dB indicate a line moves beyond the plotted scale. Linear regression fits of the mean (bold) and variance ($\pm$) at 7.5 km and 47 km are shown in the two columns after the line plots.}
    \label{fig:shal_eng}
\end{figure}

The comparision of integrated mixed layer energy, Eq. \eqref{eq:int_eng}, is shown in Fig. \ref{fig:shal_eng}. The left column shows results for 400 Hz, and the right shows 1 kHz. The complete set of source positions is shown in the top section, while the set of source positions without blocking features are shown on the bottom. The complete set of source positions shows the highest mean loss and variance, both are reduced by removing the blocking feature source positions. The background field indicates margnial range dependent effects, the highest variances is only 0.6 dB at 400 Hz and 47 km range. This means the discussion of range dependence will focus on the dynamic soundspeed fields.

The complete set of source positions predicts small mean loss over the source region of 7.5 km for 400 Hz, and small mean gains over this region for 1 kHz. This increased coupling into the mixed layer at higher frequency is expected to exist at all regions with a wide range of high angle modes including the convergence zones not studied here. After mixed layer propgation up to one converange zone length, however, the statistics at 47 km predict mean loss for all fields and frequencies.

The tilt field has the smallest energy loss at both frequencies. The total loss is more for 400 Hz, although the difference between 7.5 and 47 km is the same for both frequencies. The variance at 400 Hz is significantly higher than that at 1 kHz, and this statistic is influenced by high loss events at both frequencies. The position of the upper variance bound compared to propgation realizations indicate that an energy gain of 1.5 dB at 1 kHz is possible, while 3 dB gain at 400 Hz is a significant overprediction. The tilt field observed over the total range of the transcet is relativly favorable to propgation, and often causes maginal loss or even small gains in mixed layer energy over the length of a single convergence zone.

The most significant mean energy loss over the mixed layer is predicted for the spice field at both frequencies, with almost 1.5 dB more is predicted at 400 Hz. No mean gain of energy is seen for the source field, although the variance of energy at 7.5 km is comprable to that of the tilt field. The variance at 47 km is higher for 400 Hz, and while this statistic is loss dominated some fields show margnial gains of energy. Overall, ocean spice is an important source of mixed layer energy loss and has a significantly larger effect than tilt.

The observed field has mean statistics very similar to the spice field with significantly higher variance. This variance contains the blocking features of both the tilt and spice fields and is influenced by high loss events. The increased variance in the observed field is consitstant across frequency and range observations which indicates the two fields often have a compounding effect on each other. In the mean, however, the combination of the tilt and spice field has less loss than the prediction of the spice alone, and in aggregate the the tilt field stabalilizes the total field.

The mixed layer energy statistics of regions without blocking features are shown in the last three rows of Fig. \ref{fig:shal_eng}. For consistancy, source positions where removed from this analysis that had blocking at either frequency. The mean and RMS loss is significantly reduced for all positions and frequencies, which indicates blocking features are a significant contributor to the complete statistics. Many of the observations from the complete statistics apply to the statistics without blocking, a few exceptions are discussed here.

The mean loss at 47 km is the same or higher at 1 kHz than at 400 Hz for all fields. This is a reversal from the complete field statistics, but is consistant with the view that higher frequency acoustics are more sensitive to small-scale soundspeed fluctuations. The expected loss at 1 kHz is even higher relative to the energy at 7.5 km, which has a higher mean increase due to coupling in the source region without blocking features.

Without blocking, the RMS values at all ranges are very similar across field type and frequency. The exception to this is the spice field at 400 Hz, which has a few outliers that suggest this statistic may still be influenced by high loss events. The rms bounds fall within the cluseter of observations for all fields, which suggest a 3 dB spread in realizations at 47 km.

\begin{figure}
\includegraphics{../figures/shallow_eng_proj.png}
    \caption{Compensated mixed layer energy for decomposed sound speed fields with blocking features removed.}
    \label{fig:shal_no_block}
\end{figure}
The vertical structure of mixed layer mode energy is analyized next by projecting the energy field onto MLM1 at 400 Hz, and MLM2 at 1 kHz. The energy projection results are shown in Fig. \ref{fig:shal_no_block}, whcih has the presentation of rows and columns is the same as Fig. \ref{fig:shal_eng}. The dynamic range of these results is increased to include a lower bound of -20 dB since both the mean loss and RMS spread increased significantly. One challenge in interpreting these results is that mode amplitude can significantly increase in the analyzed range bounds, which indicates that energy has coupled into the analyzed mode. This can lead to short range dips in mode amplitude, most apparent in the relativly uncluttered 1 kHz results witout blocking. These sporatic events are expected to effect the local statistics, but this effect is minimized by reporting results from a linear regression fit.

The interpretation of the projected energy at different frequency is influenced by the number of mixed layer modes.


 regions without blocking features, the range dependence of the sound speed field can slowly change the acoustic energy significantly, sometimes accumulating the total effect of a blocking feature. The range dependence can also act to reduce diffraction loss from the channel.

The acoustic energy in the mixed layer at low frequencies is significantly affected by diffraction, which is apparent in range independent simulations. In this study, the background field is used to compute the diffraction effect in a slowly range dependent environment.



\bibliographystyle{jasanum2}
\bibliography{eRichards_master}

\end{document}
