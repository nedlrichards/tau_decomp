\documentclass[preprint,NumberedRefs]{JASA}
\usepackage{multirow}
\usepackage{booktabs}

\begin{document}
\title[Mixed layer tilt and spice]{Observations of ocean spice and isopycnal tilt sound speed structures in open ocean the mixed layer, and their impacts on acoustic propagation}
\author{Edward L. Richards}
\email{edwardlrichards@gmail.com}
\affiliation{Ocean Sciences, University of California Santa Cruz, Santa Cruz, California 95064, USA}
\author{John A. Colosi}
\affiliation{Department of Oceanography, Naval Postgraduate School, Monterey, California 93943, USA}

\preprint{E. L. Richards, JASA}	%if you want this message to appear in upper right corner of title page

\date{\today}

\begin{abstract}
Acoustic propagation in the northeastern Pacific Ocean is predicted for sound speed observations made over a 350-m deep, 1000-km long transect, which is separated into dynamic variations from the tilting of isopycnals and “spice,” density compensated temperature and salinity changes. Measured in early April of 2004, the observations reveal a roughly 100-m deep mixed layer, 100-m transition layer, and 150-m of the upper thermocline. Peak root mean square sound speed variation for the tilt field is 1.5 m/s and occurs in the transition layer, while the spice field has a peak of 0.8 m/s close to the bottom of the mixed layer. The effects of both dynamic variabilities are observed for propagation in the mixed layer and also into and out of the transition layer. Overall, the separate dynamic sound speed fields have distinct effects in each scenario: the spice field causes the most degradation of the mixed layer duct, while the tilt field has causes smaller losses and stabilizes the duct in many locations. The acoustic loss occurs both in discrete regions of mixed layer duct blocking and a background of diffuse energy loss in both decomposed fields, and these loss mechanisms have different relative importance based on frequency.
\end{abstract}

\maketitle

\section{\label{sec:intro} Introduction}
Both mechanical mixing from wind forcing at the air-sea interface and convection from surface cooling homogenize the upper ocean and form a mixed layer\citep{cole2010seasonal}. A vertically homogonous mixed layer has an adiabatic sound speed gradient of 0.016 m/s that creates an acoustic duct bounded on the top by the reflecting sea surface. The acoustic characteristics of this mixed layer acoustic duct (MLAD) are distinct from a shallow water duct with the same sound speed structure because it interacts through diffraction with the deep ocean acoustic channel\citep{porter93}, especially at lower frequencies. However, the MLAD is expected to be stable at frequencies above its lower limit\citep{Urick1982Prop} and range independent models predict long range acoustic propagation. The MLAD is an acoustically significant feature of upper ocean propagation even in scenarios with either the source or receiver below the MLAD, where acoustics interact with the duct through diffraction or range dependent scatter\citep{colosi21}. This study investigates the relative impact of two dynamically separate sources of sound speed variation, the tilt and spice fields, on the MLAD with a decomposition of an upper ocean transect observation.

While vertical mixing is expected to homogenize the vertical mixed layer properties, range dependence is also expected as the bulk properties of the surface ocean changes\citep{ferrari2000}. The variation of temperature and salinity in the upper ocean tends to be highly correlated, and the relative effects of these changes tend to moderate the overall effect on density. On large scales, temperature varies at twice the rate of this salinity moderation. On shorter scales, however, the temperature and salinity variations often compensate in density completely. Density compensated waters are either relatively warm and salty or cool and fresh, and this variation is termed ocean "spice"\citep{munk1981evolution}. The two scales of range variation are rectified at ocean fronts where density changes abruptly, and these sharp density gradients lead to further enhancement of upper ocean mixing through symmetric instabilities\citep{dasaro2011}. Similarly, density compensated spice variation tends to be front like, although mixed layer observations discussed here often show acoustically significant vertical variation at these fronts.

Importantly for the acoustic problem, density compensated temperature and salinity variations are reenforcing in sound speed. Although spice variability changes significantly with geographical location, studies have shown that spice variations can significantly change acoustic propagation\citep{colosi12,colosi13,murat2021}. The importance of spice in separated acoustic environments and the variation of spice over time remain topics of active research.

Range or time dependent measurements of the mixed layer show that it is has highly variable sound speed\citep{cole2010seasonal,rudnick1999compensation,klymak2015}, which can significantly alter or completely block acoustic duct propagation\citep{colosi2020observations,colosi21}. Typical observations show important root mean square (RMS) sound speed variation in the mixed layer acoustic duct (MLAD), a peak in the roughly 100 m wide transition layer (TRL) below the MLAD, and then a slow reduction in the upper permanent thermocline. In transect observations this sound speed field variability can be decomposed to isolate the effect of spice following Dzieciuch \emph{et al.}\citep{dzieciuch2004}, which uses spatial lowpass filters on observed isopycnal height and along isopycnal sound speed variation. These lowpass filters generate fields with sound speed variation from only isopycnal height variation or only along isopycnal sound speed variation, the tilt and spice fields, respectively.

Three upper ocean acoustic propagation scenarios are simulated for the decomposed fields at 400 and 1000 Hz with the PE model RAM\cite{collins93}. The average transect sound speed profile predicts one and three modes trapped in the MLAD at these frequencies, respectively, and this frequency comparison studies mode coupling from sound speed range dependence. Acoustic propagation is discussed for sources both inside the MLAD and the TRL to positions both inside the MLAD and in the TRL. These simulations are made over a transmission length of one convergence zone\citep{jensen2011computational}, approximately 50 km in the subtropical Pacific Ocean, which removes the consideration of coupling into the duct at convergence zones\citep{colosi2020observations}.

Acoustic simulations were made for source positions every 10 km along the transect. While most results are reported as mixed layer energy statistics, discrete blocking features\citep{colosi2020observations} observed throughout the transect are also treated separately. Blocking features create significant and localized MLAD acoustic energy loss and are usually observed for any acoustic simulation that includes the feature. Blocking features appear for both decomposed fields and are most evident at 400 Hz. Differences in relative loss with and without blocking features show these events represent a distinct loss mechanism compared to diffuse range dependent loss that exists at all ranges and are relevant to each of the three propagation scenarios discussed here.

The paper is organized as follows: Section~\ref{sec:transcet} describes the oceanographic mixed layer observations that motivate this study. Section~\ref{sec:decomposition} discusses the dynamic decomposition of the observed sound speed field into spice and tilt fields. Acoustic propagation through the separated sound speed fields is compared with the observed field in Sec.~\ref{sec:propagation}. Finally, Sec.~\ref{sec:conclusion} is the study conclusions.

\section{North Pacific transect}\label{sec:transcet}
\begin{figure}
\includegraphics{../figures/sound_speed_transcet_sld.png}
\caption{\label{fig:c_grid}{Sound speed observation of the mixed layer. The black line is the sonic layer depth.}}
\end{figure}

This study is based on a 970 km SeaSoar conductivity, salinity, and depth (CTD) transect taken over 4 days in the northeast Pacific ocean,\citep{cole2010seasonal}. The 970 km transect followed a great circle path roughly parallel to the sea surface temperature contours, although minor differences in slope between these two predict a slight warming of the mixed layer over the transect. SeaSoar CTD observations extend from the sea surface to 430 m depth, with an average cycle length of 2.6 km. These observations were first interpolated from the vehicle's sawtooth path to a grid with 1 km horizontal and 0.5 m vertical resolution\citep{colosi2020observations}, and World Ocean Atlas climatology is appended to a depth of 4000 m\citep{WOA}. Sound speed, $c$, computed from the CTD data with the Thermodynamic Equation of Seawater 2010\cite{TEOS-10} is shown in Fig. \ref{fig:c_grid}. The mixed layer acoustic duct (MLAD) is apparent as the relatively high sound speed over a strong negative sound speed gradient that extends into the deep ocean. Horizontal and vertical variability in sound speed is apparent throughout the MLAD observations from internal waves, fronts, eddies, and ocean spice\citep{colosi2020observations}.

The sonic layer depth (SLD) is a simple metric of the MLAD, defined as the depth of maximum sound speed above the thermocline, and shown as a black line in Fig.~\ref{fig:c_grid}. The MLAD is expected to strengthen with increasing SLD, a quantification of this is the mode cutoff frequency\citep{Urick1982Prop},
\begin{equation}
    f_{min}(n) = \frac{3(4n-1)}{16} \sqrt{\frac{c_0^3}{2h^3} \left( \frac{\partial c}{\partial z} \right) ^ {-1}},
    \label{eq:f_cutoff}
\end{equation}
where $f_{min}$ has units of Hz. The SLD is $h$, and a refence sound speed of $c_0 = 1500$ m/s is assumed. Several observations on mode cutoff frequency over the complete transect from Colosi and Rudnick\cite{colosi2020observations} are summarized here. Sound speed gradients vary from 0 to above 0.1 and are on average higher than the adiabatic gradient of 0.016 $s^{-1}$ assumed for a vertically homogeneous mixed layer. The first four cutoff frequencies for the transect average sound speed profile are 210, 480, 760 and 1040 Hz. These observations lead to the acoustic simulation frequencies of 400 and 1000 Hz used in this study, which have on average 1 and 3 modes in the MLAD, although individual profiles may have cut offs above one or both frequencies.

\begin{figure}
\includegraphics{../figures/ts_mean_c.png}
\caption{\label{fig:ts_diagram}{Tempurature and salinity observed along transcet, color of markers indicates transcet range of observation.}}
\end{figure}
The decompostion of observed upper ocean dynamics into spice and tilt effects quantifies the source of both the observed variations in SLD and additional considerations like changes in the sound speed gradient and non-linear profiles. The difference between these two types of dynamics are first demonstrated with conservative temperature and absolute salinity, ($\theta$, S), shown in Fig.~\ref{fig:ts_diagram}. The markers of each observation become lighter with increasing transect range, $x$. The potential density and sound speed are shown as darker and ligher contour lines, respectivly. The potential density is expected to be dynamically stable and increase with depth for each profile, in rare cases the gridded data was adjusted to enforce this requirement of minimal dynamic stability\citep{barker2017stabilizing}. Ocean spice, or density compensated variation of temperature and salinity, is apparant in the spread in ($\theta$, S) parrellel to the density contours. The observed spice variation can create significant changes in sound speed, apparent in ($\theta$, S) spread perpendicular to the sound speed contours.

The ($\theta$, S) spread is observed both as regional differences between markers of different colors and local variation between markers of similar color. This spice variation is typically quanitified relative to a curve of mean ($\theta$, S), which is computed in this study as a 100 km running mean in transcet range. This long wavelength variation is ocean properties is expected to be relativly stable, and is used to construct a background field\cite{dzieciuch2004}. Two mean ($\theta$, S) curves are shown as red lines in Fig.~\ref{fig:ts_diagram}, the solid and dashed lines are centered at $x$ of 200 and 800 km, respectivly. The mixed layer appears in the mean curves as variation almost parrellel to density contours for the lightest observations. The mean curve at 100 km has a relativly cool and dense mixed layer compared to 800 km, consistant with the lower sound speeds observed a close transect ranges in Fig.~\ref{fig:c_grid}. The properties in the transition layer (TRL) below the MLAD both freshen and shown more variation between isopycnals for the 800 km mean profile. Finally, the smallest variation is observed for the densest waters measured at the top of the permanent thermocline\cite{cole2010seasonal}.

The tilt field in the dynamic decomposition is computed from the mean ($\theta$, S) profiles, whereas the spice field includes the complete ($\theta$, S) variation observed over the transect. The additional difference between these fields is the consideration of the isopycnal positions, discussed in Sec.~\ref{sec:decompostion}. The spice variation in sound speed along isopycnals is shown for a reference pressure of 0 dbar in Fig.~\ref{fig:ts_diagram} (b). The extent of sound speed changes mostly increases with decreasing density, and the most variance are seen at the furthest transcet ranges. The magnitude of the sound speed changes often exceeds 2 m s$^{-1}$ and approaches 4 m $s^{-1}$ in the mixed layer, which are large enough to be acoustically relevant.

%An example of ocean spice is illustrated in two profiles measured at 249 and 252 km transect range, shown in Fig. \ref{fig:profiles}. The density is roughly constant for both profiles to 90 m depth. The two profiles at 249 and 252 km have SLDs at 45 and 89 m, respectively, while the sound speed gradients are 0.24 and 0.15 s$^{-1}$. Equation \eqref{eq:f_cutoff} predicts cutoff frequencies of 485 and 222 Hz, indicating the relative importance of SLD compared to the sound speed gradient.

%\begin{figure}
    %\includegraphics{../figures/sld_profile_grad.png}
    %\caption{\label{fig:profiles}{(a) Potential density, $\sigma$ (b) absolute salinity, $S_A$ and (c) sound speed, $c$, measured at 249 km (blue curves) and 252 km (orange curves). The mixed layer depth is shown as horizontal pink lines in (a), and the sonic layer depth is shown as horizontal black lines in (c).}}
%\end{figure}

\section{\label{sec:decomposition}Spice and tilt decomposition}
A dynamic decomposition is used to separate the observed sound speed variations into contributions from isopycnal tilt and spice. This decomposition follows the method proposed for a similar SeaSoar transect by Dzieciuch \emph{et al.},\citep{dzieciuch2004} however, spice in the mixed layer is treated differently here. The decomposition is first described for the case of the stratified region below the mixed layer. A linear superposition model for spice is then introduced for the marginally stratified mixed layer, and the results are compared. Finally, the root mean square (RMS) sound speed variations of the decomposed components are compared.

\subsection{Stratified ocean decomposition}
The dynamic decomposition begins by defining a gridded quantity, $\sigma$ is used for demonstration, as an interpolation of values between isopycnals. The isopycnal $z$ position is defined $\sigma(x, z) = z^{-1}(x, \sigma)$ and discretized as $z_i(x; \sigma_i)$. A two-dimensional linear interpolation is defined for $\sigma$,
\begin{equation}
    \sigma(x,y)\approx\mathcal{L}(x, y; \sigma_i, z_i).
    \label{eq:lin_intr}
\end{equation}
The isopycnal positions $z_i(x, \sigma)$ are modeled as the superposition of fine scale dynamics on a stable background position $\bar{z}(x, \sigma)$. The background density field, $\bar{\sigma}$, is defined by substituting $\bar{z}(x, \sigma)$ for $z(x, \sigma)$ in Eq. \eqref{eq:lin_intr}. The estimate of $\bar{\sigma}(x,y)$ is a vertically stretched field without fine scale tilt dynamics.

The background isopycnal position, $\bar{z}(x, \sigma)$, is estimated with a spatial low-pass filter. This study uses a cutoff of 50 km, the approximate length of one acoustic convergence zone\cite{jensen2011computational}. This estimate of $\bar{z}(x, \sigma)$ is shown as red lines in Fig. \ref{fig:cntrs}(a). The position of isopycnals inside the mixed layer are unstable and can change across the entire MLD within the horizontal resolution of observation, consistent with density changes in the mixed layer occurring at fronts. However, a number of isopycnals are observed to have multiple fronts, like the sharp increase and decrease in the depth of $\sigma=25.21$ kg/m$^3$ at the left of Fig.~\ref{fig:cntrs}(a). These isopycnal positions are expected to be poorly described as a perturbation from the contours low-passed position. Therefore, instead of tracking the unstable positions of isopycnals in the mixed layer, the dynamic decomposition only estimates the mean position for isopycnals beneath the mixed layer.

The observed unstable isopycnals move below the mixed layer and become stable with increasing range, consistent with density decreasing with an overall increase in surface temperature along the transect. A conservative estimate is made for the start position of $\bar{z}(x, \sigma)$ at a position of the isopycnal becomes correlated with the denser isopycnals. For example, Fig. \ref{fig:cntrs} shows $\sigma=25.25$ (kg/m$^3$) stays below the mixed layer past 270 km. These positions where the dynamic decomposition begins tracking isopycnal positions create horizontal discontinuities in the stable background field, consistent with a front model for density change in the mixed layer.

The stable isopycnal position estimates can similarly define the spice field without fine scale isopycnal tilt,
\begin{equation}
    S(x, z)\approx\mathcal{L}(x, z; S_i(x), z_i),
    \label{eq:lin_inter}
\end{equation}
where $S_i(x)$ is measured along the $\sigma_i$ isopycnal. Since the sampling of $S$ is determined by the isopycnal positions of $\sigma$, the accuracy of Eq. \eqref{eq:lin_inter} depends on a correlation between changes in these two variables. This sampling is expected to be most accurate in the highly stratified regions below the mixed layer, and the least accurate in the mixed layer itself.

The value of $S_i(x)$ in Eq. \eqref{eq:lin_inter} can also be modelled as the superposition of fine scale dynamic structure on a stable background, $\bar{S}_i(x)$. Like the estimate of background isopycnal position, a low-pass filter with 50 km cutoff length is used to estimate $\bar{S}_i(x)$ along each observed isopycnal. Combinations of the low-pass and observed values of $\sigma_i$ and $S_i(x)$ can be substituted into Eqs. \eqref{eq:lin_intr} and \eqref{eq:lin_inter} to produce four $S(x,z)$ fields. The stable background field is computed with ($\bar{S}_i$, $\bar{z}_i$), the tilt field with ($\bar{S_i}$, $z_i$), the spice field with ($S_i$, $\bar{z}_i$), and the observed total field is conceptually produced from ($S_i$, $z_i$).

\subsection{Mixed layer decomposition}\label{ssec:ml_decomp}
\begin{figure}
\includegraphics{../figures/sound_speed_comp.png}
    \caption{\label{fig:c_diff}{Difference in spice field sound speed between stratified decomposition and linear superposition methods. The two decompositions are equal at the stable position of the lightest tracked isopycnal. The discontinuity in stable isopycnal position at 270 km is the last tracked position of isopycnal $\sigma=25.27$ kg/m $^3$. Significant vertical variation of sound speed is apparent in the mixed layer above the last tracked position of the isopycnal.}}
\end{figure}

Estimation of the stable isopycnal position and spice with a low-pass filter is most effective outside of the mixed layer where the ocean is well stratified in regions of spice variation. Two challenges in the dynamic decomposition of Dzieciuch \emph{et al.}\citep{dzieciuch2004} arise in the mixed layer: (1) significant variations of $S$ occur in positions with small $\sigma$ gradients, and (2) isopycnal locations vary rapidly and may not have a stable position. A linear superposition model for $S$ is proposed here for the mixed layer to avoid the requirement of sampling along isopycnals.

The linear superposition model for $S$ attributes all observed variability not explained by the tilt field to the spice field, without any vertical stretching to account for the mean isopycnal levels. The linear superposition model is written
\begin{equation}
    S_{observed} = S_{bg} + \Delta S_{tilt} + \Delta S_{spice},
    \label{eq:lin_sup}
\end{equation}
where $S_{bg}$ is the background spice field. The spice field is defined as $S_{bg} + \Delta S_{spice}$, and the tilt field is $S_{bg} + \Delta S_{tilt}$. The value of $\Delta S_{spice}$ is estimated by subtracting $S_{tilt}$ from $S_{observed}$. This approach uses the background and tilt fields, which do not require vertical stretching of mixed layer isopycnals to estimate the spice field from the observed field.

The field $S_{spice}$ of the stratified decomposition of Eq.~\eqref{eq:lin_inter} and the linear superposition model are compared in Fig.~\ref{fig:c_diff}. The only difference between these methods is above the last stable isopycnal position. While the difference field often shows uniform vertical structure, there are also locations with significant vertical variation. Sampling this vertical structure with the stratified decomposition would require stable isopycnal positions well into the mixed layer, and some positions show sampling would be required in the top 25 m of the mixed layer. The linear superposition model avoids the need to introduce non-physical stable isopycnal positions to sample the observed spice variation in the mixed layer.

\begin{figure}
\includegraphics{../figures/diff_fields.png}
        \caption{\label{fig:c_fields}{Total and decomposed fields. Background field is shown in (a). The dynamic fields are shown with the background field subtracted: (b) is tilt, (c) is spice and (d) is the total field. The SLD is shown in each panel as a black line.}}
\end{figure}

The background and dynamic fields computed with the dynamic decomposition are shown in Fig. \ref{fig:c_fields}. The background field, top panel, shows a high sound speed mixed layer that overlies slower sound speeds in the thermocline. The sound speed in the mixed layer increases with increasing transect range. The SLD of the background field varies smoothly in range except at discontinuous positions where an isopycnal enters the mixed layer, Sec. \ref{ssec:ml_decomp}. The change in SLD, even at the position of discontinuities, is relatively minor compared with that of the dynamic fields shown in the bottom three panels, which predicts the background field will produce marginal range dependent effects.

The difference in sound speed from the background for the tilt, spice, and observed field are shown in panel (b), (c), and (d), respectively. The highest sound speed variations are seen in the tilt field TRL below the mixed layer, which are caused by internal waves changing the depth of the thermocline\cite{colosi21}. The tilt field also shows large scale sound speed anomalies in the top 50 m, mostly positive in the first half of the transect and negative in the second, caused by unstable isopycnal positions in the mixed layer.

The spice field has smaller magnitudes than the tilt field overall, and many of the spice features extend over the entire depth of the mixed layer. Spice contributions to sound speed also appear in the TRL, where they often appear tilted with depth. The effect of spice on the SLD is more intermittent than the tilt field. A clear separation exists between the relatively flat SLDs in the first half of the transect, and very jagged and often shallow SLDs in the second half of the transect. Finally, the observed field contains a clear superposition of the features in the spice and tilt fields, which supports the dynamic decomposition in the discussion of observed sound speed fields.

\subsection{Decomposed sound speed statistics}
\begin{figure}
\includegraphics{../figures/rms_profile.png}
        \caption{\label{fig:c_rms}{The transect mean sound speed profile and RMS measured, tilt and spice field deviations from the background. The SLD is shown as a horizontal line in both panels. A dashed line shows a fit to the sound speed profile above the SLD with a slope of 0.024 s$^{-1}$.}}
\end{figure}

The transect mean sound speed profile and RMS statistics of each dynamic field are shown in Fig.~\ref{fig:c_rms}. The mean sound speed profile has a linear increase of sound speed with depth to the 80 m SLD and is fit to a slope of 0.024 s$^{-1}$. There is a sharp decrease in sound speed below the SLD at in the TRL. The RMS statistics of the observed, tilt and spice fields all have maximum values below the SLD. The shape of the spice field RMS is like the depth averaged spice variance reported by Ferrari and Rudnick\citep{ferrari2000}. The peak value occurs near the SLD, and the RMS approaches zero by 350 m depth. The peak RMS values of the tilt and observed fields occur around 110 m depth, approximately 20 m deeper than the spice peak. The tilt RMS also shows a smaller maximum at the surface. The RMS values of tilt are significantly higher than spice for all positions except around the SLD and between 200 and 250 m depth. The spice maximum, which has half the RMS value of the tilt peak, occurs close to the tilt field minima. The spice then significantly modifies the observed RMS in the most acoustically significant region near the SLD, both increasing the observed minimum value and moving its position up by approximately 20 m.

\begin{figure}
\includegraphics{../figures/diff_spectra.png}
        \caption{\label{fig:spectra}{Power spectral density of the total, tilt, spice, and background fields in the (a) MLAD, (b) TRL and (c) thermocline.}}
\end{figure}
The power spectral density of the sound speed difference of each decomposed field are shown in Fig.~\ref{fig:spectra} in the ML, TRL, and upper thermocline. Following Colosi and Rudnick\cite{colosi2020observations}, each spectrum is an average of 4 depths: 20, 40, 60 and 80 m for the ML; 100, 120, 140, 160 m for the TRL; and 180, 200, 220, 240 m for the upper thermocline. The mean background profile is subtracted from the background field, while the background field is subtracted from each of the dynamic fields. The background field has a sharp fall off for wavelengths above 50 km (0.02 cpkm), and the subtraction of the background from the dynamic fields generally leads to a peak at 50 km wavelength and minimum at higher wavelengths. The tilt and total fields in the mixed layer are an exception to this general shape, which have significant energy above 50 km wavelength. The long wavelength energy in these fields is from near surface stratification structures in the tilt field. Examples of these features in Fig.~\ref{fig:c_fields} are the low sound speed features above 45 m depth at 800 and 875 km transect range, and the positive ML sound speed anomaly between 0 and 300 km. The spice field has a factor of 10 more energy than the tilt field at wavelengths shorter than 50 km. This wavenumber separation between the dynamic fields indicates the mixed layer tilt energy creates long range changes to the MLAD, while the spice field creates more intermittent sound speed changes.

The spectra in the TRL and upper thermocline have similar shapes. In the TRL, the peak of the tilt and total fields at 50 km wavelength is a factor of 10 above the spice field, and tilt dynamics dominate the observed RMS sound speed variation. This variation concentrated below the SLD in Fig.~\ref{fig:c_fields}(b), where the largest observed sound speed variation is seen in internal wave features that span approximately 20 m depth. In the upper thermocline the contribution of tilt and spice contribute equally to the total sound speed variation. The spice spectrum is less peaked and has slightly more energy between 50 and 10 km wavelengths than the tilt. Equipartition is observed between the tilt and spice field over much of the spectrum is a result of the shallow sampling depths, and the RMS shown in Fig.~\ref{fig:c_rms} indicates tilt variation begins to dominate below 250 m depth and spice variation approaches zero.

\section{\label{sec:propagation}Upper ocean acoustic propagation}
\begin{figure}
\includegraphics{../figures/decomp_xmission.png}
    \caption{\label{fig:decomp_x}{Left panels are sound speed field; right panels are acoustic pressure. Rows are the: (a) background (b) tilt, (c) spice, and (d) observed fields. The source region up to 7.5 km has significant down going energy for all fields, and the up going energy of the first convergence zone is apparent starting 47.5 km from the source. Significant MLAD loss between the source and first convergence zone appears as down going energy below the approximately 125 m deep MLAD. The MLAD loss is strongest around 250 and 260 km for the spice and total fields, and smaller loss is also observed in the tilt and total fields around 240 km.}}
\end{figure}

The separate effects of tilt and spice are compared with the observed sound speed field and the smooth background over 60 km propagation sections. Each propagation section is separated by 10 km transect range; a distance chosen as a compromise that samples the observed range variation with reasonable independence. Two source frequencies, 400 and 1000 Hz, are used for comparison of propagation with an average of one and three mixed layer modes, respectively.

The impact of separate upper ocean sound speed dynamics is demonstrated with three upper ocean propagation scenarios. The first considers acoustic energy in the MLAD for a source in the MLAD at 40 m depth (MLAD-MLAD), which directly quantifies MLAD propagation. The second considers the non-ducted energy in the TRL below the mixed layer for a MLAD source at 40 m depth (MLAD-TRL), which is controlled by MLAD energy loss. Finally, the acoustic energy in the MLAD is considered for a source in the TRL at 200 m depth (TRL-MLAD), which both quantifies acoustic coupling into the MLAD in the source region and the subsequent MLAD propagation. The MLAD is the dominant factor that determines the acoustic energy in all three scenarios, and the tilt, spice and observed dynamics all have different relative effects on the acoustic propagation.

Figure~\ref{fig:decomp_x} shows a representative section of MLAD-MLAD propagation at 400 Hz. Acoustic propagation is modeled with the parabolic equation (PE) code RAM\citep{collins93}. The background, tilt, spice and observed sound speed fields are shown in the left column of Fig. \ref{fig:decomp_x}. The PE results, right column  of Fig. \ref{fig:decomp_x}, show significant differences between the different sound speed fields of energy loss and vertical distribution over the propagation range.

The background acoustic field, Fig.~\ref{fig:decomp_x}(a), shows basic MLAD propagation. The MLAD is obscured by high angle propagation up to about 7.5 km from the source, and in the first convergence zone starting around 47.5 km. A highly absorbent layer introduced at the bottom of the PE domain suppresses bottom interactions that would otherwise contribute non-ducted energy to the mixed layer at source ranges between 7.5 and 47.5 km. With the higher angle arrivals removed, the mixed layer between the source region and the first convergence zone has no external sources of acoustic energy. The mixed layer acoustic pressure in this region has one maximum in depth and a slow decrease with range.

The tilt and spice fields are shown in Fig. \ref{fig:decomp_x} (b) and (c), respectively. The most acoustically significant tilt feature variation is a SLD shoaling between transect range 230 and 240 km that leads to mixed layer energy loss in the tilt field. A surface concentration of lower sound speed also appears in the tilt and observed fields between 270 and 280 km transect range. This tilt feature increases the mean sound speed gradient of the mixed layer, both strengthening the mixed layer duct and moving acoustic energy to shallower depths.

The spice sound speed field has significant sound speed variation in the MLAD, both in vertical fronts and features with significant depth variation. Most acoustically significant is a high sound speed feature between 250 and 260 km transect range. This feature causes loss of acoustic energy from the mixed layer duct that is concentrated at the front-like edges. There is also a marked change to the vertical pressure distribution that indicates coupling between the mixed layer acoustic modes. The loss of MLAD energy is characteristic of a blocking feature that significantly reduces the viability of the MLAD, and briefly increases energy in the TRL below the MLAD.

Finally, the observed field appears as a combination of the tilt and spice fields, with many of the distinct features partitioning almost exclusively into one field or the other. The acoustic propagation also shows some of the behaviors of both the spice and tilt field results, although this result is not expected to be a superposition of the component fields.

The sound speed fronts at 250 and 260 km lead to significant losses in acoustic energy, and a near blocking of MLAD propagation. Similar blocking features are observed at other ranges and can appear in one or multiple dynamic fields. A blocking feature metric is defined through mixed layer energy in Sec. \ref{ssec:blocking}. Statistics of the mixed layer energy are significantly influenced by these sporadic high loss events and are reported both with and without blocking features in Sec. \ref{ssec:energy}.

\subsection{Upper ocean acoustic energy}\label{ssec:bg}
Vertically averaged acoustic energy is used to quantify MLAD propagation effects like those shown in Fig.~\ref{fig:decomp_x}. Two measures of mixed layer energy are considered, vertical integration of the PE pressure result, and a limited depth projection of a mixed layer mode (MLM) onto the PE pressure result. The vertical integration of the PE pressure contains no information about the vertical distribution of energy, while the modal quantification describes the energy in the mixed layer with the same vertical distribution as a MLM. The limited depth mode projection is proposed as an alternative to mode amplitudes from coupled mode calculations that approximate orthogonality over the MLAD even when multiple modes have similar shapes in the mixed layer.

The vertical integration of acoustic energy is computed from the PE acoustic pressure result, $p(x, z)$,
\begin{equation}
        \textrm{E} = 20 \, \textrm{log}_{10} \left( \frac{1}{z_1 - z_0} \int^{z_1}_{z_0} \left| p(x, z) \right| \,  dz \right).
    \label{eq:int_eng}
\end{equation}
The MLAD energy is computed by integration between 0 and 125 m depth, while TRL energy is integrated between 125 and 250 m depth.

\begin{figure}
\includegraphics{../figures/bg_eng_loss.png}
    \caption{Spreading compensated acoustic energy in the RI background mixed layer, Eq.~\eqref{eq:int_eng}, for a 400 Hz source at 40 m depth. Significantly more loss is observed at transect ranges less than 300 km, lighter (yellow) lines. The mean loss is shown as thick black line, and the mean plus or minus the RMS estimate are dashed (blue) lines.}
    \label{fig:bg_eng}
\end{figure}
The integrated energy of Eq.~\eqref{eq:int_eng} is shown in Fig.~\ref{fig:bg_eng} for the RI background field and MLAD-MLAD scenario. The 60 km range averaged background field is used as the reference energy loss in variability calculations. Sources at transect ranges before 300 km are light (yellow) lines and transect ranges beyond 300 km are dark (gray) lines. At 400 Hz significantly more loss is predicted for many source positions at transect ranges less than 300 km. The largest energy losses before 300 km are 7 dB down from the 7.5 to 47.5 km, compared with a maximum of 2 dB for source positions beyond 300 km transect range. The larger loses in the background mixed layer correspond with the shallow SLD observed at starting ranges in the transect of Fig.~\ref{fig:c_grid}. The RI background mixed layer loss is far less significant at 1 kHz, where the largest losses are 0.5 dB at 47.5 km.

The limited depth mode projection computes the inner product of a MLM, $\psi(z)$, and the PE pressure field up to the $n$-th zero crossing, $z_n$,
\begin{equation}
    \textrm{E}_{\textrm{\vspace{0.001} MLM}} = 20 \, \textrm{log}_{10} \left( \left| \int^{z_n}_0 \,  p(x, z) \ \psi(z) \,  dz \ \right|\right) - 20 \,\textrm{log}_{10} \left( \int^{z_n}_0 \, \left| \psi(z) \right| \,  dz \right).
    \label{eq:proj_eng}
\end{equation}
The second term is a normalization that accounts for the partial mode energy in the limited depth integration of Eq.~\eqref{eq:proj_eng} and is equal to the water density for modes with no energy outside the mixed layer\citep{jensen2011computational}.
\begin{figure}
\includegraphics{../figures/mode_shapes.png}
    \caption{\label{fig:bg_modes}{(a) Mean sound speed profile of the background field between 230 and 280 km, (b) Shapes of mixed layer mode 1 (mode \#237) and surrounding modes at 400 Hz. The sound speed profile has a SLD of 99.5 m and is fit to a MLAD gradient of 0.017 1/s. Modes 237 and 238 have similar shapes in the mixed layer, and all modes have a significant tail that extends to the compensation depth of the mixed layer around 3200 m depth.}}
\end{figure}

An example of three normal modes for the RI background field, averaged from 230 to 290 km, are shown in Fig.~\ref{fig:bg_modes}. These modes are centered around mixed layer mode 1 (MLM1), which has no zero crossings in the mixed layer. The MLM1 is found as the mode with the most mixed layer energy within an index of the first peak in loop length\citep{jensen2011computational},
\begin{equation}
    l_{m} = \frac{2 \pi}{k_{m+1} - k_m}.
    \label{eq:loop_length}
\end{equation}
The loop length values around MLM1 are typically 50 km and define the convergence zone length, while MLM1 can be 2-3 times this value. This approach can also identify mixed layer mode 2 (MLM2) and higher.

The shape of MLM1 is not orthogonal to neighboring modes over MLAD depth. Instead, the coherent sum of similarly shaped modes reenforce or diminish the amplitude of MLM1 with marginal changes to the vertical distribution of the MLAD pressure field. This interaction between modes leads to a very long-range cycling of energy first out from and then back into the MLAD\citep{porter93,colosi2020observations}, which causes loss of mixed layer energy for the propagation ranges considered here. This energy cycling is most significant for acoustic frequencies close to the mode 1 cutoff of Eq.~\eqref{eq:f_cutoff} and is responsible for the elevated loss in the RI background before 300 km shown in Fig.~\ref{fig:bg_eng}.

\begin{figure}
\includegraphics{../figures/bg_eng_loss_3_panel.png}
        \caption{Background energy loss for three upper ocean acoustic propagation scenarios: (a) MLAD-MLAD, (b) MLAD-TRL, and (c) TRL-MLAD. Dashed lines are the mean RI background energy for the first 300 km of the transect, and solid lines are the mean over the remaining transect. The lighter (orange) lines are 400 Hz, and darker (purple) lines are 1 kHz. Black lines in (c) are a linear and second order polynomial fit to the transect mean RI background energy, which are used to remove the beat pattern in reference energy estimates.}
    \label{fig:eng_bg_3}
\end{figure}

The results of each propagation scenario are reported relative to the RI background, shown in Fig.~\ref{fig:eng_bg_3} for 400 and 1000 Hz sources as lighter (orange) and darker (purple) lines, respectively. High angle, non-ducted, energy is apparent near the source and convergence zone. The difference in the MLAD RI background between transect ranges before and after 300 km seen in Fig.~\ref{fig:bg_eng} is consistent across all three propagation scenarios, displayed as dashed and solid lines before and after this range. Although clear differences in the MLAD exist with transect range, all source positions are used to compute energy statistics after removal of the RI background.

The MLAD-MLAD scenario, Fig.~\ref{fig:eng_bg_3}(a), has the highest RI background energy. There is little difference between the 400 and 1000 Hz MLAD-MLAD energies except for the excess of 2 dB energy loss at 400 Hz and less than 300 km transect ranges. The MLAD ducted energy is apparent away from the source and convergence zones, between 7.5 and 47.5 km. The MLAD-TRL layer energy, Fig.~\ref{fig:eng_bg_3}(b), has significantly lower RI background energy and shows a modal interference beat pattern. When this background is used as an energy reference it is fit to a first and second order polynomial at 400 and 1000 Hz, respectively, shown as black lines. The MLAD-TRL has the highest energy difference between source frequencies, 400 Hz is between 7 and 10 dB higher than 1 kHz and shows less of an interference pattern. The observed energy is approximately 2 dB higher for transect ranges before 300 km, caused by increased diffraction from the duct. The width of the convergence zone is wider in this scenario, and ducted energy is considered between 7.5 and 40 km. The TRL-MLAD, Fig.~\ref{fig:eng_bg_3}(c), has a low energy background as well and a curved energy background. The energy is higher before 300 km transect range. Ducted energy is considered between the ranges of 7.5 and 40 km in this scenario.

\subsection{Blocking features}\label{ssec:blocking}
\begin{figure}
\includegraphics{../figures/integrated_loss.png}
    \caption{Integrated mixed layer energy loss over 5 km. Source ranges with an integrated loss greater than 3 dB (gray dotted line) sample blocking features. At 400 Hz, blocking features are typically registered at all source positions that contain the same feature. At 1 kHz, blocking features mostly have smaller magnitudes and are more sporadic.}
    \label{fig:blocking}
\end{figure}

Blocking features, localized sound speed features transect that cause significant loss, appear throughout each of the tilt, spice and observed transects. These locations are identified in the MLAD-MLAD scenario, although they are expected to affect the three propagation scenarios considered here. Blocking features are most common at 400 Hz, and an example was shown in Fig.~\ref{fig:decomp_x}. These sporadic blocking events significantly affect statistical characterization of acoustic energy, and the results of analysis both with and without blocking features are reported in Sec~\ref{ssec:energy}.

To identify blocking features, the MLAD energy is computed with Eq.~\eqref{eq:int_eng} and normalized to the RI background loss. A blocking feature is defined as a region between source ranges of 7.5 and 47.5 km where the MLAD energy falls at least 3 dB in 5 km, Fig.~\ref{fig:blocking}. The top and bottom panels are 400 Hz and 1 kHz, respectively. At 400 Hz there are eight clearly separated regions of high loss between the total, tilt, and spice fields. These regions are typically 50 km wide and flat topped, indicating the same feature dominates mixed layer propagation for all source positions that include the feature. The MLAD has fewer blocking features at 1 kHz than at 400 Hz. The 1 kHz blocking features are more sensitive to mode phase because they are rarely apparent for consecutive source positions.

The dynamic decomposition shows different partitioning between the dynamic fields of the eight blocking features at 400 Hz. Blocking features typically appear in the observed field and either the tilt or spice field. However, some blocking features appear in only the spice field in the range from 600 and 850 km, indicating the tilt field stabilizes the observed field at these positions. This stabilization effect is observed at positions with concentrations of low sound speed at the surface in the tilt field, Fig~\ref{fig:c_fields}(b), an example of this was also observed in Fig.~\ref{fig:decomp_x} around 270 km.

While the blocking features in the observed field are distributed across the transect range, the tilt field blocking features occur before 450 km and spice blocking features are most common after this range. This blocking feature distribution indicates that the effects of both tilt and spice fields are regionally variable over the 1000 km range of the transect.

\subsection{Upper ocean acoustic energy statistics}\label{ssec:energy}
\begin{figure}
\includegraphics{../figures/blocking_energy.png}
    \caption{400 Hz Energy compensated for range spreading for a source at 230 km in the observed sound speed field. The transmission scenarios are: (a) MLAD-MLAD and MLAD-TRL, and (b) TRL-MLAD. The RI backgrounds for each scenario are shown as black lines, where the MLAD prediction has significantly more acoustic energy in (a). The MLAD-MLAD energy projected onto MLM1 is shown as a dotted (blue) line.}
    \label{fig:ml_energy}
\end{figure}
Energy statistics are calculated over the entire transect for the different combinations of sound speed field, transmission scenario, and source frequency. Finally, source positions with and without blocking features are investigated separately. The transmission scenario energies are shown in Fig.~\ref{fig:ml_energy} for a 400 Hz source at 230 km in the observed field to aid in the interpretation of bulk transect statistics. Results are shown for the MLAD source in (a), and the TRL source in (b). The RI background is used as the reference energy and shown as black lines. The MLAD energy in Fig.~\ref{fig:ml_energy}(a) follows the background until a series of loss events starting at 250 km, the largest at 260 km is characterized as a blocking feature. The net effect of these large loss events is approximately 10 dB of loss by the first convergence zone at 47 km.

The MLM1 magnitude describes the vertical distribution of pressure in MLAD-MLAD propagation, and is shown adjusted by $-10 \, \textrm{log}_{10}(z_{SLD})$ to agree with the mean energy scaling of Eq.~\eqref{eq:int_eng}. The mode projection closely follows the total energy until the 260 km blocking feature, where the MLM1 energy falls 3 dB below the total energy. This difference in energies is observed as the MLM2 vertical distribution of energy in Fig.~\ref{fig:decomp_x}(d).

The TRL energy, Fig.~\ref{fig:ml_energy}(a), is strongly downward refracted and the source TRL of energy in this scenario is MLAD energy loss. Examples of the beams of energy into the TRL from the MLAD are seen below the large loss events in Fig.~\ref{fig:decomp_x}(d). These loss events lead to peaks in TRL energy, the largest peak occurs at 250 km and brings the TRL energy to within 2 dB of the MLAD energy. These peaks in TRL energy are only a few kilometers wide, however. While these peaks are significant gains in TRL energy, they can be followed by significant decreases in energy. The energy between 265 and 275 km is an example where a stable MLAD energy level reduced from the RI background results TRL in average energy below the RI background. While large loss events lead to large TRL energy peaks, finite MLAD energy makes a slow and consistent energy loss necessary to increase the average TRL energy over the transmission length.

The TRL-MLAD energy, Fig.~\ref{fig:ml_energy}(b), shows two separate effects required for increasing MLAD energy. First, coupling into the MLAD in the source region leads to MLAD increased energy relative to the RI background over much of the transmission. Secondly, this energy propagates inside the MLAD and experiences loss events like the MLAD-MLAD scenario in Fig.~\ref{fig:ml_energy}(a). The increase in MLAD energy from the source region is apparent between 240 and 260 km transect range, and this approximately 4 dB excess leaves the MLAD at the blocking feature around 260 km.

\begin{figure}
\includegraphics{../figures/eng_shallow.png}
    \caption{MLAD energy relative to the RI background for decomposed sound speed fields with a source at 40 m depth. Left column is propagation at 400 Hz, right column is at 1 kHz. The top 4 rows are statistics for the complete set of simulated source positions. The bottom 3 rows are statistics for source positions without blocking events (W/O Blocking). The integrated energy for all transmissions in the set are shown as light grey lines, half circles at -10 dB indicate a line moves beyond the plotted scale. Linear regression fits of the mean (bold) and RMS ($\pm$) at 7.5 km and 47.5 km are shown in the two columns right of the line plots.}
    \label{fig:shal_eng}
\end{figure}
The MLAD-MLAD energy, Eq.~\eqref{eq:int_eng}, is shown in Fig.~\ref{fig:shal_eng}. The 400 and 1000 Hz results are in the left and right columns, respectively. The complete set of source positions is shown in the top section, while the set of source positions without blocking features is shown in the bottom section.

The first result is background field indicates marginal range dependent effects, the highest RMS is only 0.6 dB at 400 Hz and 47.5 km range, and the discussion of range dependence will focus on the dynamic sound speed fields. The complete set of source positions for all dynamic fields predicts small mean loss over the source region before 7.5 km at 400 Hz, and small mean gains over this region at 1 kHz. This increased coupling into the mixed layer at higher frequency is also expected in the convergence zones, which also have a wide range of high angle modes, but these are not studied here. The statistics at 47.5 km predict mean loss for all fields and frequencies.

The tilt field has the smallest energy loss at both frequencies. The total loss is more at 400 Hz, although the loss difference between 7.5 and 47.5 km is the same for both frequencies. The RMS at 400 Hz is significantly higher than 1 kHz, and this statistic is skewed by high loss events since the upper RMS bound is not realized by any source position. However, the tilt field over the complete transect is often favorable to propagation, with marginal loss or small gains relative to the RI background. The largest mean energy loss at 47.5 km is predicted for the spice field at both frequencies, and almost 1.5 dB more loss is predicted at 400 Hz. The RMS at 47.5 km is higher than the mean at 400 Hz, and while this statistic is loss dominated some fields do show marginal gains of energy. Overall, ocean spice leads to a larger mixed layer energy loss than tilt. Finally, the observed field is like the spice with lower mean and higher RMS. This RMS contains the blocking features of both the tilt and spice fields and is skewed by high loss events.

The mixed layer energy statistics without blocking features are shown in the bottom three rows of Fig.~\ref{fig:shal_eng}. For consistency, source positions were removed from this analysis that had blocking features at either frequency. Outside of an overall adjustment in mean and RMS loss, many of the observations from the complete statistics apply to the statistics without blocking. Two notable exceptions are: (1) the mean loss at 47.5 km range is the same or higher at 1 kHz compared to 400 Hz without blocking features, and (2) RMS values are similar across field type and frequency. The higher mean loss at 1 kHz is a reversal from the complete field statistics but is consistent with the increased sensitivity of higher frequency acoustics to small-scale sound speed fluctuations. The consistency in RMS loss simplifies the comparison of the mean values and clearly identifies the stabilizing effect of tilt in the observed fields. The differences in statistics between the observation sets support the discussion of blocking features as distinct from the non-localized loss mechanisms expected to be present for all source ranges.

\begin{figure}
\includegraphics{../figures/eng_shallow_proj.png}
    \caption{Projected MLAD energy relative to the RI background for decomposed sound speed fields and a source at 40 m depth. Presentation of the result follows Fig.~\ref{fig:shal_eng}, but with a larger y-axis range of -20 to 5 dB. The mode used in the energy projection of Eq.~\eqref{eq:proj_eng} is MLM1 at 400 Hz and MLM2 at 1 kHz.}
    \label{fig:shal_proj}
\end{figure}
The vertical structure of MLAD energy is analyzed by projecting the energy field onto a MLM, Eq.~\eqref{eq:proj_eng}. The MLMs are chosen for the largest source magnitude at each frequency, MLM1 at 400 Hz and MLM2 at 1 kHz. The energy projection results are shown in Fig.~\ref{fig:shal_proj} with the same presentation as Fig.~\ref{fig:shal_eng}. The lower bound of these results is increased to -20 dB to show the increased mean and RMS loss compared with the total mixed layer energy. The reported statistics at 7.5 and 47.5 km are the result of a linear regression fit to each statistic, which minimized the effects of brief dips in mode amplitude.

The mean and RMS values for the mode projection show more loss compared with the total energy at all ranges and frequencies except for the background field. The background field shows the same statistics at 400 Hz as the integrated energy, consistent with the mode 1 shape of the background energy seen in Fig.~\ref{fig:decomp_x}(a). The 1 kHz background statistics have a mean of approximately -1 dB for both reported ranges, which indicates source region coupling from the MLM2.

The projected and total MLAD energies show the same trend for the complete source set at 400 Hz, although there is more loss in MLM1 magnitude than total energy loss. The projected statistic has around 1 dB more mean loss for the spice and observed fields at all ranges, and approximately 0.5 dB more RMS for these fields. The increased mean loss compared with the total mixed layer energy indicates that some ranges couple energy into MLM2, an example of which was shown in Fig.~\ref{fig:blocking}.

The increase in mean and RMS loss for the mode projection is significantly larger for 1 kHz. The highest mean loss is approximately -10 dB for the spice and observed fields. When combined with RMS values of more than 5 dB, many transmission ranges are expected to have nearly complete coupling out of MLM2. The coupling serves to maintain energy in the MLAD, and there is significantly less total energy loss at 1 kHz than at 400 Hz. Finally, while the tilt field shows the least energy loss at 47.5 km, the RMS values and are the highest of all fields. The large coupling in the tilt field at 1 kHz frequency may be an indication of how this field serves to decrease loss in the observed field total energy of Fig.~\ref{fig:shal_eng}.

Without blocking features, the 400 Hz projected, and total MLAD energy statistics are nearly identical. This indicates extreme blocking events are necessary to create mode coupling at 400 Hz. While there is a substantial decrease in the 1 kHz projected statistics without blocking features, the mean and RMS loss values are still higher than the total MLAD, and mode coupling is expected to be ubiquitous in the mixed layer at 1 kHz. This increase in mode coupling at 1 kHz serves to both reduce the total energy loss at blocking features and increase the diffuse energy loss at in ranges without blocking features.

\begin{figure}
\includegraphics{../figures/eng_shallow_tl.png}
        \caption{TRL-MLAD energy relative to a linear fit of the RI background at 400 Hz and a 2nd order polynomial fit at 1 kHz. Presentation of the result follows Fig. \ref{fig:shal_eng}, with y-axis between -20 and 25 dB.}
    \label{fig:eng_tl}
\end{figure}
The MLAD-TRL energy is shown next in Fig.~\ref{fig:eng_tl}. The truncated range of statistics (black lines) indicates the region used to fit a linear regression that avoids the first convergence zone, and results at 47.5 km are an extrapolation. The beat pattern in the background field is smoothed out by scattering in the dynamic fields. A similar beat pattern was removed from the reference energy by a polynomial fit, Fig.~\ref{fig:eng_bg_3}(c). This choice in reference energy then emphasizes the beat pattern in the background field, leading to relatively high RMS values, but does not introduce an artificial beat pattern into the dynamic field results.

The dynamic field energy at 400 Hz has little to no gain at 7.5 km, and losses at 47.5 km. The smallest loss is observed for the spice field, and the most loss occurs for the observed field. Consistent and gradual loss from the MLAD is required to increase the transect mean energy, discussed with Fig.~\ref{fig:ml_energy}, and these decreases in energy could represent either localized loss events or an overall stabilization of the MLAD. The statistics with blocking features removed have similar mean but consistently less RMS at 47.5 km, most apparent in the spice and observed fields. The RMS change at 47.5 km indicates that while significant gains in localized energy are common, these localized events do not make a significant impact on the expected TRL energy. This is also consistent with a visual comparison of the RMS bounds with the ensemble realization in Fig.~\ref{fig:eng_tl}, that show more localized exceedances with blocking features than without.

The mean dynamic energy at 1 kHz is increased at both 7.5 and 47.5 km, which is partly compensated between the two frequencies by the difference in mean energy. The relative frequency difference of the dynamic at 47.5 km is between 9 to 13 dB, which is comparable or larger than the approximately 10 dB difference in reference RI background energy. The largest increase is seen for the spice field. The mean increase in 1 kHz TRL energy is consistent with the larger diffuse energy loss at 1 kHz discussed for the MLAD-MLAD scenario, Fig.~\ref{fig:shal_eng}. This diffuse energy loss is expected to slowly decrease energy in the MLAD and hence increase TRL energy across the entire transect. The ensemble realizations also show many events that temporarily exceed 25 dB relative to the background, and the level of these extreme events is like the levels observed at 400 Hz.

\begin{figure}
\includegraphics{../figures/eng_deep.png}
    \caption{Energy statistics for TRL-MLAD scenario. Presentation of the result follows Fig. \ref{fig:shal_eng}, with y-axis bounds of $\pm$20 dB. The statistics are computed between 7.5 and 40 km and extended to 47.5 km with a linear regression fit.}
    \label{fig:deep_eng}
\end{figure}
Finally, the TRL-MLAD energy statistics are shown in Fig.~\ref{fig:deep_eng}. In this scenario, the background field has a large effect on propagation, especially at 1 kHz. The linear energy change for all source ranges suggests this is caused by long wavelength mode resonances with the slowly varying background field\cite{colosi21}, and this diffractive effect is significant here when compared with the low energy background. There is a larger mean and RMS effect on MLAD energy for all dynamic fields, however, with RMS values between 7 and 11 dB at 47.5 km.

As discussed with Fig.~\ref{fig:blocking}, source region coupling is necessary to first increase ducted MLAD energy in the TRL-MLAD scenario. The ducted MLAD energy is then expected to behave like the MLAD-MLAD scenario past the source region. This is simplest to see in cases when source region coupling is uncorrelated with blocking features, \emph{i.e.} the tilt and observed fields that show almost no change with and without blocking features at 7.5 km. With the observed field as an example, the energy mean at 47.5 km increases by more than 2 dB and the RMS decreases by more than 1 dB with the removal of blocking features. This change indicates the MLAD energy in the TRL-MLAD scenario is still impacted by the effects of dynamics sound speed fields on MLAD propagation.

The spice field shows an exceptional case where mean and RMS spice field energy statistics decrease significantly at 7.5 km with the removal of blocking features. This correlation then obscures the effect of blocking features on the MLAD propagation of scattered energy, and the removal of blocking features has little effect on the mean energy at 47.5 km. A second exception in TRL-MLAD energy behavior is the anonymously low 1 kHz mean in the tilt field at 7.5 km. The reduced scatter into the tilt field MLAD at 1 kHz could be caused by increased sensitivity at higher frequencies to the strong, small scale, sound speed perturbations at the base of the mixed layer, Fig.~\ref{fig:c_fields}(b).

\section{Conclusion}\label{sec:conclusion}
A 1000 km transect made of the upper 350 m of the Northeast Pacific ocean was decomposed to produce sound speed fields with separated isopycnal tilt or spice variation. This decomposition followed Dzieciuch \emph{et al.}\citep{dzieciuch2004} but used a linear superposition model within the mixed layer. The different components of this dynamic decomposition have significantly different effects on three different upper ocean acoustic propagation scenarios, which were all related to propagation in the MLAD. Many of the observed sound speed features were partitioned into either the tilt or spice fields, indicating dynamically separate oceanographic processes combined to create the upper ocean sound speed environment. The highest sound speed RMS in the tilt field was from long wavelength features near the surface and short wavelength variation due to internal waves at the base of the mixed layer. The spice field had shorter wavelength sound speed variation in the mixed layer that was often front-like, and a peak in sound speed RMS roughly 20 m shallower than the tilt field.

Acoustic propagation simulations were made at 400 and 1000 Hz, which had an average of one or three mixed layer modes, respectively. The background field, a product of the dynamic decomposition without either spice or tilt effects, showed that the transect had an acoustically significant change around one third of the total transect range. This acoustic variation was most significant at 400 Hz, where the acoustic duct was significantly more stable in the last two thirds of the transect. This is in contrast with a dynamic description of the mixed layer based on density\citep{cole2010seasonal}, where the observation is roughly consistent with range. Acoustically significant transect variation was also seen in an analysis of blocking features, where all tilt blocking features were observed in the first half of the transect. Most blocking features were observed in the spice field in the second half of the transect.

Finally, the effects of the spice, tilt and observed fields were compared over the entire length of the transect with energy statistics. These energy statistics were analyzed both with and without blocking features. In the MLAD-MLAD scenario, the largest losses were observed in the spice field. The tilt field had the least loss, while the observed field had slightly less than the tilt field. These results were consistent at both study frequencies, and both with and without blocking features, and indicated that the tilt field had a net stabilizing effect on the observed sound speed field. The entire transect statistics had high mean and RMS losses in all fields, with the most loss at 400 Hz. This loss was significantly lower in transmissions without blocking features, and these losses were highest at 1 kHz. This indicated that lower frequencies were more sensitive to discrete blocking events, while higher frequencies were more sensitive to non-localized losses.

Mode amplitude analysis of the MLAD-MLAD scenario showed higher losses for a single mode than the total field, indicating coupling was significant for both frequencies. The largest coupling was found in the observed field at 400 Hz, and in the tilt field at 1 kHz. Little difference between total and MLM1 projected energy was observed at 400 Hz without blocking features, and mode coupling at this frequency is expected only in large and coherent loss events. While the mode coupling was significantly reduced without blocking features at 1 kHz, the loss values for one mode significantly exceeded that for the total mixed layer, and mode coupling is expected to be ubiquitous at this frequency.

In the MLAD-TRL scenario, energy in the TRL is related to local MLAD loss. The interpretation of these statistics is complicated because the TRL energy is not ducted, and a consistent increase in input of energy is required to increase the average energy. The spice field had the highest average in TRL energy at both frequencies, indicating the losses from the MLAD were consistent with range. The smaller TRL energy for the tilt field indicated either this field stabilized the MLAD or had smaller, diffuse, loss events. Blocking features had a marginal effect on the mean energy in the TRL but did change the RMS energy significantly. A more notable change was seen between the average TRL energy at 400 and 1000 Hz, consistent with the smaller and more consistent loss at 1 kHz seen in the MLAD-MLAD scenario.

Finally, the TRL-MLAD scenario showed that the dynamic fields increased the ducted energy in the MLAD with scatter inside the source region. This scatter is roughly equal across the dynamic fields at 400 Hz, and significantly reduced in the tilt field at 1 kHz. The reduction of source region scatter into the MLAD at 1 kHz could be from the small wavelength and large magnitude sound speed variation at the base of the mixed layer caused by internal waves. The propagation of this energy after entering the MLAD was consistent with the MLAD-MLAD scenario, and the removal of positions with blocking features significantly increased the MLAD energy.

In all three propagation scenarios the observed upper ocean dynamics caused acoustically significant variations in sound speed. The dynamic decomposition of these variables into tilt and spice highlighted the separate effects of many features observed in the upper ocean. Of these two fields, ocean spice was often the dominant source of acoustically relevant sound speed variation in acoustic propagation scenarios related to the MLAD. The effect of tilt was often smaller in magnitude, in part because this field appeared to have a stabilizing effect on MLAD propagation over much of the transect. The effects of these dynamic fields were also variable across different regions of the transect, indicating regional differences in both dynamics that require further characterization in additional environments.


\bibliographystyle{jasanum2}
\bibliography{eRichards_master}

\end{document}
