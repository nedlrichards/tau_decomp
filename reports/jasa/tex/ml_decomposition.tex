%	JASA LaTeX Sample File, Preprint Sample
%
%  Beginner Latex users should refer to their favorite online documentation
%  here is one from the TeX Users Group 
%	https://www.tug.org/twg/mactex/tutorials/ltxprimer-1.0.pdf
%
%  Useful FAQ from  https://journals.aps.org/revtex/revtex-faq
% 

%%%%%%% For Preprint
%% For manuscript, 12pt, one column style. Preprint is required for submission.

\documentclass[preprint]{JASA}

%%%%% Preprint Options %%%%%
%% The track changes option allows you to mark changes
%% and will produce a list of changes, their line number
%% and page number at the end of the article.
 %\documentclass[preprint,trackchanges]{JASA}


%% NumberedRefs is used for numbered bibliography and citations.
%% Default is Author-Year style.
%\documentclass[preprint,NumberedRefs]{JASA}

%%%%%%%%%%%%%%%%%%%%%%%%%%%%%%%%%%%%%%%%%%%%%%%%%%%%%%%%%%%%%%%%%
%%%%%%%%%%%%%%%%%%%%%%%%%%%%%%%%%%%%%%%%%%%%%%%%%%%%%%%%%%%%%%%%%
%%%%%%% For Reprint
%% For appearance of finished article; 2 columns, 10 pt fonts
%% Also used for estimating print page length.

% \documentclass[reprint]{JASA}

%%%%% Reprint Options %%%%%

%% NumberedRefs is used for numbered bibliography and citations.
%% Default is Author-Year style.
% \documentclass[reprint,NumberedRefs]{JASA}

%% TurnOnLineNumbers
%% Make lines be numbered in reprint style:
% \documentclass[reprint,TurnOnLineNumbers]{JASA}

\begin{document}

\title[Mixed layer tilt and spice]{North Pacific upper ocean spice and isopycnal tilt sound-speed structures and their effects on acoustic propagation}
\author{Edward L. Richards}
\email{edwardlrichards@gmail.com}
\affiliation{Ocean Sciences, University of California Santa Cruz, Santa Cruz, California 95064, USA}
\author{John A. Colosi}
\affiliation{Department of Oceanography, Naval Postgraduate School, Monterey, California 93943, USA}

\preprint{E. L. Richards, JASA}	%if you want want this message to appear in upper right corner of title page

\date{\today}


\begin{abstract}

\end{abstract}

\maketitle

\section{\label{sec:intro} Introduction}

The paper is organized as follows: Section~\ref{sec:decompostion} discusses the dynamic decompostion of the observed sound speed field into spice and tilt structures.

\section{North Pacific transcet}
The focus of this study is a 970 km SeaSoar conductivity, salinity and depth (CTD) transcet taken over 4 days in the North Pacific. The observation extended from the sea surface to 430 m depth, with an average cycle length of 1.28 km. The SeaSoar observations are gridded with 1 km horizontal and 0.5 m vertical resolution before analysis.

The potential density. $\sigma$, and soundspeed, $c$, are computed from the CTD data using the Thermodynamic Equation of SeaWater 2010 (TEOS-10) \cite{TEOS-10}. A local value for spice, $\gamma$, is calculated following \cite{klymak2015spice}. Acoustically significant variation is observed in the soundspeed profiles, often over short distances, represented here by the two profiles at 249 and 252 km in Fig. \ref{fig:profiles}.

\begin{figure}
\includegraphics{../figures/sld_profile.png}
    \caption{\label{fig:profiles}{(a) Potential density (b) spiciness and (c) sound speed measured at 249 km (blue curves) and 252 km (orange curves).}}
\end{figure}

A simple metric for mixed layer acoustic propagation viability is the sonic layer depth (SLD), defined as the maximum sound speed depth above the thermocline. The SLD increases over the transect, a linear regression fit begins at 78.5 m and has a slope of 2 m / 100 km. The density at 40 m depth decreases with range, with a linear regression intercept at $\sigma=25.3$ kg/m$^3$ and a slope of -0.04 (kg/m$^3$) / 100 km. The decrease in density means that as range increases, mixed layer isopycnals transition to positions with stable stratification below the mixed layer.

\begin{figure}
\includegraphics{../figures/sound_speed_transcet_sld.png}
\caption{\label{fig:transcet}{Sound speed observation of the mixed layer, the sonic layer depth is black line.}}
\end{figure}

\section{\label{sec:decomposition}Spice and tilt decompostion}
A dynamic decomposition is used to separated the observed sound speed variations into contributions from: (1) isopycnal tilt and (2) density compensated sound speed variation proportional to ocean spice. This decompostion follows the method proposed for a similar SeaSoar transcet by \citet{dzieciuch2004}, although spice in the mixed layer is treated differently here. The decomposition is first described for the case where spice is observed in a stratfied ocean, which is applied below the mixed layer. A modified treatment of spice in the margninally stratified mixed layer is discussed next. Finally, the decomposed sound speed fields are compared.

The dynamic decomposition begins by defining a field quanitity, $\sigma$ is used as a simple demonstration, by interpolation of values along isopycnals. This definition allows for the generation of new fields that vary only in isopycnal positions, used to correct for dynamic isopycnal tilt. The isopycnal positions are defined as $\sigma(x, z) = z^{-1}(x, \sigma)$, and are discretized as $z_i(x; \sigma_i)$. A two dimensional linear interpolation is defined for $\sigma$ as
\begin{equation}
    \sigma\approx\mathcal{L}(x, y; \sigma_i, z_i).
    \label{eq:lin_intr}
\end{equation}
The positions $z_i(x, \sigma)$ are then modeled as the superposition of finescale dynamics and a stable background position $\bar{z}(x, \sigma)$. The background density field, $\bar{\sigma}$, is defined by substituting $\bar{z}(x, \sigma)$ for $z(x, \sigma)$ in Eq. \eqref{eq:lin_intr}. The estimate of $\bar{\sigma}$ is then a vertically streched version of the observed $\sigma$ field without finescale tilt dynamics.

The same isopycnal surfaces can be used to define and then compute estimates of other field quantites without finescale isopycnal tilt. While sound speed is ultimately the field quantity of interest, the variable of ocean spice is used here instead,
\begin{equation}
    \gamma=\textrm{sgn}(T-\bar{T}) \sqrt{\alpha^2(T-\bar{T})^2 +\beta^2(S-\bar{S})^2},
    \label{eq:gamma}
\end{equation}
where $\bar{T}$ and $\bar{S}$ are the isopycnal mean of temperature and salinity, and a representative value is used for $\alpha=\partial \sigma / \partial T$ and $\beta=\partial \sigma / \partial S$. $\gamma$ is potential variable that represents linearized distance in density units along (T, S) isopycnals. Following Eq. \eqref{eq:lin_intr}, the interpolated $\gamma$ field is defined
\begin{equation}
    \gamma\approx\mathcal{L}(x, z; \gamma_i(x), z_i),
    \label{eq:lin_intr_gamma}
\end{equation}
where the range dependent $\gamma_i(x)$ is measured along the $\sigma_i$ isopycnal. The accuracy of Eq. \eqref{eq:lin_intr_gamma} depends on the ocean statification through the positions of the isopycnals, $z_i$, and has the smallest error in well stratified regions.

The $\gamma$ field in Eq. \eqref{eq:lin_intr_gamma} allows for the additional definition of a stable background value along isopycnals, $\hat{\gamma}_i(x)$. This leads to four possible $\gamma$ fields with the combination of paramter pairs: the stable field, ($\bar{\gamma}_i$, $\bar{z}_i$); the tilt field, ($\bar{gamma_i}$, $z_i$); the spice field, ($gamma_i$, $\bar{z}_i$); and the total field, ($gamma_i$, $z_i$).

when $\gamma$ has significant non-linear variation over depths with little $\sigma$ gradient. In well stratified areas of the ocean, however, Eq. \ref{eq:lin_intr} and Eq. \ref{eq:lin_intr_gamma} are accurate and suffienct to compute soundspeed with the inverse of Eq. \eqref{eq:gamma}, $c(S, T, p)=c(\gamma^{-1}, p)$.

 The background isopycnal position is estimated with a spatial lowpass filter. This study uses a cutoff of 50 km for the estimate of $\bar{z}(x, \sigma)$, which is shown as red lines in Fig. \ref{fig:cntrs}(a). The estimated position where the isopycnal moves below the mixed layer is used as the start of the $\bar{z}(x, \sigma)$ estimate, apparent around 250 km for $\sigma=25.25$ (kg/m$^3$). A conservative estimate is made for the start position of $\bar{z}(x, \sigma)$ is made by eye. In contrast, isopycnals in the mixed layer are considered unstable and no attempt is made to estimate a background isopycnal position.

with a fine $\sigma_i$ spacing in the mixed layer and a coarser spacing for deeper isopycnals. The position of three isopycnals are shown in Fig. \ref{fig:cntrs}(a). Isopycnals below the mixed layer, \emph{i.e.} $\sigma = 25.29$ kg/m$^3$, are continous in $x$ and vary relativly slowly in $z$. Isopycnals above the mixed layer, \emph{i.e.} $\sigma = 25.21$ kg/m$^3$, may outcrop, causing discontinuities in $x$, and vary rapidly in $z$. The large scale trend of decreasing mixed layer density with $x$ means that isopycnals transistion from unstable mixed layer isopycnals to stable isopycnals below the mixed layer.

The dynamic decomposition uses spatial lowpass filtering with a 50 km cutoff to estimate stable isopycnal position and spice estimates, $\hat{z}_i(x; \sigma_i)$ and $\hat{\tau}_i(x, \sigma_i)$. The background and spice fields use $\hat{z}_i(x; \sigma_i)$ in place of $z_i(x; \sigma_i)$ in the interpolation of both $\sigma$ and $\tau$. The background and tilt fields use $\hat{\tau}_i(x; \sigma_i)$ in place of $\tau_i(x; \sigma_i)$ in the interpolation of $\tau$.

 Internal waves largely followed the Garret-Munk spectrum up to the mixed layer, while eddies were the largest contribution to tilt in the mixed layer.  The decomposition works by contouring the observed field into isopycnals, and then spatially lowpass filtering isopycnal position and spice along the isopycnal to estimate an unperturbed background condition.

\begin{figure}
\includegraphics{../figures/sig_tau_interp.png}
    \caption{\label{fig:cntrs}{Contours of: (a) Potential density (b) spiciness. The low pass estimate of stable isopycnal position is shown in red in panel (a). The full record shows isopycnals shoal and enter the mixed layer with decreasing range. The stable isopycnal estimate ends at the first location the isopycnal enters the mixed layer. Intermittant outcropping of the isopycnal are apperent as gaps in the $\sigma=25.21$ (kg/m$^3$) isopycnal position. The spice record has been imputed at the isopycnal outcroppings to form a continous series.}}
\end{figure}

Estimation of the stable isopycnal position and spice with a lowpass filter is most effective outside of the mixed layer where the ocean is well stratified in regions of significant spice variation. Two challenges in the dynamic decomposition of \cite{dzieciuch2004} arrise in the mixed layer: (1) significant variations of $\tau$ occcur in positions with small $\sigma$ gradients, and (2) isopycnal locations vary rapidly and may not have a stable position. The $\sigma$ field is only effected by (2), which is treated by truncating $\hat{z}_i(x; \sigma_i)$ when an isopycnal shows significant surface excursions or outcropping. The last stable isopycanl density is then extrapolated to the surface, resulting in a completely mixed layer. However, both challenges are significant to the construction of $\gamma$ since this interpolation depends on the isopycnal positions.

These challenges are particularly relevant the construction of the $\gamma$ fields, the 

Both challenges are related to the constraint of sampling the spice field at isopycnal positions imposed by the dynamic decompostion of \cite{dzieciuch2004}, and a and a linear superposition model for $\gamma$ in the mixed layer is proposed to avoid this sampling requirement.

First, while the error of the interpolator for $\sigma$ can be made small by increasing the number of isopycnals, the position of isopycnals may not occur in regions of significant spice variation. While dynamic stability limits on N$^2$ prevent $\partial \sigma / \partial z$ from reaching 0, observations come close to the theoretical sampling limit where significant changes in $\partial \tau / \partial z$ occur at positions with very small changes in $\sigma$.


 Inside the marginally stratified mixed layer, however, non-linear spice variation can occur between isopycnals with very fine spacing. An example of two mixed layer $\sigma$ and $\gamma$ profiles are shown in Fig. \ref{fig:profiles} (a) and (b), which were taken 3 km appart. Both $\sigma$ profiles are well mixed above the pycnocline at around 110 m, and the vertical spacing of isopycnals is large. However the profile at 249 km shows significant spice variation above the pycnocline, and a linear interpolation for $\gamma$ that uses isopycnal positions will be undersampled in this region.

Secondly, the lowpassed isopycnal positions are an estimate of where the isopycnals would be in the absence of isopycnal tilt from internal waves and eddies. While isopycnals are observed in the mixed layer, the position of isopycnals in the mixed layer are highly variable and yield physically implausable lowpassed positions. In keeping with the well mixed assumption, this study does not attempt to estimate a stable position for isopycnals in the mixed layer.

A linear superposition model is used instead of a linear interpolator for the estimate of $\tau$ in the mixed layer. In essence, all observed $\tau$ variablitly that is not explained by the tilt field is included in the spice field, without any warping to account for the mean isopycnal levels. The linear superposition model is written
\begin{equation}
    \tau_{observed} = \tau_{bg} + \Delta \tau_{tilt} + \Delta \tau_{spice},
\end{equation}
with the spice field is defined as $\tau_{bg} + \Delta \tau_{spice}$, and similarly for tilt. The value of $\Delta \tau_{spice}$ is be estimated by subtracting $\tau_{tilt}$ from $\tau_{observed}$.

\begin{figure}
\includegraphics{../figures/rms_profile.png}
    \caption{\label{fig:c_rms}{The mean sound speed profile and the RMS profile of the deviation from the background field of the measured, tilt and spice fields. }}
\end{figure}

\section{\label{sec:propagation}Acoustic propagation with separated spice and tilt}
\begin{figure}
\includegraphics{../figures/decomp_xmission.png}
    \caption{\label{fig:decomp_x}{Left panels is sound speed field, right panel is acoustic pressure. The rows are the (a) tilt, (b) spice and (c) total fields.}}
\end{figure}


\bibliographystyle{jasanum2}
\bibliography{eRichards_master}

\end{document}
