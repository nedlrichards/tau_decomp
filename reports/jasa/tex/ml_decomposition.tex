\documentclass[preprint,NumberedRefs]{JASA}
\usepackage{multirow}
\usepackage{booktabs}

\begin{document}
\title[Mixed layer tilt and spice]{Observations of ocean spice and isopycnal tilt sound speed structures in the open ocean mixed layer, and their impacts on acoustic propagation}
\author{Edward L. Richards}
\email{edwardlrichards@gmail.com}
\affiliation{Ocean Sciences, University of California Santa Cruz, Santa Cruz, California 95064, USA}
\author{John A. Colosi}
\affiliation{Department of Oceanography, Naval Postgraduate School, Monterey, California 93943, USA}
\preprint{E. L. Richards, JASA}	%if you want this message to appear in upper right corner of title page

\date{\today}

\begin{abstract}
A 350-m deep and 970-km long northeastern Pacific Ocean section is decomposed into sound speed variations from tilted isopycnals and ocean spice. Acoustic simulations are made with the decomposed fields for sources both inside the roughly 100 m mixed layer and just below in the 100 m transition layer. Peak root mean square sound speed variation for the decomposed spice field is 0.8 m/s and positioned right below the mixed layer, while the tilt field has 30 m deeper peak of 1.5 m/s. Acoustic energy statistics show both negative and positive variations in energy relative to a background field, with both localized and spatially diffuse energy variation. Mode coupling statistics also demonstrate vertical energy distribution and mode randomization. Localized positions of high loss, or blocking features, are most important at lower frequencies, while diffuse energy loss is more important at higher frequencies. Overall, the largest energy losses are observed for the spice field, although spatial analysis of blocking features indicates regional differences in the relative importance of the tilt and spice fields.
\end{abstract}

\maketitle

\section{\label{sec:intro} Introduction}
Mechanical mixing from wind forcing at the air-sea interface and convection from surface cooling homogenize the upper ocean and contribute to a mixed layer that is deepest in the late winter\citep{cole2010seasonal}. A horizontal and vertical observation of the top 350 m over 970 km of the northeastern Pacific Ocean at the time of the deepest mixed layer extent show acoustically significant variation is common both inside and below the mixed layer\citep{colosi2020observations}. This study investigates the relative impact of two dynamic sources of the observed sound speed variation, tilted isopycnals and ocean spice, on acoustic propagation. These fields are decomposed based on their different effects on the position of, and sound speed properties along, isopycnals\cite{dzieciuch2004}. The sound speed from both dynamic fields shows different spatial structure and root mean squared (RMS) variation in depth. Statistics of the effects of the decomposed fields on upper ocean acoustic propagation are calculated over an ensamble of source positions with two source depths and two frequencies. These results indicate that both spice and tilt dynamics are important drivers of acoustic variation over the 50 km range of study. Finally, changes in relative statistics between the fields across source depth and frequency show differences in sensitivity to each field across the different acoustic propagation scenarios.

The mixed layer acoustic duct (MLAD) is the dominant mechanism for upper ocean acoustic energy propagation for sources near the surface in the northeast Pacific. In the absence of the MLAD essentially all acoustic energy cycles into the deep ocean and returns after a convergence zone length\cite{jensen2011computational}, approximately 55 km for the study region. The canonical MLAD is formed by an upward refracting mixed layer, mixed water has an adiabatic sound speed gradient of approximately 0.016 $s^{-1}$, and the reflecting ocean surface. The acoustic characteristics of the MLAD are distinct from a shallow water duct with the same sound speed structure because it interacts with the deep ocean acoustic channel through diffraction\citep{porter93} and scattering\cite{colosi2020observations}. However, the MLAD is expected to be stable at frequencies above its cutoff\citep{Urick1982Prop} and range independent models predict long range acoustic propagation.

Range or time dependent measurements of the MLAD without dynamic field separation show that it has highly variable sound speed\citep{cole2010seasonal,rudnick1999compensation,klymak2015}, which can significantly alter or completely block acoustic duct propagation\citep{colosi2020observations,colosi21}. However, an acoustically viable MLAD is also predicted with many range dependent sound speed observations, especially at frequencies significantly above the MLAD cutoff. The propagation characteristics of the MLAD also remain important even in scenarios that consider acoustics below the MLAD due to diffraction or range dependent scatter\citep{colosi21}.

The decomposition of the ocean dynamics observed in the MLAD separates the observation based on density. Many ocean dynamics are characterized by variation of isopycnal height, including internal waves, eddies and surface forced stratification. These dynamic processes appear as tilted isopycnals in range, and the aggregate of these processes is termed the ``tilt'' field\cite{dzieciuch2004}. Many of the processes that make up the tilt field have characteristic length and time scales, and some features like eddies can be deterministically modelled in ocean models. The internal wave field can be statistically described with the Garret-Munk spectrum\cite{garrett1972space}. The expected internal wave field describes most of the RMS tilt field in the stratified transition layer (TRL) just below the MLAD. Other tilt processes like the restratification of the mixed layer at the beginning of spring are still areas of active study\cite{cole2010seasonal}. While the description of dynamic processes are related to density structure and hence do not uniquely specify water properties, this study assumes that the tilt field has spatially stable properties along each isopycnal.

In addition to tilt dynamics, range dependence of ocean slainity and temperature properties is expected due to stochastic surface forcing and the changing bulk properties of the surface ocean\citep{ferrari2000}. The variation of temperature and salinity in the upper ocean tends to be highly correlated, and the relative effects of these changes moderates density changes. On large scales, temperature varies at twice the rate of salinity moderation. On shorter scales, however, the temperature and salinity variations typically compensate in density. Density compensated waters are either relatively warm and salty or cool and fresh, and this variation is termed ocean "spice"\citep{munk1981evolution}. The two scales of range variation are rectified at ocean fronts where density changes abruptly, and these sharp density gradients can lead to further enhancement of upper ocean mixing through symmetric instabilities\citep{dasaro2011}. Similarly, density compensated spice variation at smaller scales tends to be front like, although mixed layer observations discussed here often show acoustically significant vertical variation at these fronts.

Importantly for the acoustic problem, density compensated temperature and salinity variations are reenforcing in sound speed. Although spice variability depends on geographical location, studies have shown that it can significantly change acoustic propagation\citep{colosi12,colosi13,murat2021}. The importance of spice in geographically separated acoustic environments, and the variation of spice over time remain topics of active research.

The observed sound speed variability is a combination of both tilt and spice, which are decomposed here to isolate both effects following Dzieciuch \emph{et al.}\citep{dzieciuch2004}. This method is modified in the mixed layer, where a linear superposition model is introduced to estimate the spice field. In this study spice is quanitified using salinity, which can be combined with density to quantify both temperature and sound speed. The superposition method used here includes the complete salinity observation in either the tilt or spice field, which is challenging with the method of Dzieciuch \emph{et al.}\cite{dzieciuch2004} due to the rapidly varing isopycnal positions in the mixed layer. Finally, this decomposition also produces a background mixed layer enviornment without dynamics, which is analyized to show regional differences in diffraction across the transect.

Three upper ocean acoustic propagation scenarios are simulated for the decomposed fields at 400 and 1000 Hz with the parabolic equation (PE) model RAM\cite{collins93}. The average transect sound speed profile predicts one and three modes trapped in the MLAD at these frequencies, respectively, and this comparison allows for a comparative study of frequency dependent mode coupling. Transect statistics are computed from an ensemble of source positions separated by 10 km, which are analyzed relative to background field propagation. Acoustic propagation is discussed both for sources inside the MLAD and the roughly 100-m TRL. Average acoustic energy is computed in both the MLAD and the TRL. These simulations are made over a transmission length of one convergence zone\citep{jensen2011computational}, approximately 50 km in the subtropical Pacific Ocean, which removes the consideration of coupling into the duct at convergence zones outside of the source region\citep{colosi2020observations}.

Energy statistics are reported at two ranges, just after the source region, and just before the first convergence zone. Energy changes over the transmission range are also investigated with mixed layer mode magnitudes, which show both vertical energy distribution and mode randomization. Additionally, each transect is analyzed for blocking features\citep{colosi2020observations}, or positions of large MLAD energy loss over a short distance. Comparisions between statistics with and without these features show the relative importance of energy loss from localized events and diffuse scatter. Combined together, these results show the relative importance of the dynamic fields in across transmission scenarios, which future studies may then compare to other MLAD enviornments.

The paper is organized as follows. Section~\ref{sec:transcet} describes the oceanographic mixed layer observations that motivate this study. Section~\ref{sec:decomposition} discusses the dynamic decomposition of the observed sound speed field into spice and tilt fields. Acoustic propagation through the separated sound speed fields is compared with the observed field in Sec.~\ref{sec:propagation}. Finally, Sec.~\ref{sec:conclusion} presents the study conclusions.

\section{North Pacific transect}\label{sec:transcet}
\begin{figure}
\includegraphics{../figures/sound_speed_transcet_sld.png}
\caption{\label{fig:c_grid}{Sound speed transect observation of the mixed acoustic duct and the upper transition layer. Black line is the sonic layer depth.}}
\end{figure}

This study is based on a 970 km SeaSoar conductivity, salinity, and depth (CTD) transect taken over 4 days in the northeast Pacific ocean,\citep{cole2010seasonal}. The 970 km transect followed a great circle path roughly parallel to the sea surface temperature contours, although minor differences in slope between these two predict a slight warming of the mixed layer over the transect. SeaSoar CTD observations extend from the sea surface to 430 m depth, with an average cycle length of 2.6 km. These observations were first interpolated from the vehicle's sawtooth path to a grid with 1 km horizontal and 0.5 m vertical resolution\citep{colosi2020observations}, and World Ocean Atlas climatology is appended to a depth of 4000 m\citep{WOA}. Sound speed, $c$, computed from the CTD data with the Thermodynamic Equation of Seawater 2010,\cite{TEOS10} is shown in Fig. \ref{fig:c_grid}. The MLAD is apparent as the relatively high sound speed over a strong negative sound speed gradient that extends into the deep ocean. Horizontal and vertical variability in sound speed is apparent throughout the MLAD observations caused by internal waves, fronts, eddies, and ocean spice\citep{colosi2020observations}.

The sonic layer depth (SLD) is a simple metric of the MLAD, defined as the depth of maximum sound speed above the thermocline, and shown as a black line in Fig.~\ref{fig:c_grid}. The MLAD is expected to strengthen with increasing SLD, a quantification of this is the mode cutoff frequency\citep{Urick1982Prop},
\begin{equation}
    f_{min}(n) = \frac{3(4n-1)}{16} \sqrt{\frac{c_0^3}{2h^3} \left( \frac{\partial c}{\partial z} \right) ^ {-1}},
    \label{eq:f_cutoff}
\end{equation}
where $f_{min}$ has units of Hz. The SLD is $h$, and a refence sound speed of $c_0 = 1500$ m/s is assumed. Several observations on mode cutoff frequency over the complete transect from Colosi and Rudnick\cite{colosi2020observations} are summarized here. Sound speed gradients vary from 0 to above 0.1 and are on average higher than the adiabatic gradient of 0.016 $s^{-1}$ of a homogeneous mixed layer. The first four cutoff frequencies for the transect average sound speed profile are 210, 480, 760 and 1040 Hz, although individual profiles may have cut offs above one or both frequencies.

\begin{figure}
    \includegraphics{../figures/sld_profile.png}
    \caption{\label{fig:profiles}{(a) Potential density, $\sigma$ (b) absolute salinity, $S_A$ and (c) sound speed, $c$, measured at 249 km (blue curves) and 252 km (orange curves). The mixed layer depth is shown as horizontal pink lines in (a), and the sonic layer depth is shown as horizontal black lines in (c).}}
\end{figure}

The acoustical relevance of ocean spice is illustrated first with an example of two profiles measured at 249 and 252 km transect range, shown in Fig.~\ref{fig:profiles}. The two profiles at 249 and 252 km have significantly different SLDs, shown as horizontal grey lines in Fig. ~\ref{fig:profiles}(a), of 45 and 89 m, respectively. The difference in $f_{min}$ predicted by this SLD change is moderated in part by the sound speed gradients of 0.24 and 0.15 s$^{-1}$. Overall, Eq.~\eqref{eq:f_cutoff} predicts cutoff frequencies of 485 and 222 Hz, indicating the relative importance of SLD compared to the sound speed gradient. This change in $f_{min}$ is not reflected in the density profile, however, which is roughly constant for both profiles to the TRL start at 90 m depth. Instead, the 249 km profile has positive gradients in tempurature and salinity to 45 m depth then followed by negative gradients. The variation of these properties up to the TRL start is mostly due to spice, because the relative gradients of temperature and salinity essentially compensate in density.

\begin{figure}
\includegraphics{../figures/ts_mean_c.png}
    \caption{\label{fig:ts_diagram}{Temperature and salinity observed along transect, color of markers indicates transect range of observation. Two lines in (a) show averages made over 50 km and centered at 100 (solid) and 900 km (dashed).}}
\end{figure}
Transect wide difference between tilt and spice dynamics are next demonstrated with a conservative temperature and absolute salinity diagram\cite{TEOS10}, ($\theta$, $S$), shown in Fig.~\ref{fig:ts_diagram}(a). The markers of each observation become lighter with increasing transect range, $x$. The potential density and sound speed are shown as darker and lighter contours, respectively. The potential density is dynamically stable and increases with depth, in rare cases the gridded data was adjusted to enforce this requirement of minimal dynamic stability\citep{barker2017stabilizing}. Ocean spice is observed in the spread in ($\theta$, $S$) parallel to the density contours, both as regional differences between markers of different colors and local variation between markers of similar color.

Spice is quantified relative to expected ($\theta$, $S$), computed here with a 50 km running mean in transect range. The mean ($\theta$, $S$) at $x$ of 100 and 900 km are shown as solid and dashed lines red lines in Fig.~\ref{fig:ts_diagram}(a), respectively. TRL waters make up the majority of the observations and have a positive slope, and ($\theta$, $S$) increase as density decreases. The steeper slope at 900 km shows a larger negative sound speed gradient compared with 100 km, driven by a salinity minimum near the TRL base. MLAD waters are a small portion of the lightest waters, where the two profiles have almost opposite slopes. The 900 km profile shows decreasing ($\theta$, $S$) with decreasing densitity. This MLAD stratification will increase the positive sound speed gradient and cutoff frequency. An example of this is shown above the SLD at 249 km in Fig.~\ref{fig:profiles}, and has also been discussed for a profile near the transect center.\cite{colosi2020observations} In contrast, the 100 km profile shows increasing ($\theta$, $S$) with decreasing density, and MLAD stratification will reduce the sound speed gradient and cutoff frequency.

The spice variation along isopycnals is shown in sound speed at a reference pressure of 0 dbar in Fig.~\ref{fig:ts_diagram}(b). Separate from regional level variation, sound speed variation mostly decreases with increasing density. The smallest variation is observed for the densest waters measured at the top of the permanent thermocline\cite{cole2010seasonal}. In contrast, the magnitude of the sound speed changes often exceeds 2 m s$^{-1}$ in the TRL and approaches 4 m s$^{-1}$ in the lightest waters of the MLAD. This is consistant with previous observations of spice variance along isopycnals, which showed a roughly linear and inverse relationship with density in the upper ocean\cite{ferrari2000}.

\section{\label{sec:decomposition}Spice and tilt decomposition}
The observed sound speed variations are decomposed into contributions from isopycnal tilt and spice following a method proposed for a similar SeaSoar transect by Dzieciuch \emph{et al.},\citep{dzieciuch2004}. However, spice in the marginally stratified mixed layer is treated differently here. This decomposition is first described for the case of the stratified TRL below the mixed layer. A linear superposition model for spice is then introduced for the mixed layer, and the results are compared. Finally, the RMS sound speed variations of the decomposed components are compared over the entire survey depth.

\subsection{Stratified ocean decomposition}
\begin{figure}
\includegraphics{../figures/sig_tau_interp.png}
    \caption{\label{fig:cntrs}{Isopycnal position, $z_i(z; \sigma_i)$ are shown in grey. Lowpass isopycnal position estimates, $\bar{z}_i(z; \sigma_i)$, made in the stabily stratified transition layer, are shown in red.}}
\end{figure}

The dynamic decomposition begins by defining a gridded quantity, $\sigma$ is used for demonstration, as an interpolation of values between isopycnals. The isopycnal $z$ position is defined $\sigma(x, z) = z^{-1}(x, \sigma)$ and discretized as $z_i(x; \sigma_i)$. A two-dimensional linear interpolation is defined for $\sigma$,
\begin{equation}
    \sigma(x,y)\approx\mathcal{L}(x, y; \sigma_i, z_i).
    \label{eq:lin_intr}
\end{equation}
The isopycnal positions $z_i(x, \sigma)$ are modeled as the superposition of fine scale dynamics on a stable background position $\bar{z}(x, \sigma)$. The background density field, $\bar{\sigma}$, is defined by substituting $\bar{z}(x, \sigma)$ for $z(x, \sigma)$ in Eq. \eqref{eq:lin_intr}. The estimate of $\bar{\sigma}(x,y)$ is a vertically stretched field without fine scale tilt dynamics.

The background isopycnal positions, $\bar{z}_i(x, \sigma)$, are estimated with a spatial 50 km low-pass filter. These estimates of $\bar{z}_i(x, \sigma)$ are shown as red lines in Fig. \ref{fig:cntrs}. The position of isopycnals inside the mixed layer are unstable, \emph{e.g.} $\sigma=25.17$ and 25.21 kg/m$^3$, and poorly described as a perturbation from a low-passed position. Therefore, instead of tracking the unstable positions of isopycnals in the mixed layer, the dynamic decomposition only estimates the mean position for isopycnals after they have clearly entered the statified TRL. A conservative estimate is used for the start of the stable contour, for example, the $\sigma=25.25$ kg/m$^3$ estimate begins at $x=270$ km after it begins to closely follow the denser isopycnals.

The stable isopycnal position estimates can similarly define the spice field without fine scale isopycnal tilt,
\begin{equation}
    S(x, z)\approx\mathcal{L}(x, z; S_i(x), z_i),
    \label{eq:lin_inter}
\end{equation}
where S$_i$ is the salinity along the $i$-th isopycnal. Salinity is chosen here to quantify ocean spice, which is then combined with density to uniquely specify the sound speed field\citep{TEOS10}. As described along with Fig.~\ref{fig:ts_diagram}, a regional salinity estimate $\bar{\textrm{S}}_i(x)$ is estimated with a 50 km moving average filter along each isopycnal. The moving average both acts as a lowpass filter and handles missing values at ranges without $\sigma_i$ observations.

Finally, combinations of the low-pass and observed values of $\sigma_i$ and S$_i(x)$ can be substituted into Eqs. \eqref{eq:lin_intr} and \eqref{eq:lin_inter} to produce four S$(x,z)$ fields. The stable background field is computed with ($\bar{\textrm{S}}_i$, $\bar{z}_i$), tilt with ($\bar{\textrm{S}}_i$, $z_i$), and spice with (S$_i$, $\bar{z}_i$). Finally, the observed field is conceptually produced from (S$_i$, $z_i$).

\subsection{Mixed layer decomposition}\label{ssec:ml_decomp}
\begin{figure}
\includegraphics{../figures/sound_speed_comp.png}
    \caption{\label{fig:c_diff}{Difference in spice field sound speed between the stratified decomposition and the linear superposition methods. The two decompositions are equal at the stable position of the lightest tracked isopycnal. The discontinuity in stable isopycnal position at 270 km is the last tracked position of isopycnal $\sigma=25.25$ kg/m $^3$. Significant vertical variation of sound speed is apparent in the mixed layer above the last tracked position of the isopycnal.}}
\end{figure}

Estimation of the stable isopycnal position and spice with a low-pass filter is most effective outside of the mixed layer where the ocean is well stratified in regions of spice variation. Two challenges to this approach arise in the mixed layer: (1) significant spice variations occurs in positions with small $\sigma$ gradients, and (2) isopycnal locations vary rapidly without a clear stable position. A linear superposition model for salinity is proposed here for the mixed layer to avoid the issue of sampling salinity along isopycnals.

The linear superposition model for salinity attributes all observed variability not explained by the tilt field to the spice field, without any vertical stretching to account for the mean isopycnal levels. The linear superposition model is written
\begin{equation}
    \textrm{S}_{observed} = \textrm{S}_{bg} + \Delta \textrm{S}_{tilt} + \Delta \textrm{S}_{spice},
    \label{eq:lin_sup}
\end{equation}
where S$_{bg}$ is the background spice field. The spice field is defined as S$_{bg} + \Delta \textrm{S}_{spice}$, and the tilt field is S$_{bg} + \Delta \textrm{S}_{tilt}$. The value of $\Delta \textrm{S}_{spice}$ is estimated with Eq.~\eqref{eq:lin_sup} by subtracting S$_{tilt}$ from S$_{observed}$. This approach uses the background and tilt fields, which do not require vertical stretching of mixed layer isopycnals, to estimate the spice field from the observed field.

The sound speed field is compared between the stratified decomposition of Eq.~\eqref{eq:lin_inter} and the linear superposition model of Eq.~\eqref{eq:lin_sup} in Fig.~\ref{fig:c_diff}. The difference between these methods begins above the last stable isopycnal position. While the difference field often shows uniform vertical structure, there are also locations with significant vertical variation. Sampling this spice variation with the stratified decomposition would require stable isopycnal positions well into the mixed layer. The linear superposition model avoids the need to introduce non-physical stable isopycnal positions to sample the observed spice variation in the mixed layer.

\begin{figure}
\includegraphics{../figures/diff_fields.png}
        \caption{\label{fig:c_fields}{Total and decomposed fields. Background field is shown in (a). Dynamic fields are shown with the background field subtracted: (b) tilt, (c) spice and (d) observed. The black line in each pannel is the SLD.}}
\end{figure}

The background and dynamic fields computed with the dynamic decomposition are shown in Fig.~\ref{fig:c_fields}. The background field, Fig.~\ref{fig:c_fields}(a), shows a high sound speed MLAD that overlies slower sound speeds in the TRL. The SLD of the background field varies smoothly in range except at discontinuous positions where an isopycnal enters the mixed layer, Sec. \ref{ssec:ml_decomp}. The change in SLD, even at the position of discontinuities, is relatively minor compared with that of the dynamic fields shown in the bottom three panels, and the background field is predicted to produce marginal range dependent effects.

The difference in sound speed from the background for the tilt, spice, and observed fields are shown in panel (b), (c), and (d), respectively. The highest sound speed variations are from internal waves in the tilt field TRL. The tilt field also shows large scale sound speed anomalies in the top 50 m, mostly positive in the first half of the transect and negative in the second, which is likely due to differences in surface stratification forcing\cite{colosi2020observations}. While the magnitudes of the tilt anomaly in the MLAD is relativly low, it has an important effect on the SLD. The SLD is often significantly shallower than the background field for positive tilt sound speed anomalies in the first half of the transect, which sugggests the tilt field at these positions decreases the MLAD sound speed gradient discussed with Eq.~\eqref{eq:f_cutoff}. The nagative sound speed anomalies observed in the second half of the transect are associated with deeper SLD observations, consistant with observed profiles with both positive sound speed gradients and significant stratification\cite{colosi2020observations}.

The spice field has smaller magnitudes than the tilt field overall, and many of the spice features extend over the entire MLAD depth. Spice contributions to sound speed also appear in the TRL, where they often appear tilted with depth. The effect of spice on the SLD is more intermittent than the tilt field. A clear separation exists between the relatively flat SLD in the first half of the transect, and jagged and often shallow SLD in the second half of the transect. Finally, the observed field contains a clear superposition of the features in the spice and tilt fields, which supports the dynamic decomposition for the discussion of observed sound speed fields.

\subsection{Decomposed sound speed statistics}
\begin{figure}
\includegraphics{../figures/rms_profile.png}
        \caption{\label{fig:c_rms}{The transect mean sound speed profile and RMS measured, tilt and spice field deviations from the background. The SLD is shown as a horizontal line in both panels. A dashed line shows a fit to the sound speed profile above the SLD with a slope of 0.024 s$^{-1}$.}}
\end{figure}

The transect mean sound speed profile, $\bar{c}$, and RMS statistics of each dynamic field are shown in Fig.~\ref{fig:c_rms}. A gradient of 0.024 s$^{-1}$ fits $\bar{c}$ well up to the 80 m SLD, which is followed by a sharp sound speed decrease in the TRL. The RMS statistics of the observed, tilt and spice fields, shown in Fig.~\ref{fig:c_rms}, all have maximum values in the TRL. The tilt and observed fields have similar maximum values of more than 1.5 m/s, almost 50 m below the MLAD. The spice field has a smaller maximum around 1.0 m/s, but it is also the shallowest maximima just below the SLD. This maximum occurs much deeper in range averaged calculations than in isopycnal avaraged calculations, Fig.~\ref{fig:ts_diagram}(b), because of the influence of tilted isopycnals.\citep{ferrari2000} Importantly, the spice maximum occurs close to the tilt field minima and has a large impact on the total RMS at the base of the MLAD.

The tilt field maxima in the TRL is well predicted by the GM internal wave spectrum, shown as a dashed green line in Fig.~\ref{fig:c_rms}. The GM spectrum is calculated from the bouyancy frequency, $N$, of the mean potential density profile. The RMS GM displacement is computed with the WKB method as
\begin{equation*}
    \zeta_{RMS}(z) = \zeta_0 \sqrt{\frac{N_0}{N(z)}},
\end{equation*}
where the standard GM values used are $\zeta_0$ = 7.3 m and $N_0=3$ cycles per hour. The RMS sound speed is related to this displacement by
\begin{equation*}
    \delta c_{RMS} = \bar{c}(z+\zeta)-\big(\bar{c}(z) - \gamma \, \zeta_{RMS}(z)\big),
\end{equation*}
where $\gamma$ is the adiabatic sound speed gradient of approximatly 0.016 m s$^{-1}$. The tilt RMS also has a second local maxima at the surface, which is not explained by the GM prediction. Despite the relativly smaller variance, this surface maxima is also expected to be acoustically relevant to the propagation discussed here because it is inside the MLAD.

\begin{figure}
\includegraphics{../figures/diff_spectra.png}
        \caption{\label{fig:spectra}{Power spectral density of sound speed anomoly in the total, tilt, spice, and background fields in the (a) MLAD, (b) TRL and (c) thermocline.}}
\end{figure}
Next, the power spectral density of the sound speed anomoly in the MLAD, TRL, and thermocline are shown in Fig.~\ref{fig:spectra}. Following Colosi and Rudnick\cite{colosi2020observations}, each spectrum is an average of 4 depths: 20, 40, 60 and 80 m for the MLAD; 100, 120, 140, 160 m for the TRL; and 180, 200, 220, 240 m for the thermocline. The mean background profile is subtracted from the background field, while the background field is subtracted from each of the dynamic fields. The background field has a sharp fall off for wavenumbers above 0.02 cpkm (50 km wavelength), and the subtraction of the background from the dynamic fields generally leads to a peak at 0.02 cpkm and decreases with lower wavenumbers. The mixed layer has comparativly little falloff at low wavenumbers, and the tilt field variance increases below 0.02 cpkm. An example of this long range structure is the overall positive tilt sound speed anomaly in the MLAD between 0 and 450 km. In contrast, the spice field has a factor of 10 more energy than the tilt field at wavelengths shorter than 50 km.

The TRL and upper thermocline spectra have similar shapes. For the TRL, the peak of the tilt and total fields at 50 km wavelength is a factor of 10 above the spice field. In the upper thermocline the contribution of tilt and spice contribute equally to the total sound speed variation. Equipartition observed between the tilt and spice field over much of the spectrum is a result of the shallow sampling depths, and the RMS shown in Fig.~\ref{fig:c_rms} indicates tilt variation begins to dominate below 250 m depth as spice variation approaches zero.

\section{\label{sec:propagation}Upper ocean acoustic propagation}
\begin{figure}
\includegraphics{../figures/decomp_xmission.png}
    \caption{\label{fig:decomp_x}{Left panels are sound speed field; right panels are acoustic pressure for MLAD source. Rows are the: (a) background (b) tilt, (c) spice, and (d) observed fields. The source region before 7.5 km has significant down going energy for all fields, and the up going energy of the first convergence zone begins 47.5 km from the source. Significant MLAD loss between these regions appears as down going energy below the approximately 100 m deep MLAD. MLAD loss is strongest around 250 and 260 km for the spice and total fields, and smaller loss is also observed in the tilt and total fields around 240 km.}}
\end{figure}

The acoustic effects of tilt and spice are analyized with 60 km propagation sections separated by 10 km transect range. This separation was chosen as a compromise that samples the observed range variation with reasonable independence. Two source frequencies, 400 and 1000 Hz, are used for comparison of propagation with an average of one and three mixed layer modes, respectively. Three upper ocean propagation scenarios are investigated, the first analyizes acoustic energy in the MLAD for a source at 40 m depth in the MLAD (MLAD-MLAD). The second analyizes the non-ducted energy in the TRL for the MLAD source (MLAD-TRL). Finally, the acoustic energy in the MLAD is analyized for a source in the TRL at 200 m depth (TRL-MLAD).

Figure~\ref{fig:decomp_x} shows a representative section of propagation at 400 Hz for a MLAD source at $x=$230 km, modeled with the PE solver RAM\citep{collins93}. The background, tilt, spice, and observed sound speed fields are shown in the left column of Fig.~\ref{fig:decomp_x}. The PE results are shown in the right column of Fig.~\ref{fig:decomp_x}. The background acoustic field, Fig.~\ref{fig:decomp_x}(a), shows basic MLAD propagation. The MLAD is obscured by high angle propagation up to about 7.5 km from the source, and in the first convergence zone starting around 47.5 km. An absorbent layer introduced at the bottom of the PE domain suppresses bottom interactions that would otherwise contribute non-ducted energy to the MLAD at source ranges between these regions. With the higher angle arrivals removed, the MLAD between the source region and the first convergence zone has no external sources of acoustic energy, one maximum in depth, and a slow energy decrease with range.

The tilt and spice fields are shown in Fig.~\ref{fig:decomp_x} (b) and (c), respectively. The most acoustically significant tilt field feature is SLD shoaling out to 240 km that leads to MLAD energy loss. A surface concentration of lower sound speed also appears in the tilt field after $x=270$ km. This tilt feature increases the mean sound speed gradient of the mixed layer, both strengthening the MLAD and moving acoustic energy shallower.

The spice field has significant sound speed variation in the MLAD, with both vertical fronts and features with depth variation. The most acoustically significant is a high sound speed feature between 250 and 260 km transect range. This feature causes concentrated loss of acoustic energy from the MLAD at the front-like edges. There is also a marked change to the vertical pressure distribution that indicates coupling between different MLAD modes. The energy loss is characteristic of a blocking feature that significantly reduces the viability of the MLAD, and also briefly increases energy in the TRL.

Finally, the observed field appears as a combination of the tilt and spice fields, with many of the distinct features partitioning almost exclusively into one field or the other. The acoustic propagation also shows most of the same behaviors observed for the spice and tilt fields, although it is not expected to be an exact superposition.

\subsection{Upper ocean acoustic energy}\label{ssec:bg}
Propagation effects, like those discussed with Fig.~\ref{fig:decomp_x}, are quantified with both a vertical energy integration and a limited depth projection onto a mixed layer mode (MLM). The vertical integration of PE energy contains no information about its vertical distribution, while the mode projection quantifies MLAD energy with the same vertical distribution as a MLM. A limited depth mode projection is used as an alternative to mode amplitudes from coupled mode calculations in order to approximate orthogonality over the MLAD, since multiple modes have similar shapes in the MLAD.

The spreading compensated vertical integration of acoustic energy is computed from the PE acoustic pressure result, $p(x, z)$,
\begin{equation}
    \textrm{E} = 20 \, \textrm{log}_{10} (z_1 - z_0)^{-1} \left( \int^{z_1}_{z_0} \left| p(x, z) \right| \,  dz \right).
    \label{eq:int_eng}
\end{equation}
The MLAD energy is computed by integration between 0 and 125 m depth to include the deepest SLDs, while TRL energy is integrated between 125 and 250 m depth.

The limited depth mode projection computes the inner product of a MLM, $\psi(z)$, and the PE pressure field up to the $n$-th zero crossing, $z_n$,
\begin{equation}
    \textrm{E}_{\textrm{\vspace{0.001} MLM}} = 20 \, \textrm{log}_{10} \left( \left| \int^{z_n}_0 \,  p(x, z) \ \psi(z) \,  dz \ \right|\right) - 20 \,\textrm{log}_{10} \left( \int^{z_n}_0 \, \left| \psi(z) \right| \,  dz \right).
    \label{eq:proj_eng}
\end{equation}
The second term is a normalization that accounts for the partial mode energy in the limited depth integration of Eq.~\eqref{eq:proj_eng} and is equal to the water density for modes with no energy outside the mixed layer\citep{jensen2011computational}.
\begin{figure}
\includegraphics{../figures/mode_shapes.png}
    \caption{\label{fig:bg_modes}{(a) Background sound speed, range averaged from $x=$230 to 280 km, (b) MLM1 (mode \#238) and surrounding modes at 400 Hz. The sound speed profile is fit to a 0.017 s$^{-1}$ gradient until the 99.5 m SLD. Neigboring modes have a similar shape to MLM1 over much of the MLAD, and also a significant tail down to the approximatly 3200 m MLAD compensation depth.}}
\end{figure}

Three normal modes centered around mixed layer mode 1 (MLM1) for the $x=$230 km RI background field are shown in Fig.~\ref{fig:bg_modes}. The MLM1 is found as the mode with the most MLAD energy within an index of the first peak in loop length\citep{jensen2011computational},
\begin{equation}
    l_{m} = \frac{2 \pi}{k_{m+1} - k_m}.
    \label{eq:loop_length}
\end{equation}
The loop length values around MLM1 are typically 50 km and define the convergence zone length, while MLM1 can be 2-3 times this value. This approach can also identify mixed layer mode 2 (MLM2) and higher.

The shape of MLM1 is not orthogonal to neighboring modes over MLAD depth. Instead, the coherent sum of similarly shaped modes reenforce or diminish the amplitude of MLM1 with marginal changes to the vertical distribution of the MLAD pressure field. This interaction between modes leads to a very long-range cycling of energy first out from and then back into the MLAD\citep{porter93,colosi2020observations}, which causes loss of mixed layer energy for the propagation ranges considered here. This energy cycling is most significant for acoustic frequencies close to the mode 1 cutoff of Eq.~\eqref{eq:f_cutoff}, and shows regional variation over the transect.

\begin{figure}
\includegraphics{../figures/bg_eng_loss.png}
    \caption{Spreading compensated acoustic energy in the RI background MLAD, Eq.~\eqref{eq:int_eng}, for a 400 Hz source at 40 m depth. More loss is observed at transect ranges less than 300 km, lighter (yellow) lines. The mean loss is shown as thick black line, and the mean plus or minus the RMS estimate are dashed (blue) lines.}
    \label{fig:bg_eng}
\end{figure}
A local acoustic energy reference is used to report the results of Eqs.~\ref{eq:int_eng} and \ref{eq:proj_eng} that is calculated from the background sound speed field of each section. Analysis of the background field and MLM energy further estimates the range independent (RI) background field with a range average. The integrated energy of Eq.~\eqref{eq:int_eng} for the RI background field and MLAD-MLAD scenario is shown in Fig.~\ref{fig:bg_eng}. Sources before $x=$300 km are light (yellow) lines and transect ranges beyond 300 km are dark (gray) lines. The most MLAD loss is seen at 400 Hz many source positions less than 300 km, while almost negligable loss is predicted for all positions at 1 kHz. The positions of larger MLAD loss correspond with the shallow SLD observed near the transect start in Fig.~\ref{fig:c_grid}.

\begin{figure}
\includegraphics{../figures/bg_eng_loss_3_panel.png}
        \caption{Background energy loss for three upper ocean acoustic propagation scenarios: (a) MLAD-MLAD, (b) MLAD-TRL, and (c) TRL-MLAD. Dashed lines are the mean RI background energy for the first 300 km of the transect, and solid lines are the mean over the remaining transect. The lighter (orange) lines are 400 Hz, and darker (purple) lines are 1 kHz. Black lines in (b) are a linear and second order polynomial fit to the transect mean RI background energy, used to remove the beat pattern in reference energy estimates.}
    \label{fig:eng_bg_3}
\end{figure}
The acoustic energy reference for all three transmission scenarios are shown in Fig.~\ref{fig:eng_bg_3}, with 400 and 1000 Hz shown as lighter (orange) and darker (purple) lines, respectively. High angle, non-ducted, energy is apparent near the source and convergence zone. Source positions before and after $x=300$ km are displayed as dashed and solid lines, respectively. The difference seen for the MLAD RI background before and after $x=300$ km in Fig.~\ref{fig:bg_eng} is seen in all three propagation scenarios.

The MLAD-MLAD scenario, Fig.~\ref{fig:eng_bg_3}(a), has the highest absolute background energy. There is little difference between the 400 and 1000 Hz MLAD-MLAD energies except for the excess of 2 dB energy loss at 400 Hz and less than 300 km transect ranges. The MLAD-TRL scenario, Fig.~\ref{fig:eng_bg_3}(b), has much lower absolute energy and shows a modal interference beat pattern. The MLAD-TRL reference energy is fit to a first and second order polynomial at 400 and 1000 Hz, respectively, shown as black lines. Frequency dependence of MLAD strength means the MLAD-TRL has the highest energy difference makes 400 Hz between 7 and 10 dB higher than 1 kHz and shows less of an interference pattern. The observed energy is also approximately 2 dB higher for transect ranges before 300 km, caused by increased diffraction from the duct. The width of the convergence zone is wider in this scenario, and ducted energy is considered between 7.5 and 40 km. The TRL-MLAD, Fig.~\ref{fig:eng_bg_3}(c), has a low energy background as well and a curved energy background. The energy is higher before $x=300$ km. Ducted energy is also considered between the ranges of 7.5 and 40 km in this scenario.

\subsection{Blocking features}\label{ssec:blocking}
\begin{figure}
\includegraphics{../figures/integrated_loss.png}
    \caption{Maximum MLAD energy loss over 5 km. Source ranges with an loss greater than 3 dB (gray dotted line) sample blocking features. At 400 Hz, blocking features are typically registered at all source positions that sammple the same feature. At 1 kHz, blocking features mostly have smaller magnitudes and are more sporadic.}
    \label{fig:blocking}
\end{figure}

Blocking features, localized sound speed features transect that cause significant loss, appear throughout the tilt, spice and observed transects. These locations are identified in the MLAD-MLAD scenario, and they are expected to affect each of the three propagation scenarios considered here. Blocking features are most common at 400 Hz, an example was shown in Fig.~\ref{fig:decomp_x}. Blocking events significantly affect statistical characterization of acoustic energy, and the results of analysis both with and without blocking features are reported separately in Sec.~\ref{ssec:energy}.

A blocking feature is defined as a region between source ranges of 7.5 and 47.5 km where background normalized MLAD energy loss exceeds 3 dB in 5 km. The top and bottom panels of Fig.~\ref{fig:blocking} show the maximum MLAD energy loss for each section at 400 Hz and 1 kHz, respectively. At 400 Hz there are eight clearly separated regions of high loss between the total, tilt, and spice fields. These regions are typically 50 km wide and flat topped, indicating the same feature dominates MLAD propagation for all source positions that sample the feature. The MLAD has fewer blocking features at 1 kHz than at 400 Hz. The 1 kHz blocking features are more sensitive to mode phase because they are rarely apparent for consecutive source positions.

The different partitioning of eight blocking features at 400 Hz between the decomposition fields shows regional differences in the dynamics. First, while blocking features in the observed field are distributed across the transect range, the tilt field blocking features occur before $x=450$ km and spice blocking features are prevalent after this range. The last blocking feature position corrisponds to a clear shift in the MLAD sound speed pertubation of the tilt field from mainly positive pertubations to negative pertubations in Fig~\ref{fig:c_fields}(b). Following the discussion of the ($\theta$, $S$) diagram of Fig.~\ref{fig:ts_mean_c}, surface concentrations low sound speed are expected to corrispond with cooler and fresher surface water at longer transcet ranges, which increases the sound speed gradient and the MLAD cutoff frequency. Secondly, while blocking features typically appear in the observed field and either the tilt or spice field, some blocking features appear in only the spice field in the range from 600 and 850 km. These two observations both indicate the tilt field destablizes the MLAD at transect ranges before $X=450$ km, and then stablizes it after this position.

\subsection{Upper ocean acoustic energy statistics}\label{ssec:energy}
\begin{figure}
\includegraphics{../figures/blocking_energy.png}
    \caption{400 Hz energy compensated for range spreading for a source at $x=230$ km in the observed sound speed field. The transmission scenarios are: (a) MLAD-MLAD and MLAD-TRL, and (b) TRL-MLAD. The RI backgrounds for each scenario are shown as black lines, where MLAD-MLAD has significantly more acoustic energy. The MLAD-MLAD energy projected onto MLM1 is shown as a dotted (blue) line.}
    \label{fig:ml_energy}
\end{figure}
Transcet wide energy statistics are calculated for the different combinations of transmission scenario, source frequency and decomposed sound speed field. The mean and RMS energy relative to the backgound are reported just after the source region and just before the first convergence zone. This quanitfies both source region coupling and the effects of ocean dynamic on MLAD propagation.

An example of the reported energy quanities is shown for a 400 Hz source at $x$=230 km in the observed field, Fig.~\ref{fig:ml_energy}. Panel (a) shows the MLAD source, and (b) shows the TRL source. The background field, used as a reference energy, is shown as black lines. The MLAD energy in Fig.~\ref{fig:ml_energy}(a) follows the background until a series of loss events starting at $x=250$ km, the largest at $x=260$ km is characterized as a blocking feature. The net effect of these localized events is approximately 10 dB of loss by the first convergence zone at 47 km.

The MLM1 magnitude describes the vertical distribution of pressure in MLAD-MLAD propagation, and is shown adjusted by $-10 \, \textrm{log}_{10}(z_{SLD})$ to agree with the mean energy scaling of Eq.~\eqref{eq:int_eng}. The mode projection closely follows the total energy until the $x=260$ km blocking feature, where the MLM1 energy falls 3 dB below the total energy. This energy difference is consistant with the MLM2 vertical distribution of energy observed in Fig.~\ref{fig:decomp_x}(d).

Energy in the TRL is strongly downward refracted, and the source TRL of energy in the MLAD-TRL scenario is MLAD energy loss. Examples of the beams of energy into the TRL from the MLAD are seen below the large loss events in Fig.~\ref{fig:decomp_x}(d). These loss events lead to peaks in TRL energy in Fig.~\ref{fig:ml_energy}(a), the largest peak occurs at $x=250$ km and brings the TRL energy to within 2 dB of the MLAD energy. However, these peaks are only a few kilometers wide, and are often followed by significant decreases in energy. The energy between $x=265$ and 275 km is an example where the MLAD energy is stable but reduced from the background level, which results in TRL energy well below the background level.

The TRL-MLAD energy, Fig.~\ref{fig:ml_energy}(b), shows two separate localized effects. First, coupling into the MLAD in the source region leads to MLAD increased energy relative to the background over much of the transmission. Secondly, this energy propagates inside the MLAD and experiences the same blocking feature loss discussed for the MLAD-MLAD scenario. The increase in MLAD energy from the source region is apparent between 240 and 260 km transect range, after which the approximately 4 dB excess leaves the MLAD around 260 km and there is no net energy effect at $x=270$ km.

\begin{figure}
\includegraphics{../figures/eng_shallow.png}
    \caption{MLAD-MLAD energy relative to the RI background. Left column is propagation at 400 Hz, right column is at 1 kHz. The top 4 rows are the complete set of simulated source positions. The bottom 3 rows show source positions without blocking events (W/O Blocking). The integrated energy for each transmissions is shown as light grey lines, half circles at -10 dB indicate a line moves beyond the plotted scale. Linear regression fits of the mean (bold) and RMS ($\pm$) at 7.5 km and 47.5 km are shown right of the line plots.}
    \label{fig:shal_eng}
\end{figure}
The MLAD-MLAD energy statistics are shown in Fig.~\ref{fig:shal_eng}. The 400 and 1000 Hz results are in the left and right columns, respectively. The complete set of source positions is shown in the top section, while source positions without blocking features are shown bellow. The background field shows marginal range dependent effects, the highest RMS is only 0.6 dB at 400 Hz and 47.5 km range, and the discussion of range dependence will focus on the dynamic sound speed fields.

The complete set of source positions for all dynamic fields predicts small mean loss over the source region before 7.5 km at 400 Hz, and small mean gains over this region at 1 kHz. This increased coupling into the mixed layer at higher frequency is also expected in the convergence zones, which also have a wide range of high angle modes, but these are not studied here. The statistics at 47.5 km predict mean loss for all fields and frequencies.

The tilt field has the smallest energy loss at both frequencies. The total loss is more at 400 Hz, although the loss difference between 7.5 and 47.5 km is the same for both frequencies. The RMS at 400 Hz is significantly higher than 1 kHz, and this statistic is skewed by high loss events since the upper RMS bound is not realized by any source position. However, the tilt field over the complete transect is favorable to propagation for some source positions that show marginal loss or small gains. The largest mean energy loss at 47.5 km is predicted for the spice field at both frequencies, with almost 1.5 dB more loss predicted at 400 Hz. The RMS at 47.5 km is higher than the mean at 400 Hz, and while this statistic is loss dominated some fields do show marginal gains of energy. Overall, ocean spice leads to larger mixed layer energy loss than tilt. Finally, the observed field is like the spice but with higher variance.

The mixed layer energy statistics without blocking features are shown in the bottom three rows of Fig.~\ref{fig:shal_eng}. For consistency, source positions were removed from this analysis that had blocking features at either frequency. Outside of an overall adjustment in mean and RMS loss, many of the observations from the complete statistics apply to the statistics without blocking. Two notable exceptions are: (1) the mean loss at 47.5 km range is the same or higher at 1 kHz compared to 400 Hz, and (2) RMS values are similar across field type and frequency. The higher mean loss at 1 kHz is a reversal from the complete field statistics but is consistent with the increased sensitivity of higher frequency acoustics to small-scale sound speed fluctuations. The consistency in RMS loss simplifies the comparison of the mean values and clearly identifies the stabilizing effect of tilt in the observed fields. The differences in statistics between the observation sets support the discussion of blocking features as distinct from the non-localized loss mechanisms expected to be present for all source ranges.

\begin{figure}
\includegraphics{../figures/eng_shallow_proj.png}
    \caption{Mode projected MLAD-MLAD energy. Presentation of the result follows Fig.~\ref{fig:shal_eng}, but with a larger y-axis range of -20 to 5 dB. Energy projection of Eq.~\eqref{eq:proj_eng} uses MLM1 at 400 Hz and MLM2 at 1 kHz.}
    \label{fig:shal_proj}
\end{figure}
The MLAD-MLAD energy is projected onto MLM1 at 400 Hz and MLM2 at 1 kHz in Fig.~\ref{fig:shal_proj}, in the same presentation as Fig.~\ref{fig:shal_eng}. The lower bound of these results is -20 dB to show the increased mean and RMS loss compared with the total mixed layer energy. The statistics at 7.5 and 47.5 km are the result of a linear regression fit to each statistic, which minimizes the effects of brief dips in mode amplitude.

The mean and RMS values for the mode projection show more loss compared with the total energy at all ranges and frequencies except for the background field. The projected and total MLAD energies show the same trend for the complete source set at 400 Hz. The increased mean loss compared with the total mixed layer energy indicates that some ranges couple energy into MLM2, an example of which was shown in Fig.~\ref{fig:blocking}. A larger decrease in mean and RMS energy for the mode projection is seem 1 kHz, where many transmission ranges are expected to have nearly complete coupling out of MLM2. The coupling serves to maintain energy in the MLAD, however, and there is significantly less total energy loss at 1 kHz than at 400 Hz. Finally, while the tilt field shows the highest mean energy at 47.5 km, the RMS value is larger than the spice field. This may be an indication that increased mode coupling in the tilt field decreases loss in the observed field MLAD energy of Fig.~\ref{fig:shal_eng}.

Without blocking features, the 400 Hz projected, and total MLAD energy statistics are within 0.5 dB. This indicates blocking events are necessary to create significant mode coupling at 400 Hz. While there is a substantial decrease in the 1 kHz projected statistics without blocking features, the mean and RMS loss values are still higher than the total MLAD, and mode coupling is expected to be ubiquitous in the mixed layer at 1 kHz. This increase in mode coupling at 1 kHz serves to both reduce the total energy loss at blocking features and increase the diffuse energy loss and mode randomization at ranges without blocking features.

\begin{figure}
\includegraphics{../figures/eng_shallow_tl.png}
        \caption{MLAD-TRL energy relative to a linear fit of the RI background at 400 Hz and a 2nd order polynomial fit at 1 kHz. Presentation of the result follows Fig. \ref{fig:shal_eng}, with y-axis between -20 and 25 dB.}
    \label{fig:eng_tl}
\end{figure}
The MLAD-TRL energy is shown in Fig.~\ref{fig:eng_tl}. The truncated range of statistics (black lines) indicates a linear regression region that avoids the expanded convergence zone, this fit is extrapolated to 47.5 km. The beat pattern in the background field is smoothed out by scattering in the dynamic fields. A similar beat pattern was removed from the reference energy by a polynomial fit, Fig.~\ref{fig:eng_bg_3}(c). This choice in reference energy emphasizes the beat pattern in the background field, leading to relatively high RMS values, but does not introduce an artificial beat pattern into the dynamic field results.

The dynamic field energy at 400 Hz has little to no mean gain at 7.5 km, and losses at 47.5 km. The smallest loss is observed for the spice field, and the most loss occurs for the observed field. Consistent and gradual loss from the MLAD is required to increase the transect mean energy, discussed with Fig.~\ref{fig:ml_energy}, and these decreases in energy could represent either localized loss events or an overall stabilization of the MLAD. The statistics with blocking features removed have similar mean but consistently less RMS at 47.5 km, most apparent in the spice and observed fields. The RMS change at 47.5 km indicates that although localized gains are common, these events do not make a significant impact on transect length averages of TRL energy. This is also consistent with a visual comparison of the RMS bounds with the ensemble realization in Fig.~\ref{fig:eng_tl}, that show more localized exceedances with blocking features than without.

The mean dynamic energy at 1 kHz is increased at both 7.5 and 47.5 km, which is largely compensated between the two frequencies by the difference in mean energy. The relative frequency difference of the dynamic at 47.5 km is between 9 to 13 dB, which is comparable or larger than the approximately 10 dB difference in reference RI background energy. The largest increase is seen for the spice field. The mean increase in 1 kHz TRL energy is consistent with the larger diffuse energy loss at 1 kHz discussed for the MLAD-MLAD scenario, Fig.~\ref{fig:shal_eng}. This diffuse energy loss is expected to slowly decrease energy in the MLAD and hence increase TRL energy across the entire transect. The ensemble realizations also show many events that temporarily exceed 25 dB relative to the background, with absolute levels like those observed at 400 Hz.

\begin{figure}
\includegraphics{../figures/eng_deep.png}
    \caption{TRL-MLAD energy statistics. Presentation of the result follows Fig. \ref{fig:shal_eng}, with y-axis bounds of $\pm$20 dB. The statistics are computed between 7.5 and 40 km and extended to 47.5 km with a linear regression fit.}
    \label{fig:deep_eng}
\end{figure}
Finally, TRL-MLAD energy statistics are shown in Fig.~\ref{fig:deep_eng}. In this scenario, the background field has a large effect on propagation, especially at 1 kHz. The linear energy change for all source ranges suggest long wavelength mode resonances with the slowly varying background field\cite{colosi21} are significant when compared with the low energy background. There is a larger mean and RMS effect on MLAD energy for all dynamic fields, however, with RMS values between 7 and 11 dB at 47.5 km.

As discussed with Fig.~\ref{fig:blocking}, source region coupling is important to increase ducted MLAD energy in the TRL-MLAD scenario. The ducted MLAD energy is then expected to behave like the MLAD-MLAD scenario past the source region. This is simplest to see in cases when source region coupling is uncorrelated with blocking features, \emph{i.e.} the tilt and observed fields that almost no change with and without blocking features at 7.5 km. With the observed field as an example, the energy mean at 47.5 km increases by more than 2 dB and the RMS decreases by more than 1 dB with the removal of blocking features. This change indicates the MLAD energy in the TRL-MLAD scenario is still impacted by the effects of dynamics sound speed fields on MLAD propagation.

The spice field shows a case where the RMS spice field energy statistics decrease by 2 dB at 7.5 km with the removal of blocking features. This correlation then obscures the effect of blocking features on the MLAD propagation of scattered energy, and the removal of blocking features has little effect on the mean energy at 47.5 km. A second exception in TRL-MLAD energy behavior is the anonymously low 1 kHz mean in the tilt field at 7.5 km. The reduced source region scatter into the tilt field MLAD at 1 kHz could be caused by increased sensitivity at higher frequencies to the strong, small scale, sound speed perturbations at the base of the mixed layer, Fig.~\ref{fig:c_fields}(b).

\section{Conclusion}\label{sec:conclusion}
A 970 km transect made of the upper 350 m of the Northeast Pacific Ocean was decomposed to produce sound speed fields with separated isopycnal tilt or spice variation. This decomposition followed Dzieciuch \emph{et al.}\citep{dzieciuch2004} but introduced a linear superposition model within the mixed layer. The separate dynamic decompostion components have different effects on three upper ocean acoustic propagation scenarios, which are related to thier spatial charactoristics of the MLAD and TRL. These charactoristics where sufficently different that many observed sound speed features were partitioned into either the tilt or spice fields, supporting the analysis of the upper ocean sound speed environment as a combination of dynamically separate oceanographic processes.

Acoustic propagation statistics were calculated at 400 and 1000 Hz, which had an average of one or three mixed layer modes, respectively. Statistics were calculated assuming the propagation eneviornment was uniform across the transect. This assumption is consistant with a dynamic description of the mixed layer based on density\citep{cole2010seasonal}, where the observation is roughly consistent with range. However the background field, showed some acoustically significant change around across the total transect range due to the deepening of the SLD. This acoustic variation was most significant at 400 Hz, where the MLAD was significantly more stable in the last two thirds of the transect.  Acoustically significant transect variation was also seen in an analysis of blocking features. All tilt blocking features were observed in the first half of the transect, while most blocking features were observed in the spice field in the second half of the transect. While these regional variations are not expected to significantly effect the statistical results, they indicate that acoustically significant MLAD differences could exist between different locations.

Energy statistics were used to compare the separtate effects of spice and tilt to the observed field over the transect. In the MLAD-MLAD scenario, similar losses occured for the spice and observed fields, while the tilt field had less loss. Between the two frequencies, the 400 Hz transmissions had more loss with blocking features included, while the 1 kHz transmissions had more loss with block features removed. This domonstrates two separate loss mechanisms, diffractive loss from MLAD shoaling at low frequencies and diffues volume scattering loss at higher frequencies\cite{colosi2020observations}. Also, when blocking features were removed, the spice fields produced more loss than the observed fields, which indicated that the tilt field had a net stabilizing effect on the MLAD.

Mode amplitude analysis of the MLAD-MLAD scenario showed higher losses for a single mode than the total field, indicating coupling was significant for both frequencies. The largest coupling was found in the observed field at 400 Hz, and in the tilt field at 1 kHz. Little difference between total and MLM1 projected energy was observed at 400 Hz when blocking features were removed, and mode coupling is expected only at large loss events. While the mode coupling was significantly reduced without blocking features at 1 kHz, the loss values for one mode significantly exceeded that for the total MLAD energy, and mode coupling and randomization is expected to be ubiquitous at this frequency.

In the MLAD-TRL scenario, energy in the TRL is related to local MLAD loss. The interpretation of these statistics is complicated because the TRL energy is not ducted, and a consistent increase in input of energy is required to increase the average energy. The spice field had the highest average in TRL energy at both frequencies, indicating that relativly consistant MLAD energy losses with range. The smaller TRL energy for the tilt field indicated either this field stabilized the MLAD or had sporadic loss events. Blocking features had a marginal effect on the mean energy in the TRL but did change the RMS energy significantly. A more notable change was the increase in TRL energy at 400 and 1000 Hz for all scenarios, consistent with the smaller and more consistent MLAD loss at 1 kHz.

Finally, the TRL-MLAD scenario showed that scatter inside the source region of the dynamic fields increased the ducted energy in the MLAD. This scatter is roughly equal across the dynamic fields at 400 Hz, and significantly reduced in the tilt field at 1 kHz. The reduction of source region scatter into the MLAD at 1 kHz is observed along with small wavelength and large magnitude sound speed variation at the base of the mixed layer from internal waves in the tilt field. The propagation of this energy after entering the MLAD was consistent with the MLAD-MLAD scenario, and the removal of positions with blocking features significantly increased the MLAD energy.

In all three propagation scenarios the observed upper ocean dynamics caused acoustically significant variations in sound speed. The dynamic decomposition of these variables into tilt and spice highlighted the separate effects of many features observed in the upper ocean. Of these two fields, ocean spice was often the dominant source of acoustically relevant sound speed variation in upper ocean propagation scenarios. The effect of tilt was often smaller in magnitude, in part because this field appeared to have a stabilizing effect on MLAD propagation over much of the transect. The differences in these acoustical effects show the utility of separately charactorizing these fields in other propagation enviornments, and raises the possiblity of acoustically infering for these separate dynamics.

\bibliographystyle{jasanum2}
\bibliography{eRichards_master}

\end{document}

