\documentclass[preprint,NumberedRefs]{JASA}
\usepackage{multirow}
\usepackage{booktabs}

\begin{document}
\title[Mixed layer tilt and spice]{Observations of ocean spice and isopycnal tilt sound speed structures in the mixed layer and open ocean, and their impacts on acoustic propagation}
\author{Edward L. Richards}
\email{edwardlrichards@gmail.com}
\affiliation{Ocean Sciences, University of California Santa Cruz, Santa Cruz, California 95064, USA}
\author{John A. Colosi}
\affiliation{Department of Oceanography, Naval Postgraduate School, Monterey, California 93943, USA}

\preprint{E. L. Richards, JASA}	%if you want this message to appear in upper right corner of title page

\date{\today}

\begin{abstract}
Acoustic propagation in the northeastern Pacific Ocean is predicted for observed sound speed fields that approximate two dynamic sources of sound speed variation. Variability observed over an approximately 350-m deep, 1000-km long transect is separated into effects from the tilting of isopycnals and “spice,” density compensated temperature and salinity changes. Measured in early April of 2004, the observations reveal a roughly 100-m deep mixed layer, 100-m transition layer, and 150-m of the upper thermocline. Peak root mean square sound speed variation for the tilt field is 1.5 m/s and occurs in the transition layer, while the spice field has a peak of 0.8 m/s close to the bottom of the mixed layer. Acoustic modeling at 400 and 1000 Hz shows both discrete regions of mixed layer duct blocking and a background of diffuse energy loss in both decomposed fields. The effects of mixed layer variability are observed for acoustic sources both inside and out of the mixed layer duct, and the spice field is demonstrated to cause the most significant degradation of the mixed layer duct, while the tilt field has smaller overall effect and stabilizes the duct in many locations.
\end{abstract}

\maketitle

\section{\label{sec:intro} Introduction}
Both mechanical mixing from wind forcing at the air-sea interface and convection from surface cooling homogenize the upper ocean and form a mixed layer\citep{cole2010seasonal}. A vertically homogonous mixed layer has an adiabatic sound speed gradient of 0.016 m/s that creates an acoustic duct bounded on the top by the reflecting sea surface. The acoustic characteristics of the mixed layer acoustic duct (MLAD) are distinct from a shallow water duct with the same sound speed structure because it interacts with the deep ocean acoustic channel\citep{porter93}, especially at lower frequencies. Howver, the MLAD is expected to be stable at frequencies above its lower limit\citep{Urick1982Prop} and range independent models predict long range acoustic propagation. Range or time dependent measurements of the mixed layer show that it is has highly variable sound speed\citep{cole2010seasonal,rudnick1999compensation,klymak2015}, which can significantly alter or completely block duct propagation\citep{colosi2020observations,colosi21}. This study investigates the relative impact of tilt and spice fields on the MLAD with a decomposition of a mixed layer transect observation following Dzieciuch \emph{et al.}\citep{dzieciuch2004}. Parabolic equation (PE) acoustic simulations\citep{collins93} show both fields block MLAD propagation in a discrete locations, and also have diffuse effects on the MLAD energy over all observations.

While vertical mixing is expected to homogenize the vertical mixed layer properties, range dependence is also expected as the bulk properties of the surface ocean changes\citep{ferrari2000}. The variation of temperature and salinity in the upper ocean tends to be highly correlated, and the relative effects of these changes tend to moderate the overall effect on density. On large scales, temperature varies at twice the rate of this salinity moderation. At shorter scales, however, the temperature and salinity variations often compensate in density completely. Density compensated waters are either relatively warm and salty or cool and fresh, and this variation is termed ocean "spice"\citep{munk1981evolution}. The two scales of density variation are rectified at ocean fronts were density changes abruptly. Similarly, density compensated spice variation tends to be front like, although mixed layer observations do show acoustically significant tilting of these vertical fronts.

Importantly for the acoustic problem, density compensated temperature and salinity variations are reenforcing in sound speed. Spice variations can significantly change acoustic propagation, although spice variability changes significantly with geographical location\citep{colosi12,colosi13,murat2021}. The importance of spice in acoustic environments and the variation of spice over time remain topics of active research.

While the transect observation and accompanying acoustic transmissions considered here were previously analyzed\citep{colosi2020observations}, this paper extends this analysis by decomposing the observed sound speed variation into spice and tilt contributions. Transect measurements of sound speed with high vertical and horizontal resolution allow for this dynamic decomposition and the following acoustic propagation simulations. The dynamic decomposition of a transect measurement of ocean sound speed was first made from a similar SeaSoar measurement by Dzieciuch \emph{et al.}\citep{dzieciuch2004}. This procedure allows for a comparison of the relative effects of these two dynamic sources of sound speed variation and is intended to create a framework for comparison of the relative effects of spice at different locations.

A mixed layer acoustic duct propagation scenario is simulated for the decomposed fields at 400 and 1000 Hz with the PE model RAM\cite{collins93}. The average transect sound speed profile predicts one and three modes trapped in the mixed layer duct at these frequencies, respectively. This comparison is used to study the effects of mode coupling from sound speed range dependence. A transmission length of one convergence zone\citep{jensen2011computational}, approximately 50 km in the subtropical Pacific Ocean, is chosen to remove the consideration of coupling into the duct at convergence zones\citep{colosi2020observations}.

Acoustic simulations were made for source positions every 10 km along the transect. While most results are reported as mixed layer energy statistics, discrete blocking features\citep{colosi2020observations} observed throughout the transect are treated separately. Blocking features are transect ranges where most of the acoustic energy was lost from the duct, and these high losses are usually observed for all source ranges that interact with the same feature. Blocking features appear for both decomposed fields and are most evident at 400 Hz. Differences in the relative loss across the two frequencies show these events represent a distinct loss mechanism compared to diffuse range dependent loss that exists at all ranges.

The paper is organized as follows: Section~\ref{sec:transcet} describes the oceanographic mixed layer observations that motivate this study. Section~\ref{sec:decomposition} discusses the dynamic decomposition of the observed sound speed field into spice and tilt fields. Acoustic propagation through the separated sound speed fields is compared with the observed field in Sec.~\ref{sec:propagation}. The statistics of mixed layer energy loss are then compared across a range of source positions over the transect length. Finally, the study conclusions are summarized in Sec.~\ref{sec:conclusion}.

\section{North Pacific transect}\label{sec:transcet}
%\begin{figure}
%\includegraphics{../figures/transcet.png}
%    \caption{\label{fig:transcet}{Location of the east to west SeaSoar transect in red. Isotherms in $^\circ$C are computed at 50 m depth from the World Ocean Atlas decadal spring temperature average\citep{WOA}.}}
%\end{figure}

This study is based on a 970 km SeaSoar conductivity, salinity, and depth (CTD) transect taken over 4 days in the northeast Pacific ocean,\citep{cole2010seasonal} Fig. \ref{fig:transcet}. The 970 km long transect is parallel to the $16 \ ^\circ$C isotherm at a mixed layer depth of 50 m from the World Ocean Atlas spring decadal average\citep{WOA}. Small differences in slope between the track and the isotherm predict a slight warming of the mixed layer over the transect.

\begin{figure}
\includegraphics{../figures/sound_speed_transcet_sld.png}
\caption{\label{fig:c_grid}{Sound speed observation of the mixed layer. The black line is the sonic layer depth, and the pink line is the mixed layer depth.}}
\end{figure}

The SeaSoar CTD observation\citep{colosi2020observations} extend from the sea surface to 430 m depth, with an average cycle length of 2.6 km. These observations were first interpolated from the vehicle's sawtooth path to a grid with 1 km horizontal and 0.5 m vertical resolution, and WOA climatology is appended to a depth of 4000 m\citep{WOA}. Sound speed, $c$, computed from the CTD data with the Thermodynamic Equation of Seawater 2010\cite{TEOS-10} is shown in Fig. \ref{fig:c_grid}. The mixed layer is apparent as relatively high sound speed over a strong negative sound speed gradient that extends into the deep ocean. Horizontal and vertical variability in sound speed is apparent throughout the mixed layer observations from internal waves, fronts, eddies, and ocean spice\citep{colosi2020observations}.

The observed sound speed variability is decomposed into dynamics that tilt isopycnals and those that cause ocean spice. First, CTD observations are converted into gridded values of potential density and spice, $\gamma$. These gridded quantities are necessary to separate slow varying changes of background ocean properties from the dynamics of interest\cite{dzieciuch2004}, described in Sec.~\ref{sec:decomposition}.

When compared across the entire transect, relatively warm and salty water has higher $\gamma$ than fresher and colder water with the same density. These effects are compensating in density, but reenforcing in sound speed. Spice quantifies the distance in density scaled salinity and temperature space between each observation and a mean value,
\begin{equation}
    \gamma=\textrm{sgn}(T-\bar{T}) \sqrt{\alpha_0^2(T-\bar{T})^2 +\beta_0^2(S-\bar{S})^2},
    \label{eq:gamma}
\end{equation}
where $\bar{T}$ and $\bar{S}$ are the means of temperature and salinity at constant density, and a representative ($\bar{T}$, $\bar{S}$) value defines $\alpha_0=\partial \sigma / \partial T$ and $\beta_0=\partial \sigma / \partial S$.

Ocean spice is illustrated in two profiles measured at 249 and 252 km transect range, shown in Fig. \ref{fig:profiles}. The potential density referenced to 0 dbar, $\sigma$, is essentially constant for both profiles to 90 m depth. The slight increase of density with depth is required to ensure dynamic stability, and in rare cases the observed profiles were adjusted to create dynamically stable stratification\citep{barker2017stabilizing}. The mixed layer depth (MLD) is defined as a 0.05 kg/m$^3$ increase from the shallowest observation following Cole \emph{et al.}\cite{cole2010seasonal}. The MLD is shown as a pink line in Fig.~\ref{fig:c_grid}, shows significant small scale variation but a linear regression fit has essentially no slope over the entire transcet. The MLD is also esstially identical for the profiles at 249 and 252 km shown in Fig. \ref{fig:profiles}(a), at 96.5 and 100 m respectivly.

The MLD completely determines the MLAD in the case of a mixed layer with vertically uniform temperature and salinity, when sound speed increases with the adiabatic gradient of 0.016 m$^{-1}$ to the base of the mixed layer before decreases rapidly with the decreasing temperature of the thermocline. This simple mixed layer model is a good description of the sound speed profile measured at 252 km, Fig.~\ref{fig:profiles}(c), which has a gradient of 0.015 m$^{-1}$ up to 90 m depth. This model does not describe the sound speed profile measured at 249 km, however, which has a gradient of 0.024 m$^{-1}$ up to 45 m depth and is significantly non-linear past this point to the mixed layer depth. For the profiles at 249 km in Fig. \ref{fig:profiles}, the approximately constant potential density shows that high variability in spice causes the sound speed variation above the MLD in the 249 km profile.

\begin{figure}
\includegraphics{../figures/sld_profile_grad.png}
    \caption{\label{fig:profiles}{(a) Potential density, $\sigma$ (b) spice, $\gamma$ and (c) sound speed, $c$, measured at 249 km (blue curves) and 252 km (orange curves). The mixed layer depth is shown as horizontal pink lines in (a), and the sonic layer depth is shown as horizontal black lines in (c).}}
\end{figure}

An alternative to the MLD that can better describe more observed sound speed profiles is the sonic layer depth (SLD), defined as the depth of maximum sound speed above the thermocline. The SLD is shown as a black line in Figs.~\ref{fig:c_grid} and \ref{fig:profiles}(c). By itself, the SLD also requires the assumption of an adiabatic gradient to describe acoustic propagation, but it does allow for spice and tilt dynamics in the upper ocean to change the height of the surface acoustic duct. For example, the difference between the MLD and SLD for the profile at 252 km is caused by a sharp decrease in spice around 80 m depth which does not otherwise affect the adiabatic gradient in sound speed through the mixed layer. Acoustic propagation is related to the SLD in the same way as the MLD. The SLD varies between 132 m and 32.5 m over the transect, shown as black lines in Figs.~\ref{fig:c_grid} and \ref{fig:profiles} (c). The SLD increases with range over the transect, a linear regression fit has a 78.5 m intercept and a slope of 2 m / 100 km. While the large-scale trend of increasing SLD indicates the MLAD is strengthening, significant variability is also observed in SLD depth over the transect. The rapid variation in SLD between the profiles at 249 and 252 km Fig. \ref{fig:profiles} is one example that indicates fine scale range dependence of the mixed layer is acoustically significant.

While the SLD alone can describe a mixed layer with an adiabatic gradient, the mode cutoff frequency\citep{Urick1982Prop} is a more complete metric that describes general sound speed gradients,
\begin{equation}
    f_{min}(n) = \frac{3(4n-1)}{16} \sqrt{\frac{c_0^3}{2h^3} \left( \frac{\partial c}{\partial z} \right) ^ {-1}},
    \label{eq:f_cutoff}
\end{equation}
where $f_{min}$ has units of Hz. The SLD is $h$, and a refence sound speed of $c_0 = 1500$ m/s is assumed. The two profiles at 249 and 252 km have SLDs at 45 and 89 m, respectively, while the sound speed gradients are 0.24 and 0.15 s$^{-1}$. Equation \eqref{eq:f_cutoff} predicts cutoff frequencies of 485 and 222 Hz, indicating the relative importance of SLD compared to the sound speed gradient. Several observations on mode cutoff frequency over the complete transect from Colosi and Rudnick\cite{colosi2020observations} are summarized here. Sound speed gradients vary from 0 to above 0.1 and are on average higher than the adiabatic gradient. The first four cutoff frequencies for the transect average sound speed profile are 210, 480, 760 and 1040 Hz. These observations lead to the acoustic simulation frequencies of 400 and 1000 Hz, which have on average 1 and 3 modes in the MLAD, although individual profiles may have cut offs above one or both frequencies.

While the SLD is expected to be more acoustically relevant than the MLD, differences between these two depths seen in Fig.~\ref{fig:c_grid} indicate differences in the ocean dynamics that create both non-adiabatic gradients and non-linear sound speed profiles in the upper ocean. For example, the SLD and MLD agree well with each other at 252 km, while the SLD is significantly shallower than the MLD at 249 km. This difference indicates that density compensated sound speed differences are the main cause of the non-adiabatic sound speed mixed layer gradient at 249 km. In contrast, many positions along the transect have shallower MLDs than the SLD, indicating significant density stratification in the MLD. Unlike the thermocline, the observed density stratification can increase the sound speed with depth, \emph{e.g.} the MLD peaks at 660 and 800 km. While either density compensated or stratified sound speed structure may dominate select profiles, both effects are expected to also occur simultaneously in observed profiles.

\section{\label{sec:decomposition}Spice and tilt decomposition}
A dynamic decomposition is used to separate the observed sound speed variations into contributions from isopycnal tilt and spice. This decomposition follows the method proposed for a similar SeaSoar transect by Dzieciuch \emph{et al.},\citep{dzieciuch2004} however, spice in the mixed layer is treated differently here. The decomposition is first described for the case of the stratified region below the mixed layer. A linear superposition model for spice is then introduced for the marginally stratified mixed layer, and the results are compared. Finally, the root mean square (RMS) sound speed variations of the decomposed components are compared.

\begin{figure}
\includegraphics{../figures/sig_tau_interp.png}
    \caption{\label{fig:cntrs}{Contours of: (a) potential density, $\sigma$ (b) spice, $\gamma$. The low pass estimate of each stable isopycnal position is shown in (a) as red curve. The isopycnals shoal and enter the mixed layer with decreasing range, where the isopycnal depth becomes highly variable with range. The stable isopycnal estimate ends at the first location the isopycnal shows significant decorrelation with lower isopycnals, \emph{e.g.} 270 km for $\sigma=25.25$ kg/m$^3$. The low-pass estimate of spice for the $\sigma=25.31$ kg/m$^3$ is shown in (b) as a red curve, indicative of processing for all isopycnals.}}
\end{figure}

\subsection{Stratified ocean decomposition}
The dynamic decomposition begins by defining a gridded quantity, $\sigma$ is used for demonstration, as an interpolation of values between isopycnals. The isopycnal $z$ position is defined $\sigma(x, z) = z^{-1}(x, \sigma)$ and discretized as $z_i(x; \sigma_i)$. A two-dimensional linear interpolation is defined for $\sigma$,
\begin{equation}
    \sigma(x,y)\approx\mathcal{L}(x, y; \sigma_i, z_i).
    \label{eq:lin_intr}
\end{equation}
The isopycnal positions $z_i(x, \sigma)$ are modeled as the superposition of fine scale dynamics on a stable background position $\bar{z}(x, \sigma)$. The background density field, $\bar{\sigma}$, is defined by substituting $\bar{z}(x, \sigma)$ for $z(x, \sigma)$ in Eq. \eqref{eq:lin_intr}. The estimate of $\bar{\sigma}(x,y)$ is a vertically stretched field without fine scale tilt dynamics.

The background isopycnal position, $\bar{z}(x, \sigma)$, is estimated with a spatial low-pass filter. This study uses a cutoff of 50 km, the approximate length of one acoustic convergence zone\cite{jensen2011computational}. This estimate of $\bar{z}(x, \sigma)$ is shown as red lines in Fig. \ref{fig:cntrs}(a). The position of isopycnals inside the mixed layer are unstable and can change across the entire MLD within the horizontal resolution of observation, consistent with density changes in the mixed layer occurring at fronts. However, a number of isopycnals are observed to have multiple fronts, like the sharp increase and decrease in the depth of $\sigma=25.21$ kg/m$^3$ at the left of Fig.~\ref{fig:cntrs}(a). These isopycnal positions are expected to be poorly described as a perturbation from the contours low-passed position. Therefore, instead of tracking the unstable positions of isopycnals in the mixed layer, the dynamic decomposition only estimates the mean position for isopycnals beneath the mixed layer.

The observed unstable isopycnals move below the mixed layer and become stable with increasing range, consistent with density decreasing with the increasing temperature predicted in Fig. \ref{fig:transcet}. A conservative estimate is made for the start position of $\bar{z}(x, \sigma)$ where the position of the isopycnal becomes correlated with the denser isopycnals. For example, Fig. \ref{fig:cntrs} shows $\sigma=25.25$ (kg/m$^3$) stays below the mixed layer past 270 km. These positions where the dynamic decomposition begins tracking isopycnal positions create horizontal discontinuities in the stable background field, consistent with a front model for density change in the mixed layer.

The stable isopycnal position estimates can similarly define the spice field without fine scale isopycnal tilt,
\begin{equation}
    \gamma(x, z)\approx\mathcal{L}(x, z; \gamma_i(x), z_i),
    \label{eq:lin_intr_gamma}
\end{equation}
where $\gamma_i(x)$ is measured along the $\sigma_i$ isopycnal. Since the sampling of $\gamma$ is determined by the isopycnal positions of $\sigma$, the accuracy of Eq. \eqref{eq:lin_intr_gamma} depends on a correlation between changes in these two variables. This sampling is expected to be most accurate in the highly stratified regions below the mixed layer, and the least accurate in the mixed layer itself.

The value of $\gamma_i(x)$ in Eq. \eqref{eq:lin_intr_gamma} can also be modelled as the superposition of fine scale dynamic structure on a stable background, $\bar{\gamma}_i(x)$. Like the estimate of background isopycnal position, a low-pass filter with 50 km cutoff length is used to estimate $\bar{\gamma}_i(x)$ along each observed isopycnal. Combinations of the low-pass and observed values of $\sigma_i$ and $\gamma_i(x)$ can be substituted into Eqs. \eqref{eq:lin_intr} and \eqref{eq:lin_intr_gamma} to produce four $\gamma(x,z)$ fields. The stable background field is computed with ($\bar{\gamma}_i$, $\bar{z}_i$), the tilt field with ($\bar{\gamma_i}$, $z_i$), the spice field with ($\gamma_i$, $\bar{z}_i$), and the observed total field is conceptually produced from ($\gamma_i$, $z_i$). The sound speed is then computed from ($\sigma$, $\gamma$) through an inverse of Eq. \eqref{eq:gamma} computed by iterative linearization of ($\alpha$, $\beta$).

%The values of $\alpha_0$ and $\beta_0$ from Eq. \eqref{eq:gamma} define the angle $\theta_0$, a linearized estimate of the angle in (T, S) space of maximum $d\gamma$ and zero $d\sigma$. The value of ($\sigma$, $\gamma$) is first estimated from $\theta_0$. Then, the value of $\sigma$ is refined with the local value of $\theta$, and finally $\gamma$ is corrected with $\theta_0$. This process can be repeated to a desired precision of (T, S), which are combined with the isopycnal position to compute a value of sound speed.

\subsection{Mixed layer decomposition}\label{ssec:ml_decomp}
\begin{figure}
\includegraphics{../figures/sound_speed_comp.png}
    \caption{\label{fig:c_diff}{Difference in spice field sound speed between stratified decomposition and linear superposition methods. The two decompositions are equal at the stable position of the lightest tracked isopycnal. The discontinuity in stable isopycnal position at 270 km is the last tracked position of isopycnal $\sigma=25.27$ kg/m $^3$. Significant vertical variation of sound speed is apparent in the mixed layer above the last tracked position of the isopycnal.}}
\end{figure}

Estimation of the stable isopycnal position and spice with a low-pass filter is most effective outside of the mixed layer where the ocean is well stratified in regions of spice variation. Two challenges in the dynamic decomposition of Dzieciuch \emph{et al.}\citep{dzieciuch2004} arise in the mixed layer: (1) significant variations of $\gamma$ occur in positions with small $\sigma$ gradients, and (2) isopycnal locations vary rapidly and may not have a stable position. A linear superposition model for $\gamma$ is proposed here for the mixed layer to avoid the requirement of sampling along isopycnals.

The linear superposition model for $\gamma$ attributes all observed variability not explained by the tilt field to the spice field, without any vertical stretching to account for the mean isopycnal levels. The linear superposition model is written
\begin{equation}
    \gamma_{observed} = \gamma_{bg} + \Delta \gamma_{tilt} + \Delta \gamma_{spice},
    \label{eq:lin_sup}
\end{equation}
where $\gamma_{bg}$ is the background spice field. The spice field is defined as $\gamma_{bg} + \Delta \gamma_{spice}$, and the tilt field is $\gamma_{bg} + \Delta \gamma_{tilt}$. The value of $\Delta \gamma_{spice}$ is estimated by subtracting $\gamma_{tilt}$ from $\gamma_{observed}$. This approach uses the background and tilt fields, which do not require vertical stretching of mixed layer isopycnals to estimate the spice field from the observed field.

The field $\gamma_{spice}$ of the stratified decomposition of Eq.~\eqref{eq:lin_intr_gamma} and the linear superposition model are compared in Fig.~\ref{fig:c_diff}. The only difference between these methods is above the last stable isopycnal position. While the difference field often shows uniform vertical structure, there are also locations with significant vertical variation. Sampling this vertical structure with the stratified decomposition would require stable isopycnal positions well into the mixed layer, and some positions show sampling would be required in the top 25 m of the mixed layer. The linear superposition model avoids the need to introduce non-physical stable isopycnal positions to sample the observed spice variation in the mixed layer.

\begin{figure}
\includegraphics{../figures/diff_fields.png}
        \caption{\label{fig:c_fields}{Total and decomposed fiels. Background field is shown in (a). The dynamic fields are shown with the background field subtracted: panel (b) is tilt, (c) is spice and (d) is the total field. The SLD is shown in each panel as a black line.}}
\end{figure}

The background and dynamic fields computed with the dynamic decomposition are shown in Fig. \ref{fig:c_fields}. The background field, top panel, shows a relativly high sound speed mixed layer that overlies slower sound speeds in the thermocline. The sound speed in the mixed layer increases with increasing transcet range. The SLD of the background field varies smoothly in range except at discontinous positions where an isopycnal enters the mixed layer, Sec. \ref{ssec:ml_decomp}. The change in SLD, even at the position of discontinuities, is relativly small compared with that of the dynamic fields shown in the bottom three panels, which predicts the background field will produce marginal range depdent effects.

The difference in sound speed from the background for the tilt, spice, and observed field are shown in panel (b), (c), and (d), respectivly. Many of the sound speed features in the observed field are largly separated into either the spice or tilt fields. The highest sound speed variations are seen in the tilt field transition layer below the mixed layer, which are caused by internal waves changing the depth of the thermocline\cite{colosi21}. The tilt field also has some features in the top 50 m related to the restratification of the mixed layer, which begins at the surface. The surface features of the tilt field are largly positive for the first half of the transect, and are mostly negative over the second half.

The spice field has smalller magnitudes than the tilt field overall, and many of the spice features extend over the entire depth of the mixed layer. Spice contributions to sound speed also appear in the transition layer, where they often appear tilted with depth. The effect of spice on the SLD is more intermittant than the tilt field. A clear sparation exist between the relativly flat SLDs in the first half of the transect, and very jagged and often shallow SLDs in the second half of the transcet. Finnally, the observed field contains a clear superposition of the features in the spice and tilt fields, which supports the dynamic decomposition in the discussion of observed sound speed fields.

\subsection{Decomposed sound speed statistics}
\begin{figure}
\includegraphics{../figures/rms_profile.png}
    \caption{\label{fig:c_rms}{The transect mean sound speed profile and the RMS measured, tilt and spice field deviations from the background.}}
\end{figure}

The transect mean sound speed profile and root mean square (RMS) statistics of each dynamic field are shown in Fig.~\ref{fig:c_rms}. The mean sound speed profile has a linear increase of sound speed with depth to the 80 m SLD and is fit to a slope of 0.024 s$^{-1}$. There is a sharp decrease in sound speed below the SLD at the thermocline. The RMS statistics of the observed, tilt and spice fields all have maximum values below the SLD. The shape of the spice field RMS is like the shape of the depth averaged spice variance reported by Ferrari and Rudnick\citep{ferrari2000}. While spice variance was shown to monotonically decrease with isopycnal density, a depth average spice maximum was found at the base of the mixed layer caused by variability both along isopycnals and from crossings of tilted isopycnals. Despite having significantly reduced small scale isopycnal tilt, the dynamically separated spice field still maintained a peak RMS value at the base of the mixed layer.

The peak RMS values of the tilt and observed fields occur around 110 m depth. The tilt RMS then increases to a smaller maximum at the surface. Internal waves followed the Garret-Munk spectrum up to the SLD. Eddies and fronts were the largest contribution to tilt in the near the surface.

The RMS values of tilt are significantly higher than spice except around the SLD, where the maxima of the spice field is close to the tilt field minima. The spice significantly modifies the observed RMS in this region, both increasing the minimum value and moving the position up about 20 m.

\begin{figure}
\includegraphics{../figures/diff_spectra.png}
        \caption{\label{fig:spectra}{Power spectral density of the total, tilt, spice and background fields in the (a) mixed layer, (b) transition layer and (c) thermocline.}}
\end{figure}

Power spectral density of the sound speed difference of each decomposed field are shown in Fig.~\ref{fig:spectra} in the ML, TL, and upper thermocline. Following Colosi and Rudnick\cite{colosi2020observations}, each spectrum is an average of 4 depths: 20, 40, 60 and 80 m for the ML; 100, 120, 140, 160 m for the TL; and 180, 200, 220, 240 m for the upper thermocline. The mean background profile is subtracted from the background field, while the background field is subtracted from each of the dynamic fields. This subtraction gives a sharp fall off of the background field for wavelengths above 50 km (0.02 cpkm), and the dynamic fields have a minima at low wave numerbers and a peak at 50 km. The tilt and total fields in the mixed layer are an exception to this general shape, which have significant energy above 50 km wavelength. The long wave length energy in these fields are due to near surface stratification strucures that are not included in the mean because they do not have stable isopycnal positions. Examples of this in tilt and tital fields are the low sound speed features above 45 m depth at 800 and 875 km transect range, and the strech of positive ML sound speed anomaly between 0 and 300 km. The spice field is almost a factor of 10 more energy than the tilt field at for wavelengths shorter than 50 km. This energy separation in wavenumber between the dynamic fields indicates the mixed layer tilt energy creates long range changes to the propagation enviornment, while the spice field creates more intermittant sound speed changes.

The spectra in the transition layer and upper thermocline have similar shapes. The peak of the tilt and total fields at 50 km wavelength is approximatly a factor of 10 above the spice field, and tilt dynamics dominate the observed RMS sound speed variation. This variation is concetrated below the MLD and SLD, which has the largest sound speed variation in features that span approximatly 20 m depth. In the upper thermocline the contribution of tilt and spice are approximatly the same and contribute equally to the total sound speed variation. The spice spectrum is less peaked and has slightly more energy between 50 and 10 km wavelengths than the tilt. The equipartition over much of the spectrum is a result of the relatively shallow spectrum sampling depths. The RMS shown in Fig.~\ref{fig:c_rms} indicates tilt variation begins to dominate below 250 m depth and spice variation approches zero.

\section{\label{sec:propagation}Upper ocean acoustic propagation}
The separate effects of tilt and spice are compared with the observed sound speed field and the smooth background over 60 km propagation sections. Each propagation section is separated by 10 km transect range, a distance chosen as a compromise that samples the observed range variation with reasonable independence. Two source frequencies, 400 and 1000 Hz, are used for comparison of propagation with an average of one and three mixed layer modes, respectively.

The impact of separate upper ocean sound speed dynamics is demonstrated with three upper ocean propagation scenarios. The first considers acoustic energy in the MLAD for a source in the MLAD at 40 m depth, which directly quantifies MLAD propagation. The second considers acoustic energy in the MLAD for a source below the MLAD at 200 m depth, which both quanitifies acoustic coupling into the MLAD from the source region and the subsequent MLAD propagation. Finally, the non-ducted energy in the transition layer below the mixed layer is discussed for a MLAD source at 40 m depth. The observed energy in the transition layer is driven by loss of energy from the MLAD, which is described with a simple model where the energy leaves the MLAD mixed layer at a horizontal grazing angle. The MLAD is the dominant factor that determines the acoustic energy in all three scenarios, and the tilt, spice and observed dynamics all have different relative effects on the acoustic energy.

A representative section is shown of MLAD propagation in Fig. \ref{fig:decomp_x} for a 400 Hz acoustic source at 40 m depth. Acoustic propagation is modeled with the parabolic equation (PE) code RAM\citep{collins93}, and normal modes are used to analyze the vertical structure of mixed layer acoustic energy, Sec. \ref{ssec:bg}. The PE results, right column  of Fig. \ref{fig:decomp_x}, show some fields have significant loss and changes in vertical distribution of mixed layer acoustic energy.

\begin{figure}
\includegraphics{../figures/decomp_xmission.png}
    \caption{\label{fig:decomp_x}{Left panels is sound speed field; right panel is acoustic pressure. Rows are the: (a) background (b) tilt, (c) spice, and (d) observed fields. The region up to 7.5 km from the source has significant down going energy for all fields, and up going energy of the first convergence zone is apparent starting 47.5 km from the source. Significant mixed layer loss between the source and first convergence zone appears as down going energy below the approximately 120 m deep mixed layer. The mixed layer loss is strongest around 250 and 260 km for the spice and total fields, and smaller loss is also observed in the tilt and total fields around 240 km.}}
\end{figure}

The background, tilt, spice and observed sound speed fields are shown in the left column of Fig. \ref{fig:decomp_x}. The background acoustic field shows canonical mixed layer acoustic propagation. The mixed layer duct is obscured by high angle propagation up to about 7.5 km from the source, and in the first convergence zone starting around 47.5 km. A highly absorbent layer introduced at the bottom of the PE domain suppresses bottom interactions that would otherwise contribute non-ducted energy to the mixed layer at source ranges between 7.5 and 47.5 km. With the higher angle arrivals removed, the mixed layer between the source region and the first convergence zone has no external sources of acoustic energy. The mixed layer acoustic pressure in this region has one maximum in depth with magnitude and a slow decrease with range.

The tilt and spice fields are shown in the left column of Fig. \ref{fig:decomp_x}, panels (b) and (c), respectively. The most acoustically significant tilt feature variation is a SLD shoaling between transect range 230 to 240 km that leads to mixed layer energy loss in the tilt field. A surface concentration of lower sound speed also appears in the tilt and observed fields between 270 and 280 km transect range. This tilt feature increases the mean sound speed gradient of the mixed layer, both strengthening the mixed layer duct and moving acoustic energy to shallower depths.

The spice sound speed field, left panel (c), shows both vertical fronts and features with significant depth variation. The spice field has a high sound speed feature between 250 and 260 km transect range, two profiles from which are shown in Fig. \ref{fig:profiles}. This feature causes significant loss of acoustic energy from the mixed layer duct that is concentrated at the feature edges. There is also a marked change to the vertical pressure distribution that indicates coupling between the mixed layer normal modes. The loss of mixed layer energy is characteristic of a blocking feature that significantly reduces the viability of the MLAD, and also briefly increases energy in the transition layer below the MLAD.

Similar blocking features are also observed at other ranges and can appear in one or multiple dynamic fields. A metric of blocking feature is defined through mixed layer energy in Sec. \ref{ssec:blocking}. Statistics of the mixed layer energy are significantly influenced by these sporadic high loss events, and so are reported both with and without these blocking features in Sec. \ref{ssec:energy}.

\subsection{Background upper ocean acoustic energy}\label{ssec:bg}
Two quantifications of mixed layer energy are considered, vertical integration of the PE pressure result, and a limited depth projection of a mixed layer mode (MLM) onto the PE pressure result. The vertical integration of the PE pressure contains no information about the vertical distribution of energy, while the modal quantification describes the energy in the mixed layer with the same vertical distribution as the MLM. The limited depth mode projection is proposed as an alternative to mode amplitudes from coupled mode calculations that approximates mixed layer orthogonality even when multiple modes have the similar shape in the mixed layer.

The vertical integration of acoustic energy is computed from the PE acoustic pressure result, $p(x, z)$, as
\begin{equation}
        \textrm{E} = 20 \, \textrm{log}_{10} \left( \frac{1}{z_1 - z_0} \int^{z_1}_{z_0} \left| p(x, z) \right| \,  dz \right).
    \label{eq:int_eng}
\end{equation}
The ML energy, $\textrm{E}_{ML}$, is computed by integration between 0 and 125 m depth. The transition layer energy, $\textrm{E}_{TL}$, is integrated between 125 and 250 m depth. These fixed depth integrations simplify the comparisons between ML and transition layer energies across the transect length.

\begin{figure}
\includegraphics{../figures/bg_eng_loss.png}
    \caption{Spreading compensated acoustic energy in the RI background mixed layer, Eq.~\eqref{eq:int_eng}, for a 400 Hz source at 40 m depth. Significantly more loss is observed at transect ranges less than 300 km, shown as light (yellow) lines. The mean loss is shown as thick black line, and the mean plus or minus the RMS estimate are dashed (blue) lines.}
    \label{fig:bg_eng}
\end{figure}
The integrated energy of Eq.~\eqref{eq:int_eng} is shown in Fig.~\ref{fig:bg_eng} for the RI background field for a 400 Hz source at 40 m depth. This 60 km range averaged background field is used as the reference mixed layer energy loss for each source position. Sources at transect ranges before 300 km are plotted as light (yellow) lines and transect ranges beyond 300 km are dark (gray) lines. For 400 Hz, significantly more loss is predicted for many source positions at transect ranges less than 300 km. The largest energy losses before 300 km are 7 dB down from the 7.5 to 47.5 km, compared with a maximum of 2 dB for source positions beyond 300 km transect range. The larger loses in the background mixed layer correspond with the shallow SLD observed at smaller ranges in the transect of Fig.~\ref{fig:c_grid}. The RI background mixed layer loss is far less significant at 1 kHz, where the largest losses are 0.5 dB at 47.5 km.

The limited depth mode projection computes the inner product of a MLM, $\psi(z)$, and the PE pressure field up to the $n$-th zero crossing, $z_n$,
\begin{equation}
    \textrm{E}_{\textrm{\vspace{0.001} MLM}} = 20 \, \left( \textrm{log}_{10} \frac{1}{z_n} \left| \int^{z_n}_0 \,  p(x, z) \ \psi(z) \,  dz \ \right| - \textrm{log}_{10} \frac{1}{z_n} \int^{z_n}_0 \, \left| \psi(z) \right| \,  dz \right).
    \label{eq:proj_eng}
\end{equation}
The normalization term accounts for the partial mode energy in the limited depth integration of Eq.~\eqref{eq:proj_eng}, which is equal to the water density for modes with no energy outside the mixed layer\citep{jensen2011computational}.
\begin{figure}
\includegraphics{../figures/mode_shapes.png}
    \caption{\label{fig:bg_modes}{(a) Mean sound speed profile of the background field between 230 and 280 km, (b) Shapes of mixed layer mode 1 (mode \#237) and surrounding modes at 400 Hz. The sound speed profile has a SLD of 99.5 m and is fit to a slope of 0.017 1/s. Modes 237 and 238 have similar shapes in the mixed layer, and all modes have a significant tail that extends to the compensation depth of the mixed layer around 3200 m depth.}}
\end{figure}

An example of three normal modes for the range independent (RI) background field, averaged from 230 to 290 km, are shown in Fig.~\ref{fig:bg_modes}. These modes are centered around mixed layer mode 1 (MLM1), which has no zero crossings in the mixed layer. MLM1 has total of 236 zero crossings over the vertical span to the compensation depth of the mixed layer, about 3200 m depth, and an absolute mode number of 237. To facilitate finding MLM1 in the complete mode set, it is defined here as the mode with the most mixed layer energy within a mode number of the first peak in mode loop length\citep{jensen2011computational},
\begin{equation}
    l_{m} = \frac{2 \pi}{k_{m+1} - k_m}.
    \label{eq:loop_length}
\end{equation}
The loop length values of the modes around MLM1 are approximately 50 km and define the convergence zone length. The mixed layer duct modes create high peaks in loop length values, which can also identify mixed layer mode 2 (MLM2) and higher.

The mode shape of MLM1 is not orthogonal to neighboring modes in Fig.~\ref{fig:bg_modes} over the vertical span of the mixed layer. Instead, the coherent sum of these similarly shaped modes reenforce or diminish the amplitude of MLM1 with marginal changes to the vertical distribution of the mixed layer pressure field. The interaction between modes leads to a very long-range cycling of energy first out from and then back into the mixed layer \citep{porter93,colosi2020observations}, which cause loss of mixed layer energy for the propagation ranges considered here. This energy cycling is most significant for acoustic frequencies close to the mode 1 cutoff of Eq.~\eqref{eq:f_cutoff}, and is resposible for the elevated loss in the RI background before 300 km shown in Fig.~\ref{fig:bg_eng}.

\begin{figure}
\includegraphics{../figures/bg_eng_loss_3_panel.png}
        \caption{Background energy loss for three upper ocean acoustic propagation scenarios: (a) MLAD energy for a 40 m deep source, (b) MLAD energy for a 200 m deep source, and (c) TL energy for a 40 m deep source. Dashed lines are the mean RI background energy for the first 300 km of the transcet, and solid lines are the mean over the remaining transect. The lighter orange lines are for a 400 Hz source, while darker purple lines are a 1 kHz source.}
    \label{fig:eng_bg_3}
\end{figure}

Each of the three propagation scenarios results are reported as the mean and RMS energy relative to the RI background, which is shown in Fig.~\ref{fig:eng_bg_3} for 400 and 1000 Hz sources as dark (purple) and light (orange) lines, respectivly. High angle and unducted energy is apparent near the source and convergence zone, and energy statistics are computed away from these regions. The difference in the MLAD RI background between transect ranges before and after 300 km is consistant across all three propagation scenarios, displayed as dashed and solid lines before and after this range. Although clear differences in the MLAD exist with transect range, all ranges are used to compute energy statistics after removal of the RI background.

The shallow source MLAD scenario, Fig.~\ref{fig:eng_bg_3} panel (a), has the highest RI background energy. There is little difference between the 400 and 1000 Hz MLAD energies except for an excess of 2 dB energy loss at 400 Hz and less than 300 km transcet ranges. The deep source scenario shown in panel (b) has significantly lower RI background energy. The lower energy background make the source and convergence zones wider, creating a curved energy background. The deep source energy in the MLAD is 2-3 dB higher at 400 Hz than the 1 kHz. The energy is also 2-3 dB higher for both frequencies at less than 300 km transect range, indicating increased loss in a shallow source MLAD scenario leads to MLAD energy gain in the deep source scenario. Finally, the transistion layer energy in panel (c) has a modal interference beat pattern. This scenario has the highest energy difference between source frequncies, 400 Hz has between 7 and 10 dB higher than 1 kHz and shown significantly less of an interference pattern.

\subsection{Blocking features}\label{ssec:blocking}
\begin{figure}
\includegraphics{../figures/integrated_loss.png}
    \caption{Maximum mixed layer energy loss over 5 km. Source ranges with a loss greater than 3 dB (gray dotted line) over 5 km are considered to contain a blocking feature. At 400 Hz, blocking features are typically detected at all source positions that contain the same feature. At 1 kHz, blocking features have mostly smaller magnitudes and are more sporadic.}
    \label{fig:blocking}
\end{figure}

Acoustic simulations with a 400 Hz source at 40 m show localized blocking features in the observed transect that cause significant loss in the mixed layer duct, \emph{e.g.} transect ranges 250 and 260 km in Fig~\ref{fig:decomp_x} (c) and (d). The large losses from sporadic blocking events significantly effect statistical characterization of acoustic energy, and the results of analysis both with and without blocking features are reported in Sec~\ref{ssec:energy}. To identify blocking features, the MLAD energy is first computed with Eq.~\eqref{eq:int_eng} for a 40 m source and normalized to the RI background loss. A blocking feature then defined as a region between source ranges of 7.5 and 47.5 km over which the MLAD energy falls at least 3 dB in 5 km. While a transmission range may have more than one blocking region, the maximum loss is used to characterize the most dominant blocking feature.

The maximum excess MLAD loss over 5 km is shown in Fig.~\ref{fig:blocking}, with 400 Hz and 1 kHz in the top and bottom panels, respectivly. At 400 Hz there are a few regions of high loss for the total, tilt, and spice fields. These regions are typically flat topped with approximately 50 km width, indicating the same feature dominates mixed layer propagation for all source positions that include the feature. The MLAD has fewer blocking features at 1 kHz than at 400 Hz, and at 1 kHz these features are more sensitive to mode phase because they are rarely apparent for consecutive source positions. The comparison of the two frequencies show that many sound speed features only block mixed layer propagation at low frequencies, however these positions are treated separately for statistical characterization at 1 kHz because they are still expected to cause significant mode coupling.

The dynamic decomposition shows different partitioning between the dynamic fields of the eight blocking features at 400 Hz. Blocking features typically appear in the observed field and either the tilt or spice field. However, some blocking features appear in only the spice field in the range from 600 and 850 km, indicating the tilt field stabilizes the observed field at these positions. This stabilization effect is observed at positions with concentrations of low sound speed at the surface in the tilt field seen in Fig~\ref{fig:c_fields}, an example of this was also observed in Fig.~\ref{fig:decomp_x} around 270 km.

While the blocking features in the observed field are distributed across the transect range, the tilt field blocking features occur before 450 km and spice blocking features dominate after this range. This blocking feature distribution indicates that the effects of both tilt and spice fields are regionally variable over the 1000 km range of the transect.

While blocking features reduce the energy in the MLAD, this energy temporarily increases the acoustic energy in the transition layer. This increase can directly computed by Eq.~\eqref{eq:int_eng}, but can also be predicted accuratly from the energy leaving the mixed layer. The change in MLAD energy between two range steps of the PE that is excess of the expected cylindrical spreading is
\begin{equation*}
    \Delta \textrm{E}_n = -\left(E_n - E_{n-1} \frac{r_{n-1}}{r_n}\right),
\end{equation*}
where the subscript denotes the $n$-th PE range step. The range ratio term compensates for cyclindrical range spreding.

The energy $\Delta \textrm{E}_n$ is strongly downward diffracted in the transition layer, and only remains inside the depth bounds of integration for a horizontal distance $\Delta x$. The $\Delta x$ is estimated here by assuming this energy leaves the MLAD at horizontal grazing at the SLD, with horizontal wavenumber $k_x = 2 \pi / c_{SLD}$. The horizontal distance traveled by this ray is then
\begin{equation*}
    \Delta x = \int_{z_0}^{z_1} \frac{k_x}{k_z} dz,
\end{equation*}
where $(2 \pi / c(z))^2 = k_x^2 + k_z^2$, and the range averaged sound speed profile is used for $c(z)$. The horizontal distance $\Delta x$ is generally less than 1 km.

Each transition layer energy contribution, $\Delta \textrm{E}_n$ is assumed to add constructivly over the distance $r_n$ to $r_n + \Delta x$. The vertical integration over the transition layer in Eq.~\eqref{eq:int_eng} then behaves like a moving sum of length $N=\textrm{round}(\Delta x / dr)$,
\begin{equation}
    E_{TL} = \sum_{i=n-N}^{n} \Delta E_i \frac{r_i}{r_n}
\end{equation}
Since $\Delta x$ is much less than the 5 km distance used to define a blocking feature, the magnitude of integrated energy lost to a blocking feature is significantly reduced in the transition layer. As an example with a typical $\Delta x$ of 700 m, a 3 dB loss of MLAD energy that is linear over 5 km and starts at -32 dB only increases transition layer energy to approximatly -43 dB over a range of 5 km.

\subsection{Upper ocean acoustic energy statistics}\label{ssec:energy}
\begin{figure}
\includegraphics{../figures/eng_shallow.png}
    \caption{MLAD energy relative to the RI background for decomposed sound speed fields with a source at 40 m depth. Left column is propagation at 400 Hz, right column is at 1 kHz. The top 4 rows are statistics for the complete set of simulated source positions. The bottom 3 rows are statistics for source positions without blocking events (W/O Blocking). The integrated energy for all transmissions in the set are shown as light grey lines, half circles at -10 dB indicate a line moves beyond the plotted scale. Linear regression fits of the mean (bold) and RMS ($\pm$) at 7.5 km and 47.5 km are shown in the two columns right of the line plots.}
    \label{fig:shal_eng}
\end{figure}

The comparison of integrated MLAD energy, Eq.~\eqref{eq:int_eng}, is shown in Fig.~\ref{fig:shal_eng} for 40 m source depth. The 400 and 1000 Hz results are in the left and right columns, respectively. The complete set of source positions is shown in the top section, while the set of source positions without blocking features is shown in the bottom section. The mean and RMS loss are reduced by removing source positions with blocking features. The background field indicates marginal range dependent effects, the highest RMS is only 0.6 dB at 400 Hz and 47.5 km range, and the discussion of range dependence will focus on the dynamic sound speed fields.

The complete set of source positions predicts small mean loss over the source region before 7.5 km at 400 Hz, and small mean gains over this region at 1 kHz. This increased coupling into the mixed layer at higher frequency is also expected in the convergence zones, which also have a wide range of high angle modes, but these are not studied here. The statistics at 47.5 km predict mean loss for all fields and frequencies.

The tilt field has the smallest energy loss at both frequencies. The total loss is more for 400 Hz, although the loss difference between 7.5 and 47.5 km is the same for both frequencies. The RMS at 400 Hz is significantly higher than that at 1 kHz, and this statistic is skewed by high loss events since the upper RMS bound is not realized by any source position. However, the tilt field over the complete transect is often favorable to propagation, with marginal loss or small gains relative to the RI background. The largest mean energy loss at 47.5 km is predicted for the spice field at both frequencies, and almost 1.5 dB more loss is predicted at 400 Hz. The RMS at 47.5 km is higher than the mean at 400 Hz, and while this statistic is loss dominated some fields do show marginal gains of energy. Overall, ocean spice leads to a significantly larger mixed layer energy loss at both study frequencies than tilt  Finally, the observed field has mean statistics like the spice field with significantly higher RMS. This RMS contains the blocking features of both the tilt and spice fields and is skewed by high loss events.

The mixed layer energy statistics of regions without blocking features are shown in the bottom three rows of Fig.~\ref{fig:shal_eng}. For consistency, source positions were removed from this analysis that had blocking features at either frequency. Outside of an overall adjustment in mean and RMS loss, many of the observations from the complete statistics apply to the statistics without blocking. Two notable exceptions are: (1) the mean loss at 47.5 km range is the same or higher at 1 kHz compared to 400 Hz without blocking features, and (2) RMS values are similar across field type and frequency. The higher mean loss at 1 kHz is a reversal from the complete field statistics but is consistent with the increased sensitivity of higher frequency acoustics to small-scale sound speed fluctuations. The consistency in RMS loss simplifies the comparison of the mean values and clearly identifies the stabilizing effect of tilt in the observed fields. The differences in statistics between the observation sets support the discussion of blocking features as distinct from the non-localized loss mechanisms expected to be present for all source ranges.

\begin{figure}
\includegraphics{../figures/eng_shallow_proj.png}
    \caption{Projected MLAD energy relative to the RI background for decomposed sound speed fields and a source at 40 m depth. Presentation of the result follows Fig.~\ref{fig:shal_eng}, but with a larger y-axis range of -20 to 5 dB. The mode used in the energy projection of Eq.~\eqref{eq:proj_eng} is MLM1 at 400 Hz and MLM2 at 1 kHz.}
    \label{fig:shal_proj}
\end{figure}
The vertical structure of MLM energy is analyzed by projecting the energy field onto MLM with Eq.~\eqref{eq:proj_eng}. The MLMs with the largest source amplitudes are used at each frequency, MLM1 at 400 Hz and MLM2 at 1 kHz. The energy projection results are shown in Fig.~\ref{fig:shal_proj} with the same presentation as Fig.~\ref{fig:shal_eng}. The lower bound of these results is increased to -20 dB to show the increased mean and RMS loss of a single mode compared with the total mixed layer energy. Mode coupling can lead to sporadic dips in mode amplitude, but this this local effect is minimized in the statistics with a linear regression fit.

The mean and RMS values for the mode projection show more loss compared with the total energy at all ranges and frequencies except for the background field. The background field shows the same statistics at 400 Hz as the integrated energy, consistent with the mode 1 shape of the background energy seen in Fig.~\ref{fig:decomp_x}(a). The 1 kHz statistics have a mean of approximately -1 dB for the entire range, indicating not all energy is projected onto MLM2. Like the total energy, the background statistics is flat across the transmission range and most range dependent loss is found in the dynamic fields.

The projected and total mode energy show the same trend for the complete source set at 400 Hz, although there is more loss in MLM1 magnitude than total energy loss. The projected statistic has around 1 dB more mean loss for the spice and observed fields at all ranges, and approximately 0.5 dB more RMS for these fields. The increased mean loss compared with the total mixed layer energy indicates that some ranges couple energy into MLM2, an example of which was shown in Fig.~\ref{fig:decomp_x}.

The increase in mean and RMS loss for the mode projection is significantly larger for 1 kHz. The highest mean loss is approximately -10 dB for the spice and observed fields. When combined with RMS values of more than 5 dB, many transmission ranges are expected to have near complete coupling out of MLM2. The coupling serves to maintain energy in mixed layer since there is significantly less total energy loss for these fields at 1 kHz than at 400 Hz. Finally, while the tilt field shows the least energy loss at 47.5 km, the RMS values and are the highest of all fields. The large coupling in the tilt field at 1 kHz frequency may be an indication how this field serves to decrease loss in the observed field total energy.

Without blocking features, the projected and total energy statistics at 400 Hz are nearly identical, indicating extreme blocking events are necessary to create mode coupling. While there is a substantial decrease in the projected statistics at 1 kHz, the mean and RMS loss values are substantially higher than the total mixed layer energy, and mode coupling is expected to be ubiquitous in the mixed layer at 1 kHz.

\begin{figure}
\includegraphics{../figures/eng_deep.png}
    \caption{Mixed layer energy relative to the RI background for a source at 200 m depth. Presentation of the result follows Fig. \ref{fig:shal_eng}, with y-axis bounds of $\pm$20 dB. The statistics are computed to 40 km range and and extended to 47.5 km with a linear regression fit.}
    \label{fig:deep_eng}
\end{figure}
The MLAD energy for a 200 m deep source is shown next in Fig.~\ref{fig:deep_eng}. The width of the convergence zone in the mixed layer is wider compared than the 40 m source case, most apparent as a rise in energy after 40 km in all 400 Hz simulations. While statistics are reported at 47.5 km, this is a linear regression estimate from a fit made between 7.5 to 40 km. There is significant spread in the mixed layer energy for all fields with a deep source, with RMS values between 7 and 11 dB at 47.5 km. Unlike the shallow source scenario, the background field also has a large effect on propagation, especially at 1 kHz. The roughly linear energy progression for all source ranges suggests this is caused by long wavelength mode resonances with the slowly varing background field\cite{colosi21}, and this diffractive effect is significant here when compared with the low energy background.

The deep source scenario requires either diffraction or scatter into the ducted MLAD modes from wide angled energy in the source region. All dynamic fields have significantly higher RMS energy at 7.5 km range than the background field, which indicates the importance of source region coupling to all dynamic fields. With two exceptions, the mean energy at 7.5 km is both positive and uneffected by blocking features. The tilt field at 1 kHz is the only field with negative mean energy, which could be due to increased sensitivity to the strong, small scale, sound speed pertubations at the base of the mixed layer seen in Fig.~\ref{fig:c_fields}(b). The only field where both the mean and RMS energy are correllated significantly with blocking features is spice, where blocking feature positions increase both statistics. The impact of blocking features are simplest for fields where the 7.5 km energy is uncorrelated with blocking features.

At 400 Hz the mean effect of the dynamic fields is a small loss for all fields, with the most loss in the observed field. The large RMS values above 8 dB show significant spread around these means. With blocking features removed, the tilt and observed fields show no mean energy loss in the mixed layer and reduced RMS values. The increase in the mean value is larger than the decrease in RMS value, indicating a significant number of propagation ranges are have more MLAD gain when blocking features are removed. The mean spice field loss increases at 400 Hz with blocking features removed, however. Decreases in the spice field mean and RMS values at 7.5 km with the removal of blocking features indicate this decrease in energy at 47.5 km may be related to a correlation between increased source coupling and source positions with blocking features.

The same behavior described for 400 Hz largly is seen at 1 kHz, with the exceptions of significantly higher RMS in the tilt field and a net increase in MLAD energy with the spice field. The large RMS values for the tilt field field suggest that this result may be dominated by a few high loss events. The higher increase in MLAD energy with the removal of blocking features at 1 kHz relative to 400 Hz further supports the conclusion that increased mode coupling maintains MLAD energy.

\begin{figure}
\includegraphics{../figures/eng_shallow_tl.png}
        \caption{Transition layer energy relative to a 2nd order fit of the RI background for a source at 40 m depth. Presentation of the result follows Fig. \ref{fig:shal_eng}, with y-axis between -20 and 25 dB.}
    \label{fig:eng_tl}
\end{figure}
Finally, the energy in the transition layer is shown for the 40 m source depth scenario in Fig.~\ref{fig:eng_tl}. Unlike the MLAD, the transition layer is strongly downward refracting, and energy only remains in the transition layer over a short distance. For a source in the MLAD, all of this transition layer energy before the first converegence enters from the MLAD. Outside of the source region, energy exits the MLAD both from diffraction and from range dependent energy coupling. The energy in the transion layer is inversely related to these MLAD energy loss, and fields that create large MLAD losses create large temporary energy gains in the transition layer.

\section{Conclusion}\label{sec:conclusion}
A transect of the mixed layer was decomposed to produce fields with separated isopycnal tilt or spice sound speed variation. This decomposition followed Dzieciuch \emph{et al.}\citep{dzieciuch2004} and introduced a linear superposition model for the mixed layer. The different horizontal and vertical structure of the two dynamic fields effected acoustic propagation very differently, both between the two decomposed fields and when compared with the observed field. Acoustic propagation simulations were made at 400 and 1000 Hz, which had an average of one or three mixed layer modes, respectively. There was significant localized range dependent loss observed in PE simulations for many different source positions from blocking features in both the decomposed and observed fields. Blocking features appear for all source positions that transmit through these features, and these effects were most clearly demonstrated at 400 Hz.

A mixed layer energy loss metric was introduced to quantify blocking features, which were identified in all dynamic fields. The was a clear separation in blocking feature location between the tilt and spice field across the transect, suggesting regional variability of these effects.

Mixed layer energy loss statistics were analyzed both with and without blocking features. Blocking features lead to high mean and RMS losses in all fields, and these losses were highest at 400 Hz. The total loss was significantly lower in transmissions without blocking features, and these losses were highest at 1 kHz. This indicated that lower frequencies were more sensitive to discrete blocking events, while higher frequencies were more sensitive to non-localized losses. Mode amplitude analysis showed higher losses for a single mode than the total field, indicating coupling was significant for both frequencies. Little difference between total and MLM1 projected energy was observed at 400 Hz without blocking features, and mode coupling at this frequency is expected only in large and coherent loss events. While the mode coupling was significantly reduced without blocking features at 1 kHz, the loss values for one mode significantly exceeded that for the total mixed layer, and mode coupling is expected to be ubiquitous. This increase in mode coupling at 1 kHz both reduces the loss at blocking features and increases the diffuse loss in ranges without blocking features. This diffuse loss was minimal for the tilt field but was significant for the spice and observed fields.
\bibliographystyle{jasanum2}
\bibliography{eRichards_master}

\end{document}
